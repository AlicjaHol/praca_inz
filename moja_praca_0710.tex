%
% Szablon, v. 2.9
% p.wlaz@pollub.pl
%

% PROSZĘ NIE USUWAĆ
% KOMENTARZY Z~PREAMBUŁY
% JEŻELI KTOŚ WAM WMAWIA, ŻE
% TO PRZYSPIESZY COKOLWIEK
% -- MYLI SIĘ!


\documentclass[12pt]{mwbk}


%%%%%%% marinesy, rozmiary, to warto dopasować do drukarki
\usepackage[a4paper,twoside,top=2.6cm,bottom=2.6cm,inner=3cm,outer=2.6cm]{geometry}

%%%%%%%% polszczyzna
\usepackage[T1]{polski}


%%%%%%%%% sposób kodowania literek w edytorze
\usepackage[utf8]{inputenc}

\usepackage[font=small,labelfont=bf,justification=centering]{caption}


%%%% gdyby ktoś chciał powyklejać z~pedeefa
%%%% teksty za pomocą AcroReadera, to 
%%%% poniższe dwie linijki pomogą w~tym
%%%% Może to być przydatne, gdyby ktoś na podstawie
%%%% elektronicznej wersji chciał przygotować dane do 
%%%% badania antyplagiatowego
%%%% ponieważ prace są w
%%%% tych czasach różnymi
%%%% programami antyplagiatowymi
%%%% proszę absolutni NIE
%%%% USUWAĆ następujących
%%%% dwu linijek
\input glyphtounicode.tex
\pdfgentounicode = 1

%%%%%%%%%%%%%%%%%%%%%%%%%%%%%%%%%%%%%
%%%%% jeśli chcesz by główny tekst oraz wzory matematyczne były
%%%%% składane czcionką typu Times Roman (w~odróżnieniu od standardowej
%%%%% TeXowej, czyli Computer Modern Roman) to linia poniżej
%%%%% ma być 'aktywna', następna nieaktywna, 
%%%%% jeśli zrobisz odwrotnie (pierwsza nieaktywna,
%%%%% druga aktywna) uzyskasz skład czcionką
%%%%% Computer Modern Roman mającą wielu wiernych
%%%%% fanów w~świecie TeXa). Konsekwencją jednak będą zmiany
%%%%% rozmiarów czcionek dla rozdziałó i podrozdziałów - rzecz bez większego
%%%%% znaczenia, wynikająca z pewnych zaszłości historycznych (ComputerModern
%%%%% niegdyś były używane wyłącznie w postaci tzw. bitmap)
%\usepackage{mathptmx} \usepackage{tgtermes}
\usepackage{lmodern}

%%% WSZELKIE ZMIANY W~PREAMBULE RÓB ROZWAŻNIE
%%% NIE JESTEŚ PEWNY/PEWNA ICH EFEKTU TO~SPRAWDŹ 
%%% CZY W~PRGRAMIE ADOBE READER (i~to dokładnie
%%% o~ten program chodzi, nie o~jakikolwiek)
%%% Z~WYNIKOWEGO PLIU PDF DA SIE PRAWIDŁOWO
%%% WYKLEIĆ TEKST Z~POLSKIMI LITERAMI, BEZ KRZAKÓW,
%%%% BEZ DZIWACTW.


%%%%%%%%%%%%%% pozostałe pakiety używane w~pracy, to już zależy od
%%%%%%%%%%%%%% autora, więc może być tego więcej
\usepackage{fancyhdr}
\usepackage{graphicx}
\usepackage{amsmath}
\usepackage{amsthm}
\usepackage{amssymb}
\usepackage{url}
\usepackage{longtable}
\usepackage{array,hhline}



%%%%%%%% hyperref po to by przeglądarka pedeef ukazywala na odwołania
%%%%%%%% prawidłowo skonstruowane za pomocą \ref, \cite i.t.d. jako
%%%%%%%% hiperłącza
\usepackage{hyperref}



%%%%% dla fanów ``profesjonalnych'' tabel w~stylu zachodnich książek

\usepackage{booktabs} \heavyrulewidth=1.5bp \lightrulewidth=0.5bp


%%%%%%%%%%% poniżej uniwersalny sposób na ucywilizowanie znaków 
%%%%%%%%%%% niewiększości, niezależny od pakietu {polski}, ale za to 
%%%%%%%%%%% zależny od {amssymb}, ma tą zaletę, że działa np. z Timesem
%%%%%%%%%%% w matematyce
\let\leq\leqslant\let\le\leq\let\geq\geqslant\let\ge\geq


%%%%%%% jeżeli będziesz chciał włączać do swojej pracy fragmenty programów, 
%% to ponizsza linijka przyda się, jeśli nie - usuń ją

\usepackage{fancyvrb}


%%%%%%%%%%%%%%%%% struktury do tworzenia twierdzeń i~tym podobnych

\theoremstyle{plain}
\newtheorem{twier}{Twierdzenie}[chapter] % pierwsze to nazwa środowiska,
                                      %drugie to wyświetlana nazwa
				% to trzecie w~nawiasie kwadratowym
				% wskazuje numer dolepiony z~lewej do
				% numeru twierdzenia (tu numer
				% 'chapter', 
\newtheorem{lemat}{Lemat}[chapter]

\theoremstyle{definition}
\newtheorem{defi}{Definicja}[chapter]

\theoremstyle{remark}
\newtheorem{uwaga}{Uwaga}[chapter]
\newtheorem{wniosek}{Wniosek}[chapter]

%%%%% więcej możliwości w~dokumentacji amsthm



%%%%%%%%%%%%%%%%%%%%%%%%%%%%%%%%%%%%%%%%%5
%%%%%%%%%%%%%%%%%%%%%%%%%%%%%%%%%%%%%%%%%%
%%%%%%%%% wcięcie akapitowe %%%%%%%%%%%%%%
%%%%%%%%%%%%%%%%%%%%%%%%%%%%%%%%%%%%%%%%%%
%%%%%% ustawić w~zaleceń i~gustu %%%%%%%%%
%%%%%%%%%%%%%%%%%%%%%%%%%%%%%%%%%%%%%%%%%%
%%%%%%%% zalecenie na stronie wydziałowej
%%%%%%%% było 1.25cm i wyglądało jakoś 
%%%%%%%% śmiesznie duże, więc spłoszony nieco
%%%%%%%% wpisałem 1cm, ale uważny czytelnik już
%%%%%%%% zapewne się domyśli, że podmiana napisu 
%%%%%%%% =1cm na =1.25cm sprawi, że wcięcia na początku
%%%%%%%% akapitu ustawią się na (nieco przydużą)
%%%%%%%% wartość 1.25cm 

\parindent=1cm



%%%%%%%%%%%%%%%%%%%%%%%%%%%%%%%%
%%%%% tu pewne poluzowanie rozmieszczenia elementów tabelek
%%%%% możecie sobie poeksperymentować, by dopasować do swych
%%%%% gustów, a przede wszystkim gustów promotorów (promotorek)
  \tabcolsep=4mm          
  %\renewcommand\arraystretch{1.3}
%%%%%%%%%%%%%%%%%%%%%%%%%%%%%%%%%%



%%%%%%%%% teraz żywa pagina (aka 'running headline') i~numerowanie stron
%%%%%%%%%%%%%%%%%%%%%%%%%%%%%%%%%%%%%%%%%%%%%%%%%%%%%%%%%%%%%%%%%%%%%%%%
%%%%%na górze mają być śródtytuły, na dole (po stronie zewneętrznej)
%%%%%numery stron. Poszedłem kapkę dalej i~na stronach ropoczynających
%%%%%rozdział nie ma paginy (górki).
%%%%% Oczywiście jeśli ostatnia strona
%%%%% jest pusta (uzupełnia jeno parzystość) to tam żadnej stopki ani 
%%%%% górki byc mnie może - ma być pusta.
%%%%%%%%%%%%%%%%%%%%%%%%%%%%
\pagestyle{fancy}
\fancyhead{}% oczyszczenie
\fancyhead[RO]{\rightmark} %% na nieparzystych 'podległe' śródtytuły
\fancyhead[LE]{\leftmark} %% na parzystych 'ważniejsze'
\fancyfoot{}% oczyszczenie
\fancyfoot[RO,LE]{\arabic{page}}  %% numer na dole (po prawej na
%% nieparzystych, po lewej na parzystych)
\renewcommand\headrulewidth{0.4pt} %%% cienka hrulka oddzielająca paginę
                                    %%% od kolumny tekstu
\fancypagestyle{closing}{%%%%%% to styl dla stron zamykających rozdział
\fancyhead{}% oczyszczenie
\fancyhead[RO]{\rightmark} %% na nieparzystych 'podległe'
\fancyhead[LE]{\leftmark} %% na parzystych 'ważniejsze'
\fancyfoot{}% oczyszczenie
\fancyfoot[RO,LE]{\arabic{page}}  %% numer na dole (po prawej na
                                  %% powyższą linijkę usuń jeśli nie
				  %% chcesz numerów na niepełnych
				  %% kolumnach (zamykających rozdział)
\renewcommand\headrulewidth{0.4pt}
}
\fancypagestyle{opening}{%%% styl stron rozpoczynających rozdział
\fancyhead{}% oczyszczenie
\fancyfoot{}% oczyszczenie
\fancyfoot[RO,LE]{\arabic{page}}  %% numer na dole (po prawej na
\renewcommand\headrulewidth{0pt}
}
\fancypagestyle{plain}{%%%% styl zwykły, niektóre konstrukcje
                       %%%% (typu \titlepage, którego ja tu nie używam
                       %%%% ale może są jakieś inne o których nawet nie chce 
                       %%% mi się myśleć, więc dla spokoju robię to po swojemu
\fancyhead{}% oczyszczenie
\fancyfoot{}% oczyszczenie
\fancyfoot[RO,LE]{\arabic{page}}  %% numer na dole (po prawej na
\renewcommand\headrulewidth{0pt}
}

%%%%%%%%%%%%%%%%%%%%%%%%%%%%%%%%%5
%%%%%%%%%%%%%%%%%%%%%%%%%%%%%%%%%%
%%% lekka modyfikcja 'markow' do paginy
%%% uznalem, ze jesli ktos nie da \section (np we wstepnie czy
%%% podsumowaniu to niech na obu sronach w~paginie pojawia sie tytuł
%%% chaptera, bo standardowo, to na nieparzystej stronie w takiej sytuacji
%%% nad górną linią ziałaby pustka, co mogłoby wprowadzać konsternację
\makeatletter
    \def\chaptermark#1{%
      \markboth{%
        \ifHeadingNumbered
     \if@mainmatter
     \@chapapp\
            \thechapter.\enspace
          \fi
        \fi
        #1}{%
        \ifHeadingNumbered
     \if@mainmatter
     \@chapapp\
            \thechapter.\enspace
          \fi
        \fi
        #1%
	}}%
    \def\sectionmark#1{%
      \markright{%
        \ifHeadingNumbered \thesection.\enspace \fi
        #1}}
%%%%%%%%%%%%%%%%%%%%%%%%%%%%%%%%%%%%%%%%%%%%%%%
%%%%%%%%%%%%%%%%%%%%%%%%%%%%%%%%%%%%%%%%%%%%%%%%
%%%%%%%%%%%% wielkości czcionek dla chapter i~section
%%%%%%%%%%%% 16 dla rozdziału, 14 dla podrozdziału - te domyślne
%%%%%%%%%%%% w klasie mwbk były całkiem ładne, ale żeby nie było
%%%%%%%%%%%% że nie potrafię ustawić
%%%%%%%%%%%%%%%%%%%%%%%%%%%%%%%%%%%%%%%%%%%%%%%%%%%
\SetSectionFormatting[breakbefore,wholewidth]{chapter}
        {0\p@}
        {\FormatRigidChapterHeading{6.4\baselineskip}{12\p@}%
	{\large\@chapapp\space}{\fontsize{16}{19}\selectfont}}
        {1.6\baselineskip}
\SetSectionFormatting{section}
        {24\p@\@plus5\p@\@minus2\p@}
	{\FormatHangHeading{\fontsize{14}{16}\selectfont}}
        {10\p@\@plus3\p@}
\makeatother	



%%%%%%%%%%%%%%%%%%%%%%%%%%%%%%%%%%%%%%%%%%%%%%
%%%%%%%%%%%%%%%%%%%%%%%%%%%%%%%%%%%%%%%%%%%%%%
%%%%%%%%%%%%%% jakies inne pomocnicze definicje, ja na przykład lubię
% \R
%%%%%%%%%%%%%%%%%%%%%%%5
%%%%%%%%%%%%%%%%%%%%%%%
%%%% tak naprawdę są t potrzebne tylko po to
%%%% by zadziałały przykłady poniżej w tekście
%%%% które w sposób dość losowy zostały 
%%%% pobrane z jakichś moich starych plików
%%%%%%%%%%%%%%%%%%%%%%%%%%%%%%%%%%
%%%%%%%%%%%%%%%%%%%%%%%%%%%%%%%%%%%
%%%% w realnej pracy te poniższe śmieci możecie oczywiście
%%%% usunąć
%%%%%%%%%%%%%%%%%%%%%%%%%%%%
\newcommand\R{\mathbb{R}}
\newcommand{\ff}{\mathbf{f}}
\newcommand{\hh}{\mathbf{h}}
\newcommand{\xx}{\mathbf{x}}
\newcommand{\yy}{\mathbf{y}}
\newcommand{\zz}{\mathbf{z}}
\newcommand{\gggg}{\mathbf{g}}
\newcommand{\skalar}[2]{\pmb{\langle}#1,#2\pmb{\rangle}}
%%%%%%%%%%%% koniec tych dodatkowych definicji

%%%%%% trocę więcej ``luzu'' przy rozmieszczaniu {fgur} i~{table}

 \renewcommand{\topfraction}{0.9}	% max fraction of floats at top
    \renewcommand{\bottomfraction}{0.8}	% max fraction of floats at bottom
    %   Parameters for TEXT pages (not float pages):
    \setcounter{topnumber}{2}
    \setcounter{bottomnumber}{2}
    \setcounter{totalnumber}{4}     % 2 may work better
    \setcounter{dbltopnumber}{2}    % for 2-column pages
    \renewcommand{\dbltopfraction}{0.9}	% fit big float above 2-col. text
    \renewcommand{\textfraction}{0.07}	% allow minimal text w. figs
    %   Parameters for FLOAT pages (not text pages):
    \renewcommand{\floatpagefraction}{0.7}	% require fuller float pages
    % N.B.: floatpagefraction MUST be less than topfraction !!
    \renewcommand{\dblfloatpagefraction}{0.7}	% require fuller float pages
    % remember to use [htp] or [htpb] for placement

    
%%% DWA proste polecenia służące do ujednolicenia podawania źródeł przy rysunkach i~tabelkach    
    
    \newcommand\zrodlo[1]{\par\vspace{-3mm}{\small\textit{Źródło: }#1 }}
    \newcommand\zrodlotab[1]{{\par\vspace{2mm}\small\textit{Źródło: }#1 }}

\raggedbottom   %%% to znaczy, że nie będzie siłowego wyrównywania typowych
                %%     stron do jednakowej wysokości

\linespread{1.3}

\begin{document}

%%%%%%%%%%%%%%%%%%%%%%%%%%%%%%%%%%%%%%%%%
%%%%%%%%%%%%%%%%%%%%%%%%%%%%%%%%%%%%%%%%%
%%%%%%%% STRONA TYTUŁOWA %%%%%%%%%%%%%%%%

\thispagestyle{empty}  % tu wszak nie chcemy żadnej numeracji stron


%%%%%%%%%%%%%%%%%%%%%%%%%%%%%%%%%%%%%%%%%%%%%%%%%%%%%%%%%%%%%%%
%%%%%tytuły definiuje jako makrodefinicje, gdyż zamierzam je%%%
%%%%%powtórzyć na stronie ze streszczeniami, to nic nie boli%%%
%%%%%a gwarantuje, że będą one takie same, i~tak ma być.%%%%%%%
%%%%%%%%%%%%%%%%%%%%%%%%%%%%%%%%%%%%%%%%%%%%%%%%%%%%%%%%%%%%%%%
\newcommand\tytul{Zastosowanie modeli mieszanych w analizie rozwoju pandemii wywołanej wirusem Covid-19 na świecie}

\newcommand\tytulangielski{The use of mixed-effects models in the analysis of the Covid-19 pandemic in the world}


\begin{center}


{\large \bf POLITECHNIKA LUBELSKA}

{\bf WYDZIAŁ PODSTAW TECHNIKI}

\emph{Kierunek:} MATEMATYKA

%%% BEZ SPEC.!!! \emph{Specjalność:} Matematyka w~finansach i~ubezpieczeniach

\vfill %%%% \vfill to taki rozpychacz w pionie, pcha ile mu pozwolą
     

\includegraphics[width=3.5cm]{rys/logopl}

\vfill

\textbf{Praca inżynierska}

\vfill
\vfill
\vfill

\large
\tytul

\vfill

\emph{\tytulangielski}


\vfill
\vfill
\vfill
\vfill
\vfill

\begin{tabular}[t]{l}
\emph{Praca wykonana pod kierunkiem:}
\\
dra Dariusza Majerka
\end{tabular}
\hfill
\begin{tabular}[t]{l}
	\emph{Autor:}
\\
Alicja Hołowiecka\\
nr albumu: 89892 
\end{tabular}

\vfill
\vfill
\vfill

\textbf{Lublin 2020}

\end{center}


%%%%% koniec tytułów


%%%%%%%%%%%%%%%%%%%%%%%5
%%%%%%%%%%%%%%%%%%%%%%
%%% teraz spis treści
%%%%%%%%%%%%%%%%%%%%%
%%% pamiętaj! po jakiejkolwiek zmianie w tekście
%%% która wpływa na zmianę spisu treści, spis będzie dobry co najmniej
%%% po dwóch przebiegach latexa - to samo dotyczy odwołań do wzorów i literatury
%%% ogólnie to przed wydrukiem warto przelatexować o jedne raz więcej niż
%%% to się wydaje konieczne, no chyba że korzystamy z funkcji typu BUILD
%%% w zintegrowanym systemie wspomagającym TeX, BUILD powinien takie sprawy 
%%% wziąć pod uwagę

\tableofcontents


\chapter*{Wstęp}




Pandemia choroby COVID-19 jest wydarzeniem, które wstrząsnęło całym światem w roku 2019. Właściwie nikt chyba nie może powiedzieć, że nie poczuł się dotknięty przez sytuację związaną z rozprzestrzenianiem się wirusa. Pierwsze przypadki pojawiły się pod koniec 2019 roku we wschodnich Chinach, w mieście Wuhan. Na początku 2020 roku chorowali już obywatele większości państw na świecie. Na moment pisania tej pracy, sytuacja nadal nie jest opanowana i nie wiadomo, jak się rozwinie.

Biorąc to pod uwagę, tym ważniejszy wydaje się temat poruszany w tej pracy. Wiele jednostek naukowych podejmuje próby znalezienia odpowiedniego modelu, aby przewidzieć rozwój pandemii. Przedstawione w tej pracy modele mieszane co prawda nie pozwalają na dokładną predykcję, ale są dobrym narzędziem, aby odkryć, które czynniki mają wpływ na rozwój pandemii w przeciętnym kraju.











\chapter{Teoretyczne podstawy badań własnych}
W tej części pracy przedstawimy metody matematyczne, które zostaną użyte w części praktycznej tej pracy. Zgodnie z tematem, będą to głównie modele mieszane.
\section{Modele liniowe}
Na początek przypomnimy podstawowe wiadomości o modelach liniowych.

Model regresji prostej ma postać 
$$y=x \beta_1+\beta_0 + \varepsilon$$

gdzie oszacowania parametrów $\beta_1$, $\beta_0$ obliczamy następująco:

$$\hat{\beta_1}=\frac{Cov(x,y)}{Var(x)},$$
$$\hat{\beta_0}=\overline{y}-\overline{x}\hat{\beta_1}.$$

Model interpretujemy w ten sposób, że jeżeli zmienna $x$ wzrośnie o 1, to zmienna $y$ zmieni się o $\beta_1$.
\subsection{Metody estymacji parametrów modelu liniowego}
\begin{enumerate}
	\item Metoda najmniejszych kwadratów, OLS (ang. \emph{Ordinary Least Squares}) - w metodzie tej minimalizujemy błąd kwadratowy, czyli sumę kwadratów reszt, którą oznaczamy RSS (ang. \emph{Residual Sum of Squares}).
	
	$$RSS= \sum_{i=1}^{n}(y_i-\hat{y_i})^2$$
	
	Twierdzenie Gaussa-Markowa: taki estymator jest BLUE (Best Linear Unbiased Estimator), przy odpowiednich założeniach.
	
	\item Metoda największej wiarogodności, ML (ang.\emph{Maximum Likelihood}) polega na maksymalizacji wartości funkcji prawdopodobieństwa ze względu na $\beta$ (w praktyce maksymalizujemy zwykle logarytm z tej funkcji)
	
	$$\hat{\sigma}^{2}_{ML}=RSS/n$$
	
	Estymując $\sigma^2$, maksymalizujemy funkcję wiarogodności zarówno ze względu na $\beta$, jak i $\sigma^2$.
	
	Estymatory uzyskane tą metodą są asymptotycznie nieobciążone.
	
	\item Resztowa metoda największej wiarogodności, REML (ang. \emph{Residual/Restricted Maximum Likelihood Method}) - z estymacji parametru $\sigma^2$ usuwamy wpływ parametrów zakłócających $\beta$.
	
	$$\hat{\sigma}^2_{REML}=RSS/(n-p)$$
	
	Estymatory uzyskane tą metodą są nieobciążone \cite{biecek}.
\end{enumerate}

\subsection{Badanie istotności parametrów}

$$H_0: \beta_i = 0$$

\section{Modele mieszane} 
W powyżej opisanych modelach liniowych z efektami stałymi zakładamy niezależność kolejnych pomiarów, dlatego nie są to odpowiednie modele, kiedy mamy np. kilka pomiarów dla pojedynczego elementu. W takim przypadku możemy użyć modeli liniowych z efektami mieszanymi (stałymi i losowymi), które krótko nazywamy modelami mieszanymi.

Modeli mieszanych używamy w przypadu powtarzanych pomiarów bądź w przypadku hierarchicznej lub zagnieżdżonej struktury. Takie dane charakteryzują się korelacją między obserwacjami z tej samej grupy, co nie pozwala na użycie modelu liniowego z efektami stałymi. Dlatego do modelu wprowadza się czynnik losowy. 

Czynnik stały jest pewnym parametrem, którego wartość estymujemy na podstawie próbki, natomiast czynnik losowy jest zmienną losową, dla której próbujemy oszacować parametry jej rozkładu \cite{faraway}.
	
Przykładową sytuacją, gdzie możemy użyć modelu mieszanego, jest badanie działania leku na grupie pacjentów, gdzie dokonujemy kilku pomiarów na danym pacjencie. W tym przypadku nie interesuje nas konkretny pacjent, ale raczej wpływ leku na przeciętnego pacjenta. Dodatkowo, traktujemy pacjentów jako losowo wybranych. Podejście modelu mieszanego będzie polegało na potraktowaniu wpływu pacjenta jako czynnik zakłócający. 

Rozważamy model postaci
$$y=X\beta +Z u + \varepsilon$$
gdzie $X$ - macierz zmiennych będących efektami stałymi, $Z$ - macierz zmiennych będących efektami losowymi, $\beta$ to wektor nieznanych efektów stałych, $\varepsilon \sim \mathcal{N}(0, \sigma^2 I_{n\times n})$ to zakłócenie losowe, a $u \sim \mathcal{N} (0, \sigma^2D)$ to wektor zmiennych losowych odpowiadających efektom losowym \cite{biecek}.

Znając $D$, możemy estymować parametry $\beta$ uogólnioną metodą najmniejszych kwadratóW. Do estymowania nieznanego $D$ możemy użyć np. metodą największej wiarogodności.

\subsection{Metody estymacji}

Do oceny wartości parametrów modelu mieszanego można stosować metody ML (Największej Wiarogodności) oraz REML (Resztowej Największej Wiarogodności), wspomniane w tej pracy przy okazji modeli liniowych. W przypadku modeli mieszanych obydwoma metodami możemy uzyskać estymatory obciążone, ale to obciążenie jest zazwyczaj mniejsze w przypadku estymatorów uzyskanych metodą REML.

Różnica między metodą REML i ML polega na tym, że w metodzie REML najpierw usuwamy wpływ efektów stałych.

\subsection{Badanie istotności parametrów}

$$H_0: \sigma^2_j=0$$

Te same metody co dla efektów stałych

\chapter{Badania własne}
\section{Probelmy szczegółowe i cele}
\subsection{Hipoteza 1}
Wpływ kraju (efektu losowego) jest większy niż wpływ czasu (czynnika stałego) w modelu mieszanym.
\section{Zbiór danych i jego wstępne przygotowanie}

Zbiór danych pochodzi z witryny internetowej Our World In Data \cite{owid}, gdzie dane zostały zebrane z różnych źródeł, m. in. ze Światowej Organizacji Zdrowia (WHO) oraz Europejskiego Centrum ds. Zapobiegania i Kontroli Chorób (ECDC). W zbiorze znajduje się 210 krajów, dane dotyczące terytoriów międzynarodowych oraz łącznie dla całego świata. Mamy ponad 40 kolumn z różnymi parametrami - w dalszej części pracy opiszemy, które zmienne będą przez nas użyte.

W zbiorze znajdowało się wiele braków danych. W przypadku zmiennych takich jak liczba zachorowań, zostały one wypełnione poprzez przepisanie danych z poprzedniego dnia. Dla każdego kraju zostały usunięte dane sprzed rozpoczęcia się epidemii na jego terytorium (\texttt{total cases=0}), dni są numerowane kolejnymi liczbami całkowitymi.
\section{Modele}

Pierwszy model, jaki przetestujemy, to zależność liczby zachorowań na milion mieszkańców w zależności od czasu, gdzie efektem losowym jest kraj.

\begin{verbatim}
mod <- lme(total_cases_per_million~time, random = ~1|location, data = covid)
\end{verbatim}

\begin{figure}[htbp]
	\centering
	\includegraphics{plot_all_countries.png}
	\caption{Wykres przedstawiający rozwój pandemii we wszystkich krajach, tak, wiem, że nic na nim nie widać}
	\label{fig:plot_all}
	\zrodlo{Opracowanie własne}
\end{figure}


\section{Dyskusja wyników}



\chapter*{Podsumowanie i~wnioski}

Badania jednoznacznie pokazują, że efekt kraju jako czynnika zakłócającego jest bardzo istotny, bardziej niż jakikolwiek inny czynnik stały (np. czas).

Na to, co jest nazywane w tej pracy ,,efektem kraju'', składa się tak naprawdę wiele innych czynników, m. in. gęstość zaludnienia, sytuacja ekonomiczna danego kraju, odsetek osób z chorobami towarzyszącymi, rozkład wieku, jak również przyjęta strategia walki z koronawirusem, na którą z kolei składają się m. in. liczba wykonywanych testów, przepisy w sprawie zamykania szkół, miejsc publicznych, ograniczenie kontaktów międzyludzkich, i wiele innych.

W mojej pracy nie zajmowałam się badaniem, w jaki sposób te czynniki wpływają na wzrost lub spadek liczby zachorowań, chcę jedynie zasygnalizować, że mogą być istotne, skoro wykazany został wpływ efektu kraju na liczbę zachorowań.

\begin{thebibliography}{99}

\bibitem{biecek} Przemysław Biecek, \emph{Analiza danych z programem R. Modele liniowe z efektami stałymi, losowymi i mieszanymi}, Wydawnictwo Naukowe PWN, Wydanie II, Warszawa 2013
\bibitem{faraway} Julian J. Faraway, \emph{Extending the Linear Model with R. Generalized Linear, Mixed Effects and Nonparametric Regression Models. Second Edition}, CRC Press Taylor \& Francis  Group, 2016

\bibitem{owid} \url{https://ourworldindata.org/coronavirus}

\end{thebibliography}



\listoffigures

\listoftables


\chapter*{Załączniki}
\begin{enumerate}
%\item Oświadczenie o oryginalności pracy i możliwości jej wykorzystania. 
%\item Opinia promotora na temat oryginalności pracy oraz w~sprawie dopuszczenia do obrony pracy dyplomowej.
%\item Potwierdzenie analizy antyplagiatowej.
\item Płyta CD z niniejszą pracą w wersji elektronicznej.
\end{enumerate}




\chapter*{Streszczenie (Summary)}

\bigskip
\bigskip

\begin{center}
  \textbf{\tytul}
\end{center}

W tej pracy przedstawione są pojęcia związane z modelami liniowymi z efektami stałymi i losowymi. Następnie opisane są badania własne na zbiorze danych dotyczącym rozprzestrzeniania się choroby COVID-19 w różnych krajach na świecie.


\bigskip

\begin{center}
  \textbf{\textit{\tytulangielski}}
\end{center}



\selecthyphenation{english}
{\it
In this paper, concepts related to linear models with fixed and random effects are presented. Then, our own research is described on the dataset on the spread of COVID-19 in various countries around the world.
}

\end{document}

