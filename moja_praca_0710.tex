%
% Szablon, v. 2.9
% p.wlaz@pollub.pl
%

% PROSZĘ NIE USUWAĆ
% KOMENTARZY Z~PREAMBUŁY
% JEŻELI KTOŚ WAM WMAWIA, ŻE
% TO PRZYSPIESZY COKOLWIEK
% -- MYLI SIĘ!


\documentclass[12pt]{mwbk}


%%%%%%% marinesy, rozmiary, to warto dopasować do drukarki
\usepackage[a4paper,twoside,top=2.6cm,bottom=2.6cm,inner=3cm,outer=2.6cm]{geometry}

%%%%%%%% polszczyzna
\usepackage[T1]{polski}


%%%%%%%%% sposób kodowania literek w edytorze
\usepackage[utf8]{inputenc}

\usepackage[font=small,labelfont=bf,justification=centering]{caption}


%%%% gdyby ktoś chciał powyklejać z~pedeefa
%%%% teksty za pomocą AcroReadera, to 
%%%% poniższe dwie linijki pomogą w~tym
%%%% Może to być przydatne, gdyby ktoś na podstawie
%%%% elektronicznej wersji chciał przygotować dane do 
%%%% badania antyplagiatowego
%%%% ponieważ prace są w
%%%% tych czasach różnymi
%%%% programami antyplagiatowymi
%%%% proszę absolutni NIE
%%%% USUWAĆ następujących
%%%% dwu linijek
\input glyphtounicode.tex
\pdfgentounicode = 1

%%%%%%%%%%%%%%%%%%%%%%%%%%%%%%%%%%%%%
%%%%% jeśli chcesz by główny tekst oraz wzory matematyczne były
%%%%% składane czcionką typu Times Roman (w~odróżnieniu od standardowej
%%%%% TeXowej, czyli Computer Modern Roman) to linia poniżej
%%%%% ma być 'aktywna', następna nieaktywna, 
%%%%% jeśli zrobisz odwrotnie (pierwsza nieaktywna,
%%%%% druga aktywna) uzyskasz skład czcionką
%%%%% Computer Modern Roman mającą wielu wiernych
%%%%% fanów w~świecie TeXa). Konsekwencją jednak będą zmiany
%%%%% rozmiarów czcionek dla rozdziałó i podrozdziałów - rzecz bez większego
%%%%% znaczenia, wynikająca z pewnych zaszłości historycznych (ComputerModern
%%%%% niegdyś były używane wyłącznie w postaci tzw. bitmap)
%\usepackage{mathptmx} \usepackage{tgtermes}
\usepackage{lmodern}

%%% WSZELKIE ZMIANY W~PREAMBULE RÓB ROZWAŻNIE
%%% NIE JESTEŚ PEWNY/PEWNA ICH EFEKTU TO~SPRAWDŹ 
%%% CZY W~PRGRAMIE ADOBE READER (i~to dokładnie
%%% o~ten program chodzi, nie o~jakikolwiek)
%%% Z~WYNIKOWEGO PLIU PDF DA SIE PRAWIDŁOWO
%%% WYKLEIĆ TEKST Z~POLSKIMI LITERAMI, BEZ KRZAKÓW,
%%%% BEZ DZIWACTW.


%%%%%%%%%%%%%% pozostałe pakiety używane w~pracy, to już zależy od
%%%%%%%%%%%%%% autora, więc może być tego więcej
\usepackage{fancyhdr}
\usepackage{graphicx}
\usepackage{amsmath}
\usepackage{amsthm}
\usepackage{amssymb}
\usepackage{url}
\usepackage{longtable}
\usepackage{array,hhline}



%%%%%%%% hyperref po to by przeglądarka pedeef ukazywala na odwołania
%%%%%%%% prawidłowo skonstruowane za pomocą \ref, \cite i.t.d. jako
%%%%%%%% hiperłącza
\usepackage{hyperref}



%%%%% dla fanów ``profesjonalnych'' tabel w~stylu zachodnich książek

\usepackage{booktabs} \heavyrulewidth=1.5bp \lightrulewidth=0.5bp


%%%%%%%%%%% poniżej uniwersalny sposób na ucywilizowanie znaków 
%%%%%%%%%%% niewiększości, niezależny od pakietu {polski}, ale za to 
%%%%%%%%%%% zależny od {amssymb}, ma tą zaletę, że działa np. z Timesem
%%%%%%%%%%% w matematyce
\let\leq\leqslant\let\le\leq\let\geq\geqslant\let\ge\geq


%%%%%%% jeżeli będziesz chciał włączać do swojej pracy fragmenty programów, 
%% to ponizsza linijka przyda się, jeśli nie - usuń ją

\usepackage{fancyvrb}


%%%%%%%%%%%%%%%%% struktury do tworzenia twierdzeń i~tym podobnych

\theoremstyle{plain}
\newtheorem{twier}{Twierdzenie}[chapter] % pierwsze to nazwa środowiska,
                                      %drugie to wyświetlana nazwa
				% to trzecie w~nawiasie kwadratowym
				% wskazuje numer dolepiony z~lewej do
				% numeru twierdzenia (tu numer
				% 'chapter', 
\newtheorem{lemat}{Lemat}[chapter]

\theoremstyle{definition}
\newtheorem{defi}{Definicja}[chapter]

\theoremstyle{remark}
\newtheorem{uwaga}{Uwaga}[chapter]
\newtheorem{wniosek}{Wniosek}[chapter]

%%%%% więcej możliwości w~dokumentacji amsthm



%%%%%%%%%%%%%%%%%%%%%%%%%%%%%%%%%%%%%%%%%5
%%%%%%%%%%%%%%%%%%%%%%%%%%%%%%%%%%%%%%%%%%
%%%%%%%%% wcięcie akapitowe %%%%%%%%%%%%%%
%%%%%%%%%%%%%%%%%%%%%%%%%%%%%%%%%%%%%%%%%%
%%%%%% ustawić w~zaleceń i~gustu %%%%%%%%%
%%%%%%%%%%%%%%%%%%%%%%%%%%%%%%%%%%%%%%%%%%
%%%%%%%% zalecenie na stronie wydziałowej
%%%%%%%% było 1.25cm i wyglądało jakoś 
%%%%%%%% śmiesznie duże, więc spłoszony nieco
%%%%%%%% wpisałem 1cm, ale uważny czytelnik już
%%%%%%%% zapewne się domyśli, że podmiana napisu 
%%%%%%%% =1cm na =1.25cm sprawi, że wcięcia na początku
%%%%%%%% akapitu ustawią się na (nieco przydużą)
%%%%%%%% wartość 1.25cm 

\parindent=1cm



%%%%%%%%%%%%%%%%%%%%%%%%%%%%%%%%
%%%%% tu pewne poluzowanie rozmieszczenia elementów tabelek
%%%%% możecie sobie poeksperymentować, by dopasować do swych
%%%%% gustów, a przede wszystkim gustów promotorów (promotorek)
  \tabcolsep=4mm          
  %\renewcommand\arraystretch{1.3}
%%%%%%%%%%%%%%%%%%%%%%%%%%%%%%%%%%



%%%%%%%%% teraz żywa pagina (aka 'running headline') i~numerowanie stron
%%%%%%%%%%%%%%%%%%%%%%%%%%%%%%%%%%%%%%%%%%%%%%%%%%%%%%%%%%%%%%%%%%%%%%%%
%%%%%na górze mają być śródtytuły, na dole (po stronie zewneętrznej)
%%%%%numery stron. Poszedłem kapkę dalej i~na stronach ropoczynających
%%%%%rozdział nie ma paginy (górki).
%%%%% Oczywiście jeśli ostatnia strona
%%%%% jest pusta (uzupełnia jeno parzystość) to tam żadnej stopki ani 
%%%%% górki byc mnie może - ma być pusta.
%%%%%%%%%%%%%%%%%%%%%%%%%%%%
\pagestyle{fancy}
\fancyhead{}% oczyszczenie
\fancyhead[RO]{\rightmark} %% na nieparzystych 'podległe' śródtytuły
\fancyhead[LE]{\leftmark} %% na parzystych 'ważniejsze'
\fancyfoot{}% oczyszczenie
\fancyfoot[RO,LE]{\arabic{page}}  %% numer na dole (po prawej na
%% nieparzystych, po lewej na parzystych)
\renewcommand\headrulewidth{0.4pt} %%% cienka hrulka oddzielająca paginę
                                    %%% od kolumny tekstu
\fancypagestyle{closing}{%%%%%% to styl dla stron zamykających rozdział
\fancyhead{}% oczyszczenie
\fancyhead[RO]{\rightmark} %% na nieparzystych 'podległe'
\fancyhead[LE]{\leftmark} %% na parzystych 'ważniejsze'
\fancyfoot{}% oczyszczenie
\fancyfoot[RO,LE]{\arabic{page}}  %% numer na dole (po prawej na
                                  %% powyższą linijkę usuń jeśli nie
				  %% chcesz numerów na niepełnych
				  %% kolumnach (zamykających rozdział)
\renewcommand\headrulewidth{0.4pt}
}
\fancypagestyle{opening}{%%% styl stron rozpoczynających rozdział
\fancyhead{}% oczyszczenie
\fancyfoot{}% oczyszczenie
\fancyfoot[RO,LE]{\arabic{page}}  %% numer na dole (po prawej na
\renewcommand\headrulewidth{0pt}
}
\fancypagestyle{plain}{%%%% styl zwykły, niektóre konstrukcje
                       %%%% (typu \titlepage, którego ja tu nie używam
                       %%%% ale może są jakieś inne o których nawet nie chce 
                       %%% mi się myśleć, więc dla spokoju robię to po swojemu
\fancyhead{}% oczyszczenie
\fancyfoot{}% oczyszczenie
\fancyfoot[RO,LE]{\arabic{page}}  %% numer na dole (po prawej na
\renewcommand\headrulewidth{0pt}
}

%%%%%%%%%%%%%%%%%%%%%%%%%%%%%%%%%5
%%%%%%%%%%%%%%%%%%%%%%%%%%%%%%%%%%
%%% lekka modyfikcja 'markow' do paginy
%%% uznalem, ze jesli ktos nie da \section (np we wstepnie czy
%%% podsumowaniu to niech na obu sronach w~paginie pojawia sie tytuł
%%% chaptera, bo standardowo, to na nieparzystej stronie w takiej sytuacji
%%% nad górną linią ziałaby pustka, co mogłoby wprowadzać konsternację
\makeatletter
    \def\chaptermark#1{%
      \markboth{%
        \ifHeadingNumbered
     \if@mainmatter
     \@chapapp\
            \thechapter.\enspace
          \fi
        \fi
        #1}{%
        \ifHeadingNumbered
     \if@mainmatter
     \@chapapp\
            \thechapter.\enspace
          \fi
        \fi
        #1%
	}}%
    \def\sectionmark#1{%
      \markright{%
        \ifHeadingNumbered \thesection.\enspace \fi
        #1}}
%%%%%%%%%%%%%%%%%%%%%%%%%%%%%%%%%%%%%%%%%%%%%%%
%%%%%%%%%%%%%%%%%%%%%%%%%%%%%%%%%%%%%%%%%%%%%%%%
%%%%%%%%%%%% wielkości czcionek dla chapter i~section
%%%%%%%%%%%% 16 dla rozdziału, 14 dla podrozdziału - te domyślne
%%%%%%%%%%%% w klasie mwbk były całkiem ładne, ale żeby nie było
%%%%%%%%%%%% że nie potrafię ustawić
%%%%%%%%%%%%%%%%%%%%%%%%%%%%%%%%%%%%%%%%%%%%%%%%%%%
\SetSectionFormatting[breakbefore,wholewidth]{chapter}
        {0\p@}
        {\FormatRigidChapterHeading{6.4\baselineskip}{12\p@}%
	{\large\@chapapp\space}{\fontsize{16}{19}\selectfont}}
        {1.6\baselineskip}
\SetSectionFormatting{section}
        {24\p@\@plus5\p@\@minus2\p@}
	{\FormatHangHeading{\fontsize{14}{16}\selectfont}}
        {10\p@\@plus3\p@}
\makeatother	



%%%%%%%%%%%%%%%%%%%%%%%%%%%%%%%%%%%%%%%%%%%%%%
%%%%%%%%%%%%%%%%%%%%%%%%%%%%%%%%%%%%%%%%%%%%%%
%%%%%%%%%%%%%% jakies inne pomocnicze definicje, ja na przykład lubię
% \R
%%%%%%%%%%%%%%%%%%%%%%%5
%%%%%%%%%%%%%%%%%%%%%%%
%%%% tak naprawdę są t potrzebne tylko po to
%%%% by zadziałały przykłady poniżej w tekście
%%%% które w sposób dość losowy zostały 
%%%% pobrane z jakichś moich starych plików
%%%%%%%%%%%%%%%%%%%%%%%%%%%%%%%%%%
%%%%%%%%%%%%%%%%%%%%%%%%%%%%%%%%%%%
%%%% w realnej pracy te poniższe śmieci możecie oczywiście
%%%% usunąć
%%%%%%%%%%%%%%%%%%%%%%%%%%%%
\newcommand\R{\mathbb{R}}
\newcommand{\ff}{\mathbf{f}}
\newcommand{\hh}{\mathbf{h}}
\newcommand{\xx}{\mathbf{x}}
\newcommand{\yy}{\mathbf{y}}
\newcommand{\zz}{\mathbf{z}}
\newcommand{\gggg}{\mathbf{g}}
\newcommand{\skalar}[2]{\pmb{\langle}#1,#2\pmb{\rangle}}
%%%%%%%%%%%% koniec tych dodatkowych definicji

%%%%%% trocę więcej ``luzu'' przy rozmieszczaniu {fgur} i~{table}

 \renewcommand{\topfraction}{0.9}	% max fraction of floats at top
    \renewcommand{\bottomfraction}{0.8}	% max fraction of floats at bottom
    %   Parameters for TEXT pages (not float pages):
    \setcounter{topnumber}{2}
    \setcounter{bottomnumber}{2}
    \setcounter{totalnumber}{4}     % 2 may work better
    \setcounter{dbltopnumber}{2}    % for 2-column pages
    \renewcommand{\dbltopfraction}{0.9}	% fit big float above 2-col. text
    \renewcommand{\textfraction}{0.07}	% allow minimal text w. figs
    %   Parameters for FLOAT pages (not text pages):
    \renewcommand{\floatpagefraction}{0.7}	% require fuller float pages
    % N.B.: floatpagefraction MUST be less than topfraction !!
    \renewcommand{\dblfloatpagefraction}{0.7}	% require fuller float pages
    % remember to use [htp] or [htpb] for placement

    
%%% DWA proste polecenia służące do ujednolicenia podawania źródeł przy rysunkach i~tabelkach    
    
    \newcommand\zrodlo[1]{\par\vspace{-3mm}{\small\textit{Źródło: }#1 }}
    \newcommand\zrodlotab[1]{{\par\vspace{2mm}\small\textit{Źródło: }#1 }}

\raggedbottom   %%% to znaczy, że nie będzie siłowego wyrównywania typowych
                %%     stron do jednakowej wysokości

\linespread{1.3}

\begin{document}

%%%%%%%%%%%%%%%%%%%%%%%%%%%%%%%%%%%%%%%%%
%%%%%%%%%%%%%%%%%%%%%%%%%%%%%%%%%%%%%%%%%
%%%%%%%% STRONA TYTUŁOWA %%%%%%%%%%%%%%%%

\thispagestyle{empty}  % tu wszak nie chcemy żadnej numeracji stron


%%%%%%%%%%%%%%%%%%%%%%%%%%%%%%%%%%%%%%%%%%%%%%%%%%%%%%%%%%%%%%%
%%%%%tytuły definiuje jako makrodefinicje, gdyż zamierzam je%%%
%%%%%powtórzyć na stronie ze streszczeniami, to nic nie boli%%%
%%%%%a gwarantuje, że będą one takie same, i~tak ma być.%%%%%%%
%%%%%%%%%%%%%%%%%%%%%%%%%%%%%%%%%%%%%%%%%%%%%%%%%%%%%%%%%%%%%%%
\newcommand\tytul{Zastosowanie modeli mieszanych w analizie rozwoju pandemii choroby COVID-19 na świecie}

\newcommand\tytulangielski{The use of mixed-effects models in the analysis of the COVID-19 pandemic in the world}


\begin{center}


{\large \bf POLITECHNIKA LUBELSKA}

{\bf WYDZIAŁ PODSTAW TECHNIKI}

\emph{Kierunek:} MATEMATYKA

%%% BEZ SPEC.!!! \emph{Specjalność:} Matematyka w~finansach i~ubezpieczeniach

\vfill %%%% \vfill to taki rozpychacz w pionie, pcha ile mu pozwolą
     

\includegraphics[width=3.5cm]{rys/logopl}

\vfill

\textbf{Praca inżynierska}

\vfill
\vfill
\vfill

\large
\tytul

\vfill

\emph{\tytulangielski}


\vfill
\vfill
\vfill
\vfill
\vfill

\begin{tabular}[t]{l}
\emph{Praca wykonana pod kierunkiem:}
\\
dra Dariusza Majerka
\end{tabular}
\hfill
\begin{tabular}[t]{l}
	\emph{Autor:}
\\
Alicja Hołowiecka\\
nr albumu: 89892 
\end{tabular}

\vfill
\vfill
\vfill

\textbf{Lublin 2020}

\end{center}


%%%%% koniec tytułów


%%%%%%%%%%%%%%%%%%%%%%%5
%%%%%%%%%%%%%%%%%%%%%%
%%% teraz spis treści
%%%%%%%%%%%%%%%%%%%%%
%%% pamiętaj! po jakiejkolwiek zmianie w tekście
%%% która wpływa na zmianę spisu treści, spis będzie dobry co najmniej
%%% po dwóch przebiegach latexa - to samo dotyczy odwołań do wzorów i literatury
%%% ogólnie to przed wydrukiem warto przelatexować o jedne raz więcej niż
%%% to się wydaje konieczne, no chyba że korzystamy z funkcji typu BUILD
%%% w zintegrowanym systemie wspomagającym TeX, BUILD powinien takie sprawy 
%%% wziąć pod uwagę

\tableofcontents


\chapter*{Wstęp}




Pandemia choroby COVID-19 jest wydarzeniem, które wstrząsnęło całym światem w roku 2019. Właściwie nikt chyba nie może powiedzieć, że nie poczuł się dotknięty przez sytuację związaną z rozprzestrzenianiem się wirusa. Pierwsze przypadki pojawiły się pod koniec 2019 roku we wschodnich Chinach, w mieście Wuhan. Na początku 2020 roku chorowali już obywatele większości państw na świecie. Na moment pisania tej pracy, sytuacja nadal nie jest opanowana i nie wiadomo, jak się rozwinie.

Biorąc to pod uwagę, tym ważniejszy wydaje się temat poruszany w tej pracy. Wiele jednostek naukowych podejmuje próby znalezienia odpowiedniego modelu, aby przewidzieć rozwój pandemii. Przedstawione w tej pracy modele mieszane co prawda nie pozwalają na dokładną predykcję, ale są dobrym narzędziem, aby odkryć, które czynniki mają wpływ na rozwój pandemii w przeciętnym kraju.











\chapter{Teoretyczne podstawy badań własnych}
W tej części pracy przedstawimy metody matematyczne, które zostaną użyte w części praktycznej tej pracy. Zgodnie z tematem, będą to głównie modele mieszane.
\section{Modele liniowe}
Na początek przypomnimy podstawowe wiadomości o modelach liniowych.

Model regresji prostej ma postać 
$$y=x \beta_1+\beta_0 + \varepsilon$$

gdzie oszacowania parametrów $\beta_1$, $\beta_0$ obliczamy następująco:

$$\hat{\beta_1}=\frac{Cov(x,y)}{Var(x)},$$
$$\hat{\beta_0}=\overline{y}-\overline{x}\hat{\beta_1}.$$

Model interpretujemy w ten sposób, że jeżeli zmienna $x$ wzrośnie o 1, to zmienna $y$ zmieni się o $\beta_1$.
\subsection{Metody estymacji parametrów modelu liniowego}
\begin{enumerate}
	\item Metoda najmniejszych kwadratów, OLS (ang. \emph{Ordinary Least Squares}) - w metodzie tej minimalizujemy błąd kwadratowy, czyli sumę kwadratów reszt, którą oznaczamy RSS (ang. \emph{Residual Sum of Squares}).
	
	$$RSS= \sum_{i=1}^{n}(y_i-\hat{y_i})^2$$
	
	Twierdzenie Gaussa-Markowa: taki estymator jest BLUE (Best Linear Unbiased Estimator), przy odpowiednich założeniach.
	
	\item Metoda największej wiarogodności, ML (ang.\emph{Maximum Likelihood}) polega na maksymalizacji wartości funkcji prawdopodobieństwa ze względu na $\beta$ (w praktyce maksymalizujemy zwykle logarytm z tej funkcji)
	
	$$\hat{\sigma}^{2}_{ML}=RSS/n$$
	
	Estymując $\sigma^2$, maksymalizujemy funkcję wiarogodności zarówno ze względu na $\beta$, jak i $\sigma^2$.
	
	Estymatory uzyskane tą metodą są asymptotycznie nieobciążone.
	
	\item Resztowa metoda największej wiarogodności, REML (ang. \emph{Residual/Restricted Maximum Likelihood Method}) - z estymacji parametru $\sigma^2$ usuwamy wpływ parametrów zakłócających $\beta$.
	
	$$\hat{\sigma}^2_{REML}=RSS/(n-p)$$
	
	Estymatory uzyskane tą metodą są nieobciążone \cite{biecek}.
\end{enumerate}

\subsection{Badanie istotności parametrów}

$$H_0: \beta_i = 0$$

\section{Modele mieszane} 
W powyżej opisanych modelach liniowych z efektami stałymi zakładamy niezależność kolejnych pomiarów, dlatego nie są to odpowiednie modele, kiedy mamy np. kilka pomiarów dla pojedynczego elementu. W takim przypadku możemy użyć modeli liniowych z efektami mieszanymi (stałymi i losowymi), które krótko nazywamy modelami mieszanymi.

Modeli mieszanych używamy w przypadu powtarzanych pomiarów bądź w przypadku hierarchicznej lub zagnieżdżonej struktury. Takie dane charakteryzują się korelacją między obserwacjami z tej samej grupy, co nie pozwala na użycie modelu liniowego z efektami stałymi. Dlatego do modelu wprowadza się czynnik losowy. 

Czynnik stały jest pewnym parametrem, którego wartość estymujemy na podstawie próbki, natomiast czynnik losowy jest zmienną losową, dla której próbujemy oszacować parametry jej rozkładu \cite{faraway}.
	
Przykładową sytuacją, gdzie możemy użyć modelu mieszanego, jest badanie działania leku na grupie pacjentów, gdzie dokonujemy kilku pomiarów na danym pacjencie. W tym przypadku nie interesuje nas konkretny pacjent, ale raczej wpływ leku na przeciętnego pacjenta. Dodatkowo, traktujemy pacjentów jako losowo wybranych. Podejście modelu mieszanego będzie polegało na potraktowaniu wpływu pacjenta jako czynnik zakłócający. 

Rozważamy model postaci
$$y=X\beta +Z u + \varepsilon$$
gdzie $X$ - macierz zmiennych będących efektami stałymi, $Z$ - macierz zmiennych będących efektami losowymi, $\beta$ to wektor nieznanych efektów stałych, $\varepsilon \sim \mathcal{N}(0, \sigma^2 I_{n\times n})$ to zakłócenie losowe, a $u \sim \mathcal{N} (0, \sigma^2D)$ to wektor zmiennych losowych odpowiadających efektom losowym \cite{biecek}.

Znając $D$, możemy estymować parametry $\beta$ uogólnioną metodą najmniejszych kwadratóW. Do estymowania nieznanego $D$ możemy użyć np. metodą największej wiarogodności.

\subsection{Metody estymacji}

Do oceny wartości parametrów modelu mieszanego można stosować metody ML (Największej Wiarogodności) oraz REML (Resztowej Największej Wiarogodności), wspomniane w tej pracy przy okazji modeli liniowych. W przypadku modeli mieszanych obydwoma metodami możemy uzyskać estymatory obciążone, ale to obciążenie jest zazwyczaj mniejsze w przypadku estymatorów uzyskanych metodą REML.

Różnica między metodą REML i ML polega na tym, że w metodzie REML najpierw usuwamy wpływ efektów stałych.

\subsection{Badanie istotności parametrów}

$$H_0: \sigma^2_j=0$$


\subsection{Wybór najlepszego modelu}

Metody, które mają zastosowanie dla modeli liniowych z efektami stałymi, nie zawsze dają się zastosować w przypadku modeli mieszanych. Wymienimy teraz kilka metod doboru najlepszego modelu i opiszemy, które z nich są najskuteczniejsze \cite{faraway}.

\begin{enumerate}
	\item Wskaźnik wiarygodności(ang. \texttt{likelihood ratio}) - tworzymy dwa modele, model 0, który nie zawiera elementów, których istotność chcemy zbadać, i model 1, który zawiera te elementy. Pozostałe zmienne muszą być takie same w obu modelach.
	
	Statystyka testowa wygląda następująco:
	$$2(l(\hat{\beta_1}, \hat{\sigma_1}, \hat{D_1}|y)-l(\hat{\beta_0}, \hat{\sigma_0}, \hat{D_0}|y)),$$
	gdzie l - logarytm z funkcji prawdopodobieństwa.
	
	Tego testu nie można używać do modeli wyznaczonych metodą \texttt{REML}.
	
	\item Test F dla efektów stałych - metoda taka sama jak ta używana przy modelach z efektami stałymi. W przypadku modeli mieszanych może sprawiać problemy, ponieważ statystyka testowa niekoniecznie musi mieć rozkład F. Należy także wprowadzać poprawkę na liczbę stopni swobody.
	
	Na ogół ta metoda daje dobre rezultaty dla mniej skomplikowanych modeli, gdy układ jest zbalansowany (wszystkie grupy są równoliczne). Dla modeli bardziej skomplikowanych, lub kiedy brak równoliczności, wartości p oraz statystyki t mogą być błędne.
	
	\item Użycie metod \textit{bootstrapowych}, aby znaleźć dokładniejsze wartości p-value. Należy wygenerować dane z modelu 0 (na podstawie oszacowanych parametrów) i obliczyć statystykę \textit{likelihood ratio}. Tą procedurę powtarzamy wielokrotnie i oceniamy istotność.
	
	\item Kryteria informacyjne - służą do wyboru najlepszego spośród modeli. Najpopularniejszym jest Kryterium Informacyjne Akaikego (ang. \texttt{Akaike Information Criterion, AIC}). Jest ono zdefiniowane następującym wzorem:
	
	$$-2(\texttt{max log likelihood})+ 2p,$$
	
	gdzie $p$ to liczba parametrów modelu.
	
	Można stosować to kryterium do modeli, które różnią się jedynie efektami stałymi, gdzie liczba efektów losowych jest identyczna dla wszystkich modeli, które porównujemy. Gdyby modele różniły się liczbą efektów losowych, należałoby rozważyć, w jaki sposób zliczyć liczbę parametrów $p$.
	
	Kryterium Akaikego jest miarą utraconej informacji, więc po obliczeniu go dla roważanych modeli, należy wybrać ten, gdzie otrzymana wartość jest najmniejsza.

\end{enumerate}

\subsection{Interpretacja parametrów modelu mieszanego}
\subsection{Predykcja z modelu mieszanego}
\chapter{Badania własne}
\section{Zbiór danych i jego wstępne przygotowanie}

Zbiór danych pochodzi z witryny internetowej Our World In Data \cite{owid}, gdzie dane zostały zebrane z różnych źródeł, m. in. ze Światowej Organizacji Zdrowia (WHO) oraz Europejskiego Centrum ds. Zapobiegania i Kontroli Chorób (ECDC). W zbiorze znajduje się 210 krajów, dane dotyczące terytoriów międzynarodowych oraz łącznie dla całego świata. Mamy ponad 40 kolumn z różnymi parametrami - w dalszej części pracy opiszemy, które zmienne będą przez nas użyte.

W zbiorze znajdowało się wiele braków danych. Dla każdego kraju zostały usunięte dane sprzed rozpoczęcia się epidemii na jego terytorium (\texttt{total cases=0}), dni są numerowane kolejnymi liczbami całkowitymi.

Ze zbioru danych zostały usunięte wszystkie kraje o populacji poniżej miliona mieszkańców, ponieważ w większości były to nieduże wysepki, dla których dane były wybrakowane. Oprócz tego, kilka innych krajów zostało usuniętych, ponieważ mimo większej populacji, dane były niepełne.

Do formułowania hipotez i budowania modeli będziemy się posługiwać następującymi zmiennymi:

\begin{itemize}
	\item liczba zachorowań - jest to liczba potwierdzonych przypadków koronawirusa w danym kraju od momentu rozpoczęcia epidemii. Zamiast wartośći liczby zachorowań, będziemy używać liczby zachorowań na milion mieszkańców (\texttt{total cases per million}),
	\item liczba wykonanych testów - będziemy używać liczby wykonanych testów w przeliczeniu na tysiąc mieszkańców danego kraju (\texttt{total tests per thousand}),
	\item wskaźnik siły obostrzeń (\texttt{sringency index}) - wskaźnik tego, jak silne obostrzenia wprowadził rząd danego kraju. Jest to kombinacja dziewięciu innych zmiennych, m.in. zamykanie szkół, polityka wykonywania testów, ograniczenie kontaktów międzyludzkich itp. Może przyjmować wartości od 0 do 100, im większa wartość, tym silniejsze obostrzenia w danym kraju \cite{stringency},
	\item gęstość zaludnienia (\texttt{population density}),
	\item PKP danego kraju na osobę (\texttt{GDP per capita}) - Produkt Krajowy Brutto, przeliczony na hipotetyczną walutę dolara międzynarodowego \cite{dollars},
	\item część społeczeństwa żyjąca w skrajnym ubóstwie (\texttt{extreme poverty})
	\item śmiertelność z powodu chorób sercowych (\texttt{cardiovasc death rate}) - stan na rok 2017
	\item powszechność występowania cukrzycy (\texttt{diabetes prevalence}) - odsetek populacji z cukrzycą, brane pod uwagę są osoby w wieku od 20 do 70 lat, stan na rok 2017
	\item oczekiwana długość życia (\texttt{life expectancy}) - kraje zostaną podzielone na kategorie ze wzgędu na tą zmienną, wyróżnimy kraje, w których oczekiwana długość życia jest poniżej 50 lat, między 50 a 54, między 55 a 59 i tak dalej aż do grupy krajów z oczekiwaną długością życia powyżej 80 lat.
	
\end{itemize}




%\begin{figure}[htbp]
%	\centering
%	\includegraphics{plot_all_countries.png}
%	\caption{Wykres przedstawiający rozwój pandemii we wszystkich krajach, tak, wiem, że nic na nim nie widać}
%	\label{fig:plot_all}
%	\zrodlo{Opracowanie własne}
%\end{figure}


\section{Dyskusja wyników}

\subsection{Model 1}

Hipoteza 1: Czas ma istotny wpływ na liczbę zachorowań.

\begin{figure}[htbp]
	\centering
	\includegraphics[width=\linewidth]{rys/total_cases_countries.png}
	\caption{Wykres przedstawiający zależność liczby zachorowań na milion mieszkańców w zależności od czasu z podziałem na kraje}
	\label{fig:total_cases_countries}
	\zrodlo{Opracowanie własne}
\end{figure}



Pierwszy model ma postać 
\begin{verbatim}
mod <- lme(total_cases_per_million~time, 
random = ~1|location, 
data = covid)
\end{verbatim}

a więc przedstawia zależność liczby zachorowań od czasu, a kraj jest efektem losowym.

\begin{figure}[htbp]
	\centering
	\includegraphics[width=\linewidth]{rys/mod1_predict.png}
	\caption{Wykres przedstawiający zależność między liczbą zachorowań a czasem oszacowaną za pomocą Modelu 1}
	\label{fig:mod1_predict}
	\zrodlo{Opracowanie własne}
\end{figure}

\begin{figure}[htbp]
	\centering
	\includegraphics[width=\linewidth]{rys/mod1_slope_predict.png}
	\caption{Wykres przedstawiający zależność między liczbą zachorowań a czasem oszacowaną za pomocą modelu typu random intercept and slope}
	\label{fig:mod1_slope_predict}
	\zrodlo{Opracowanie własne}
\end{figure}

Dla efektu losowego otrzymujemy następujący wynik:


\begin{table}[!htbp]
	\begin{center}
		\begin{tabular}{l c}
			\hline
			& Model 1 \\
			\hline
			(Intercept)               & $-1321.86^{***}$ \\
			& $(294.61)$       \\
			time                      & $31.27^{***}$    \\
			& $(0.24)$         \\
			\hline
			AIC                       & $715285.90$      \\
			BIC                       & $715320.03$      \\
			Log Likelihood            & $-357638.95$     \\
			Num. obs.                 & $37529$          \\
			Num. groups: location     & $152$            \\
			Var: location (Intercept) & $13018644.39$    \\
			Var: Residual             & $10840327.89$    \\
			\hline
			\multicolumn{2}{l}{\scriptsize{$^{***}p<0.001$; $^{**}p<0.01$; $^{*}p<0.05$}}
		\end{tabular}
		\caption{Wyniki dla modelu 1}
		\label{table:model1}
	\end{center}
\end{table}



Widać zatem, że efekt losowy jest odpowiedzialny za około 45\% wariancji resztowej.



Zarówno wyraz wolny, jak i współczynnik przy zmiennej Czas, są istotne statystycznie. Dodatkowo, korelacja pomiędzy liczbą zachorowań a czasem jest dodatnia, więc wraz z upływem czasu liczba zachorowań rośnie dla przeciętnego kraju.
\subsection{Model 2}

Hipoteza 2: Liczba wykonywanych testów na COVID-19 ma związek z liczbą zachorowań.

Drugi model to:
\begin{verbatim}
mod1 <- lme(total_cases_per_million~time+total_tests_per_thousand,
random = ~1|location,
data = covid_na)
\end{verbatim}

Badamy tutaj, czy czas oraz liczba wykonywanych testów mają wpływ na liczbę zachorowań, jeżeli kraj traktujemy jako czynnik losowy.

Dla tego modelu otrzymujemy następujące wyniki:
\begin{table}
	\begin{center}
		\begin{tabular}{l c}
			\hline
			& Model 1 \\
			\hline
			(Intercept)                 & $1319.61^{***}$ \\
			& $(399.12)$      \\
			total\_tests\_per\_thousand & $31.95^{***}$   \\
			& $(0.28)$        \\
			\hline
			AIC                         & $363631.09$     \\
			BIC                         & $363662.51$     \\
			Log Likelihood              & $-181811.54$    \\
			Num. obs.                   & $19045$         \\
			Num. groups: location       & $95$            \\
			Var: location (Intercept)   & $14965179.24$   \\
			Var: Residual               & $11168063.61$   \\
			\hline
			\multicolumn{2}{l}{\scriptsize{$^{***}p<0.001$; $^{**}p<0.01$; $^{*}p<0.05$}}
		\end{tabular}
		\caption{Wyniki dla modelu 2}
		\label{table:model2}
	\end{center}
\end{table}

Widać po pierwsze, że efekt losowy jest odpowiedzialny za ponad połowę zmienności resztowej modelu. Po drugie, widać, że oba efekty stałe są istotne statystycznie, i oba mają wpływ stymulujący na liczbę zachorowań.

\subsection{Model 3}


Hipoteza 3: kraje o różnej oczekiwanej długości życia różnią się liczbą zachorowań.

Trzeci model wygląda następująco:

\begin{verbatim}
mod2 <- lme(total_cases_per_million~time+age,
random=~1|location,
data=covid)

\end{verbatim}

Prezentuje on zależność liczby zachorowań od czasu i od oczekiwanej długości życia w danym kraju. Występuje także efekt losowy kraju.

Dla modelu trzeciego otrzymujemy następujące wyniki:

\begin{table}
	\begin{center}
		\begin{tabular}{l c}
			\hline
			& Model 1 \\
			\hline
			(Intercept)               & $640.36$       \\
			& $(1343.96)$    \\
			age60-64                  & $-41.47$       \\
			& $(1502.59)$    \\
			age65-69                  & $-248.44$      \\
			& $(1551.85)$    \\
			age70-74                  & $1509.97$      \\
			& $(1476.43)$    \\
			age75-79                  & $3319.36^{*}$  \\
			& $(1436.70)$    \\
			age80 and above           & $4104.39^{**}$ \\
			& $(1480.87)$    \\
			agebelow 55               & $-371.40$      \\
			& $(1993.64)$    \\
			\hline
			AIC                       & $729555.23$    \\
			BIC                       & $729632.03$    \\
			Log Likelihood            & $-364768.62$   \\
			Num. obs.                 & $37529$        \\
			Num. groups: location     & $152$          \\
			Var: location (Intercept) & $10768986.56$  \\
			Var: Residual             & $15929745.12$  \\
			\hline
			\multicolumn{2}{l}{\scriptsize{$^{***}p<0.001$; $^{**}p<0.01$; $^{*}p<0.05$}}
		\end{tabular}
		\caption{Wyniki dla modelu 3}
		\label{table:model3}
	\end{center}
\end{table}

Ponownie efekt losowy odpowiada za większą część wariancji resztowej. Czas ponownie jest istotny i ma wpływ stymulujący. Dla grupy wiekowej 75-79 różnica w średniej liczbie zachorowań jest na granicy istotności statystycznej. Dopiero dla krajów o oczekiwanej długości życia powyżej 80 lat pojawia się istotna różnica - liczba zachorowań w tych krajach jest największa. W pozostałych grupach wiekowych nie można mówić o istotnych różnicach. 
\subsection{Model 4}

Hipoteza 4: Kraje o różnej gęstości zaludnienia różnią się liczbą zachorowań.

W czwartym modelu badamy zależnośc liczby zachorowań od czasu i gęstości zaludnienia, a kraj jest czynnikiem losowym.

\begin{verbatim}
mod3 <- lme(total_cases_per_million~time+population_density,
random=~1|location,
data = covid)
\end{verbatim}

Wyniki są następujące:

\begin{table}
	\begin{center}
		\begin{tabular}{l c}
			\hline
			& Model 1 \\
			\hline
			(Intercept)               & $2402.87^{***}$ \\
			& $(306.53)$      \\
			population\_density       & $0.78$          \\
			& $(0.44)$        \\
			\hline
			AIC                       & $729672.88$     \\
			BIC                       & $729707.01$     \\
			Log Likelihood            & $-364832.44$    \\
			Num. obs.                 & $37529$         \\
			Num. groups: location     & $152$           \\
			Var: location (Intercept) & $13106475.48$   \\
			Var: Residual             & $15929750.91$   \\
			\hline
			\multicolumn{2}{l}{\scriptsize{$^{***}p<0.001$; $^{**}p<0.01$; $^{*}p<0.05$}}
		\end{tabular}
		\caption{Wyniki dla modelu 4}
		\label{table:model4}
	\end{center}
\end{table}

Zatem gęstośc zaludnienia nie jest czynnikiem istotnie różnicującym liczbę zachorowań w krajach.

\subsection{Model 5}

Hipoteza 5: Kraje różniące się siłą obostrzeń mają istotne różnice w liczbie zachorowań.


Piaty model ma następującą postać:

\begin{verbatim}
covid_si <- drop_na(covid, stringency_index)
mod4 <- lme(total_cases_per_million~time+stringency_index,
random=~1|location,
data = covid_si)
\end{verbatim}

W tym modelu sprawdzamy zależność liczby zachorowań od czasu i siły obostrzeń, kraj jest czynnikiem losowym.

Otrzymujemy następujące wyniki:

\begin{table}
	\begin{center}
		\begin{tabular}{l c}
			\hline
			& Model 1 \\
			\hline
			(Intercept)               & $2932.54^{***}$ \\
			& $(301.93)$      \\
			stringency\_index         & $-9.12^{***}$   \\
			& $(1.16)$        \\
			\hline
			AIC                       & $687061.29$     \\
			BIC                       & $687095.20$     \\
			Log Likelihood            & $-343526.65$    \\
			Num. obs.                 & $35524$         \\
			Num. groups: location     & $148$           \\
			Var: location (Intercept) & $12678743.39$   \\
			Var: Residual             & $14372604.68$   \\
			\hline
			\multicolumn{2}{l}{\scriptsize{$^{***}p<0.001$; $^{**}p<0.01$; $^{*}p<0.05$}}
		\end{tabular}
		\caption{Wyniki dla modelu 5}
		\label{table:model5}
	\end{center}
\end{table}
Tak jak w poprzednich modelach, czynnik losowy odpowiada za największą część zmienności resztowej. Czas jest istotny statystycznie. Wskaźnik siły obostrzeń także jest istotny i ma wpływ stymulujący, co oznaczałoby, że im silniejsze obostrzenia, tym więcej zachorowań. Ta interpretacja prawdopodobnie jest niepoprawna, można się domyślać, że raczej zachodzi odwrotna zależność - w krajach z największą liczbą zachorowań są wprowadzane najsurowsze obostrzenia.

\subsection{Model 6}

Hipoteza 6: Kraje o różnej wysokości wskaźnika rozwoju społecznego (HDI) różnią się liczbą zachorowań.

Szósty model przedstawia zależność liczby zachorowań od czasu i wskaźnika rozwoju społecznego, a kraj jest czynnikiem losowym.

\begin{verbatim}
covid_hdi <- drop_na(covid, human_development_index)
mod5 <- lme(total_cases_per_million~time+human_development_index,
random=~1|location,
covid_hdi)
\end{verbatim}

Z tego modelu mamy następujący wynik:

\begin{table}
	\begin{center}
		\begin{tabular}{l c}
			\hline
			& Model 1 \\
			\hline
			(Intercept)               & $-4489.15^{***}$ \\
			& $(1237.25)$      \\
			human\_development\_index & $9957.74^{***}$  \\
			& $(1711.04)$      \\
			\hline
			AIC                       & $720398.99$      \\
			BIC                       & $720433.07$      \\
			Log Likelihood            & $-360195.50$     \\
			Num. obs.                 & $37070$          \\
			Num. groups: location     & $150$            \\
			Var: location (Intercept) & $10865235.65$    \\
			Var: Residual             & $15798887.42$    \\
			\hline
			\multicolumn{2}{l}{\scriptsize{$^{***}p<0.001$; $^{**}p<0.01$; $^{*}p<0.05$}}
		\end{tabular}
		\caption{Wyniki dla modelu 6}
		\label{table:model6}
	\end{center}
\end{table}

 Efekty stałe są istotne i oba mają wpływ stymulujacy. W krajach z wyższym wskaźnikiem rozwoju społecznego, zachorowań jest znacząco więcej.

\subsection{Model 7}

Hipoteza 7: Kraje o różnej wysokości odsetka śmierci z powodu chorób sercowych różnią się liczbą zachorowań.

Model siódmy wygląda następująco:

\begin{verbatim}
mod6 <- lme(total_cases_per_million~time+cardiovasc_death_rate,
random=~1|location,
data= covid)
\end{verbatim}

i oprócz zależności liczby zachorowań od czasu zawiera także zależność od odsetka śmierci spowodowanych chorobami sercowymi. Kraj jest traktowany jako efekt losowy.

Otrzymujemy następujące podsumowanie:

\begin{table}
	\begin{center}
		\begin{tabular}{l c}
			\hline
			& Model 1 \\
			\hline
			(Intercept)               & $4982.81^{***}$ \\
			& $(672.06)$      \\
			cardiovasc\_death\_rate   & $-9.25^{***}$   \\
			& $(2.32)$        \\
			\hline
			AIC                       & $729657.48$     \\
			BIC                       & $729691.61$     \\
			Log Likelihood            & $-364824.74$    \\
			Num. obs.                 & $37529$         \\
			Num. groups: location     & $152$           \\
			Var: location (Intercept) & $12094158.57$   \\
			Var: Residual             & $15929749.88$   \\
			\hline
			\multicolumn{2}{l}{\scriptsize{$^{***}p<0.001$; $^{**}p<0.01$; $^{*}p<0.05$}}
		\end{tabular}
		\caption{Wyniki dla modelu 7}
		\label{table:model7}
	\end{center}
\end{table}

Czynnik losowy zachowuje się podobnie jak we wszystkich poprzednich modelach. Oba czynniki stałe są istotne. Co ciekawe, odsetek śmierci spowodowanych chorobami serca wpływa ograniczająco na liczbę zachorowań. Może to być związane z tym, że osoby chore na serce bardziej uważają, aby się nie zarazić, tym samym zmniejszają liczbę zachorowań w danym kraju.

\subsection{Model 8}

Hipoteza 8: Kraje o różnej wysokości odsetka osób chorych na cukrzycę różnią się liczbą zachorowań.

\begin{verbatim}
mod7 <- lme(total_cases_per_million~time+diabetes_prevalence,
random=~1|location,
data= covid)
\end{verbatim}

Model ten jest analogiczny do poprzedniego, z tym że zamiast chorób sercowych mamy tu odsetek chorych na cukrzycę.

Otrzymujemy nastepujący wynik:
\begin{table}
	\begin{center}
		\begin{tabular}{l c}
			\hline
			& Model 1 \\
			\hline
			(Intercept)               & $93.18$        \\
			& $(612.83)$     \\
			diabetes\_prevalence      & $337.40^{***}$ \\
			& $(74.81)$      \\
			\hline
			AIC                       & $729646.52$    \\
			BIC                       & $729680.65$    \\
			Log Likelihood            & $-364819.26$   \\
			Num. obs.                 & $37529$        \\
			Num. groups: location     & $152$          \\
			Var: location (Intercept) & $11775109.20$  \\
			Var: Residual             & $15929749.48$  \\
			\hline
			\multicolumn{2}{l}{\scriptsize{$^{***}p<0.001$; $^{**}p<0.01$; $^{*}p<0.05$}}
		\end{tabular}
		\caption{Wyniki dla modelu 8}
		\label{table:model8}
	\end{center}
\end{table}

Czynnik losowy ma taką samą istotność jak poprzednio, czas także. Rozpowszechnienie cukrzycy wpływa stymulująco na liczbę zachorowań. Może to być spowodowane tym, że osoby chore na cukrzycę mają słabszy organizm i są bardziej narażone na zakażenie.
\subsection{Model 9}

Hipoteza 9: Kraje o różnej wysokości odsetka osób żyjących w skrajnej biedzie różnią się liczbą zachorowań.

W tym modelu sprawdzamy zależność liczby zachorowań od czasu i odsetka osób żyjących w skrajnym ubóstwie. Kraj jest czynnikiem losowym.

\begin{verbatim}
covid_ep <- drop_na(covid, extreme_poverty)
mod8 <- lme(total_cases_per_million~time+extreme_poverty,
random=~1|location,
data= covid_ep)
\end{verbatim}

Model ten ma następujące podsumowanie:

\begin{table}
	\begin{center}
		\begin{tabular}{l c}
			\hline
			& Model 1 \\
			\hline
			(Intercept)               & $3074.37^{***}$ \\
			& $(306.39)$      \\
			extreme\_poverty          & $-55.02^{***}$  \\
			& $(11.98)$       \\
			\hline
			AIC                       & $520055.39$     \\
			BIC                       & $520088.22$     \\
			Log Likelihood            & $-260023.70$    \\
			Num. obs.                 & $27083$         \\
			Num. groups: location     & $110$           \\
			Var: location (Intercept) & $6944468.09$    \\
			Var: Residual             & $12546263.85$   \\
			\hline
			\multicolumn{2}{l}{\scriptsize{$^{***}p<0.001$; $^{**}p<0.01$; $^{*}p<0.05$}}
		\end{tabular}
		\caption{Wyniki dla modelu 9}
		\label{table:model9}
	\end{center}
\end{table}

Czynnik losowy jest odpowiedzialny za niecałą połowę zmienności resztowej. Oba czynniki stałe są istotne. Im większy jest w danym kraju odsetek osób żyjących w biedzie, tym wyższa liczba zachorowań. Prawdopodobnie jest to spowodowane mniejszą dostępnością do służby zdrowia w biedniejszych krajach i mniejszą liczbą wykonywanych testów.

\subsection{Model 10}

Hipoteza 10: Kraje o różnej wysokości PKB różnią się liczbą zachorowań.


W modelu dziesiątym pojawia się zależność liczby zachorowań od czasu i PKB. Kraj jest efektem losowym.

\begin{verbatim}
covid_gdp <- drop_na(covid, gdp_per_capita)
mod9 <- lme(total_cases_per_million~time+gdp_per_capita,
random=~1|location,
data= covid_gdp)
\end{verbatim}

Wyniki są następujące:
\begin{table}
	\begin{center}
		\begin{tabular}{l c}
			\hline
			& Model 1 \\
			\hline
			(Intercept)               & $546.41$      \\
			& $(338.86)$    \\
			gdp\_per\_capita          & $0.11^{***}$  \\
			& $(0.01)$      \\
			\hline
			AIC                       & $720911.67$   \\
			BIC                       & $720945.75$   \\
			Log Likelihood            & $-360451.83$  \\
			Num. obs.                 & $37057$       \\
			Num. groups: location     & $150$         \\
			Var: location (Intercept) & $8906070.23$  \\
			Var: Residual             & $16132368.91$ \\
			\hline
			\multicolumn{2}{l}{\scriptsize{$^{***}p<0.001$; $^{**}p<0.01$; $^{*}p<0.05$}}
		\end{tabular}
		\caption{Wyniki dla modelu 10}
		\label{table:model10}
	\end{center}
\end{table}

Efekt losowy odpowiada za niecałą połowę zmienności resztowej modelu. Efekty stałe są istotne statystycznie. Im wyższe PKB danego kraju, tym wyższa liczba zachorowań.
\chapter*{Podsumowanie i~wnioski}



\begin{longtable}{| p{.30\textwidth} | p{.60\textwidth} |}
	\hline
	Cecha & Wpływ na liczbę zachorowań \\ \hline 
	Czas & istotny, wraz z upływem czasu rośnie liczba zachorowań \\ \hline
	Liczba wykonywanych testów na COVID-19 & istotny, wraz ze wzrostem liczby testów rośnie liczba zachorowań \\ \hline
	Oczekiwana długość życia & istotny, o ile ta wartość przekracza 80 lat, wówczas zachorowań jest więcej niż dla krajów o krótszej oczekiwanej długości życia \\ \hline 
	Gęstość zaludnienia & nieistotny \\ \hline
	Wskaźnik siły obostrzeń & istotny, jest wysoki w krajach o dużej liczbie zachorowań \\ \hline
	Wskaźnik rozwoju społecznego & istotny, w krajach o wysokim wskaźniku rozwoju jest więcej zachorowań \\ \hline
	Śmiertelność z powodu chorób sercowych & istotny, im wyższy jest ten współczynnik, tym mniej zachorowań na COVID-19 \\ \hline
	Powszechność występowania cukrzycy & istotny, im wyższy jest ten współczynnik, tym więcej zachorowań na COVID-19 \\ \hline
	Część populacji żyjąca w skrajnym ubóstwie & istotny, im większa jest część mieszkańców żyjąca w biedzie, tym mniej zachorowań \\ \hline
	PKB na osobę & istotny, im wyższe PKB, tym więcej zachorowań \\ \hline
	\caption{Porównanie istotności i wpływu różnych czynników na liczbę zachorowań w przeciętnym kraju}
	\label{tab:istotnosc}
	\end{longtable}

\begin{table}
	\centering
	\begin{tabular}{|c|c|}
		\hline 
		Nr modelu&   AIC \\ \hline
		1 &  715298.0  \\ \hline
		2 &  363631.1  \\ \hline
		3 &   729555.2  \\ \hline
		4 &   729672.9  \\ \hline
		5 &   687061.3  \\ \hline
		6 &   720399.0  \\ \hline
		7 &   729657.5  \\ \hline
		8 &   729646.5 \\ \hline
		9 &   520055.4 \\ \hline
		10&   720911.7 \\ \hline
	\end{tabular}
\caption{Porównanie indeksów Akaike dla poszczególnych modeli}
\label{tab:akaike}
\end{table}

Indeks Akaike jest miarą utraconej informacji. Im mniejszy jest indeks Akaike, tym lepiej model wyjaśnia badane zjawisko. Widzimy z tabeli \ref{tab:akaike}, że najmniejsze AIC występuje dla modelu nr 2. Jest to model, gdzie występuje zależność między liczbą zachorowań a liczbą wykonywanych testów. Ta zależność jest bardzo oczywista, ponieważ liczbę zachorowań zlicza się na podstawie tego, ile testów dało wynik pozytywny. Modelem z drugim najniższym AIC jest model nr 9, gdzie badamy istotność wskaźnika części populacji żyjącej w skrajnym ubóstwie.



Wszystkie modele jednoznacznie pokazują, że efekt kraju jako czynnika zakłócającego jest bardzo istotny, w wielu przypadkach bardziej niż jakikolwiek inny czynnik stały (np. czas).

Na to, co jest nazywane w tej pracy ,,efektem kraju'', składa się tak naprawdę wiele innych czynników, m. in. gęstość zaludnienia, sytuacja ekonomiczna danego kraju, odsetek osób z chorobami towarzyszącymi, rozkład wieku, jak również przyjęta strategia walki z koronawirusem, na którą z kolei składają się m. in. liczba wykonywanych testów, przepisy w sprawie zamykania szkół, miejsc publicznych, ograniczenie kontaktów międzyludzkich, i wiele innych.

W mojej pracy nie zajmowałam się badaniem, w jaki sposób te czynniki wpływają na wzrost lub spadek liczby zachorowań, chcę jedynie zasygnalizować, że mogą być istotne, skoro wykazany został wpływ efektu kraju na liczbę zachorowań.

\begin{thebibliography}{99}

\bibitem{biecek} Przemysław Biecek, \emph{Analiza danych z programem R. Modele liniowe z efektami stałymi, losowymi i mieszanymi}, Wydawnictwo Naukowe PWN, Wydanie II, Warszawa 2013
\bibitem{faraway} Julian J. Faraway, \emph{Extending the Linear Model with R. Generalized Linear, Mixed Effects and Nonparametric Regression Models. Second Edition}, CRC Press Taylor \& Francis  Group, 2016

\bibitem{owid} \url{https://ourworldindata.org/coronavirus}

\bibitem{stringency} \url{https://www.bsg.ox.ac.uk/research/research-projects/coronavirus-government-response-tracker} (dostęp 31.10.2020)

\bibitem{dollars} \url{https://ourworldindata.org/what-are-ppps} (dostęp 31.10.2020)

\end{thebibliography}



\listoffigures

\listoftables


\chapter*{Załączniki}
\begin{enumerate}
%\item Oświadczenie o oryginalności pracy i możliwości jej wykorzystania. 
%\item Opinia promotora na temat oryginalności pracy oraz w~sprawie dopuszczenia do obrony pracy dyplomowej.
%\item Potwierdzenie analizy antyplagiatowej.
\item Płyta CD z niniejszą pracą w wersji elektronicznej.
\end{enumerate}




\chapter*{Streszczenie (Summary)}

\bigskip
\bigskip

\begin{center}
  \textbf{\tytul}
\end{center}

W tej pracy przedstawione są pojęcia związane z modelami liniowymi z efektami stałymi i losowymi. Następnie opisane są badania własne na zbiorze danych dotyczącym rozprzestrzeniania się choroby COVID-19 w różnych krajach na świecie.


\bigskip

\begin{center}
  \textbf{\textit{\tytulangielski}}
\end{center}



\selecthyphenation{english}
{\it
In this paper, concepts related to linear models with fixed and random effects are presented. Then, our own research is described on the dataset on the spread of COVID-19 in various countries around the world.
}

\end{document}

