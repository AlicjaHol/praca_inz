%
% Szablon, v. 2.9
% p.wlaz@pollub.pl
%

% PROSZĘ NIE USUWAĆ
% KOMENTARZY Z~PREAMBUŁY
% JEŻELI KTOŚ WAM WMAWIA, ŻE
% TO PRZYSPIESZY COKOLWIEK
% -- MYLI SIĘ!


\documentclass[12pt]{mwbk}


%%%%%%% marinesy, rozmiary, to warto dopasować do drukarki
\usepackage[a4paper,twoside,top=2.6cm,bottom=2.6cm,inner=3cm,outer=2.6cm]{geometry}

%%%%%%%% polszczyzna
\usepackage[T1]{polski}


%%%%%%%%% sposób kodowania literek w edytorze
\usepackage[utf8]{inputenc}

\usepackage[font=small,labelfont=bf,justification=centering]{caption}


%%%% gdyby ktoś chciał powyklejać z~pedeefa
%%%% teksty za pomocą AcroReadera, to 
%%%% poniższe dwie linijki pomogą w~tym
%%%% Może to być przydatne, gdyby ktoś na podstawie
%%%% elektronicznej wersji chciał przygotować dane do 
%%%% badania antyplagiatowego
%%%% ponieważ prace są w
%%%% tych czasach różnymi
%%%% programami antyplagiatowymi
%%%% proszę absolutni NIE
%%%% USUWAĆ następujących
%%%% dwu linijek
\input glyphtounicode.tex
\pdfgentounicode = 1

%%%%%%%%%%%%%%%%%%%%%%%%%%%%%%%%%%%%%
%%%%% jeśli chcesz by główny tekst oraz wzory matematyczne były
%%%%% składane czcionką typu Times Roman (w~odróżnieniu od standardowej
%%%%% TeXowej, czyli Computer Modern Roman) to linia poniżej
%%%%% ma być 'aktywna', następna nieaktywna, 
%%%%% jeśli zrobisz odwrotnie (pierwsza nieaktywna,
%%%%% druga aktywna) uzyskasz skład czcionką
%%%%% Computer Modern Roman mającą wielu wiernych
%%%%% fanów w~świecie TeXa). Konsekwencją jednak będą zmiany
%%%%% rozmiarów czcionek dla rozdziałó i podrozdziałów - rzecz bez większego
%%%%% znaczenia, wynikająca z pewnych zaszłości historycznych (ComputerModern
%%%%% niegdyś były używane wyłącznie w postaci tzw. bitmap)
%\usepackage{mathptmx} \usepackage{tgtermes}
\usepackage{lmodern}

%%% WSZELKIE ZMIANY W~PREAMBULE RÓB ROZWAŻNIE
%%% NIE JESTEŚ PEWNY/PEWNA ICH EFEKTU TO~SPRAWDŹ 
%%% CZY W~PRGRAMIE ADOBE READER (i~to dokładnie
%%% o~ten program chodzi, nie o~jakikolwiek)
%%% Z~WYNIKOWEGO PLIU PDF DA SIE PRAWIDŁOWO
%%% WYKLEIĆ TEKST Z~POLSKIMI LITERAMI, BEZ KRZAKÓW,
%%%% BEZ DZIWACTW.


%%%%%%%%%%%%%% pozostałe pakiety używane w~pracy, to już zależy od
%%%%%%%%%%%%%% autora, więc może być tego więcej
\usepackage{fancyhdr}
\usepackage{graphicx}
\usepackage{amsmath}
\usepackage{amsthm}
\usepackage{amssymb}
\usepackage{url}
\usepackage{longtable}
\usepackage{array,hhline}
\usepackage{tabu}


%%%%%%%% hyperref po to by przeglądarka pedeef ukazywala na odwołania
%%%%%%%% prawidłowo skonstruowane za pomocą \ref, \cite i.t.d. jako
%%%%%%%% hiperłącza
\usepackage{hyperref}



%%%%% dla fanów ``profesjonalnych'' tabel w~stylu zachodnich książek

\usepackage{booktabs} \heavyrulewidth=1.5bp \lightrulewidth=0.5bp


%%%%%%%%%%% poniżej uniwersalny sposób na ucywilizowanie znaków 
%%%%%%%%%%% niewiększości, niezależny od pakietu {polski}, ale za to 
%%%%%%%%%%% zależny od {amssymb}, ma tą zaletę, że działa np. z Timesem
%%%%%%%%%%% w matematyce
\let\leq\leqslant\let\le\leq\let\geq\geqslant\let\ge\geq


%%%%%%% jeżeli będziesz chciał włączać do swojej pracy fragmenty programów, 
%% to ponizsza linijka przyda się, jeśli nie - usuń ją

\usepackage{fancyvrb}


%%%%%%%%%%%%%%%%% struktury do tworzenia twierdzeń i~tym podobnych

\theoremstyle{plain}
\newtheorem{twier}{Twierdzenie}[chapter] % pierwsze to nazwa środowiska,
                                      %drugie to wyświetlana nazwa
				% to trzecie w~nawiasie kwadratowym
				% wskazuje numer dolepiony z~lewej do
				% numeru twierdzenia (tu numer
				% 'chapter', 
\newtheorem{lemat}{Lemat}[chapter]

\theoremstyle{definition}
\newtheorem{defi}{Definicja}[chapter]

\theoremstyle{definition}
\newtheorem{uwaga}{Uwaga}[chapter]
\newtheorem{wniosek}{Wniosek}[chapter]

%%%%% więcej możliwości w~dokumentacji amsthm



%%%%%%%%%%%%%%%%%%%%%%%%%%%%%%%%%%%%%%%%%5
%%%%%%%%%%%%%%%%%%%%%%%%%%%%%%%%%%%%%%%%%%
%%%%%%%%% wcięcie akapitowe %%%%%%%%%%%%%%
%%%%%%%%%%%%%%%%%%%%%%%%%%%%%%%%%%%%%%%%%%
%%%%%% ustawić w~zaleceń i~gustu %%%%%%%%%
%%%%%%%%%%%%%%%%%%%%%%%%%%%%%%%%%%%%%%%%%%
%%%%%%%% zalecenie na stronie wydziałowej
%%%%%%%% było 1.25cm i wyglądało jakoś 
%%%%%%%% śmiesznie duże, więc spłoszony nieco
%%%%%%%% wpisałem 1cm, ale uważny czytelnik już
%%%%%%%% zapewne się domyśli, że podmiana napisu 
%%%%%%%% =1cm na =1.25cm sprawi, że wcięcia na początku
%%%%%%%% akapitu ustawią się na (nieco przydużą)
%%%%%%%% wartość 1.25cm 

\parindent=1cm



%%%%%%%%%%%%%%%%%%%%%%%%%%%%%%%%
%%%%% tu pewne poluzowanie rozmieszczenia elementów tabelek
%%%%% możecie sobie poeksperymentować, by dopasować do swych
%%%%% gustów, a przede wszystkim gustów promotorów (promotorek)
  \tabcolsep=4mm          
  %\renewcommand\arraystretch{1.3}
%%%%%%%%%%%%%%%%%%%%%%%%%%%%%%%%%%



%%%%%%%%% teraz żywa pagina (aka 'running headline') i~numerowanie stron
%%%%%%%%%%%%%%%%%%%%%%%%%%%%%%%%%%%%%%%%%%%%%%%%%%%%%%%%%%%%%%%%%%%%%%%%
%%%%%na górze mają być śródtytuły, na dole (po stronie zewneętrznej)
%%%%%numery stron. Poszedłem kapkę dalej i~na stronach ropoczynających
%%%%%rozdział nie ma paginy (górki).
%%%%% Oczywiście jeśli ostatnia strona
%%%%% jest pusta (uzupełnia jeno parzystość) to tam żadnej stopki ani 
%%%%% górki byc mnie może - ma być pusta.
%%%%%%%%%%%%%%%%%%%%%%%%%%%%
\pagestyle{fancy}
\fancyhead{}% oczyszczenie
\fancyhead[RO]{\rightmark} %% na nieparzystych 'podległe' śródtytuły
\fancyhead[LE]{\leftmark} %% na parzystych 'ważniejsze'
\fancyfoot{}% oczyszczenie
\fancyfoot[RO,LE]{\arabic{page}}  %% numer na dole (po prawej na
%% nieparzystych, po lewej na parzystych)
\renewcommand\headrulewidth{0.4pt} %%% cienka hrulka oddzielająca paginę
                                    %%% od kolumny tekstu
\fancypagestyle{closing}{%%%%%% to styl dla stron zamykających rozdział
\fancyhead{}% oczyszczenie
\fancyhead[RO]{\rightmark} %% na nieparzystych 'podległe'
\fancyhead[LE]{\leftmark} %% na parzystych 'ważniejsze'
\fancyfoot{}% oczyszczenie
\fancyfoot[RO,LE]{\arabic{page}}  %% numer na dole (po prawej na
                                  %% powyższą linijkę usuń jeśli nie
				  %% chcesz numerów na niepełnych
				  %% kolumnach (zamykających rozdział)
\renewcommand\headrulewidth{0.4pt}
}
\fancypagestyle{opening}{%%% styl stron rozpoczynających rozdział
\fancyhead{}% oczyszczenie
\fancyfoot{}% oczyszczenie
\fancyfoot[RO,LE]{\arabic{page}}  %% numer na dole (po prawej na
\renewcommand\headrulewidth{0pt}
}
\fancypagestyle{plain}{%%%% styl zwykły, niektóre konstrukcje
                       %%%% (typu \titlepage, którego ja tu nie używam
                       %%%% ale może są jakieś inne o których nawet nie chce 
                       %%% mi się myśleć, więc dla spokoju robię to po swojemu
\fancyhead{}% oczyszczenie
\fancyfoot{}% oczyszczenie
\fancyfoot[RO,LE]{\arabic{page}}  %% numer na dole (po prawej na
\renewcommand\headrulewidth{0pt}
}

%%%%%%%%%%%%%%%%%%%%%%%%%%%%%%%%%5
%%%%%%%%%%%%%%%%%%%%%%%%%%%%%%%%%%
%%% lekka modyfikcja 'markow' do paginy
%%% uznalem, ze jesli ktos nie da \section (np we wstepnie czy
%%% podsumowaniu to niech na obu sronach w~paginie pojawia sie tytuł
%%% chaptera, bo standardowo, to na nieparzystej stronie w takiej sytuacji
%%% nad górną linią ziałaby pustka, co mogłoby wprowadzać konsternację
\makeatletter
    \def\chaptermark#1{%
      \markboth{%
        \ifHeadingNumbered
     \if@mainmatter
     \@chapapp\
            \thechapter.\enspace
          \fi
        \fi
        #1}{%
        \ifHeadingNumbered
     \if@mainmatter
     \@chapapp\
            \thechapter.\enspace
          \fi
        \fi
        #1%
	}}%
    \def\sectionmark#1{%
      \markright{%
        \ifHeadingNumbered \thesection.\enspace \fi
        #1}}
%%%%%%%%%%%%%%%%%%%%%%%%%%%%%%%%%%%%%%%%%%%%%%%
%%%%%%%%%%%%%%%%%%%%%%%%%%%%%%%%%%%%%%%%%%%%%%%%
%%%%%%%%%%%% wielkości czcionek dla chapter i~section
%%%%%%%%%%%% 16 dla rozdziału, 14 dla podrozdziału - te domyślne
%%%%%%%%%%%% w klasie mwbk były całkiem ładne, ale żeby nie było
%%%%%%%%%%%% że nie potrafię ustawić
%%%%%%%%%%%%%%%%%%%%%%%%%%%%%%%%%%%%%%%%%%%%%%%%%%%
\SetSectionFormatting[breakbefore,wholewidth]{chapter}
        {0\p@}
        {\FormatRigidChapterHeading{6.4\baselineskip}{12\p@}%
	{\large\@chapapp\space}{\fontsize{16}{19}\selectfont}}
        {1.6\baselineskip}
\SetSectionFormatting{section}
        {24\p@\@plus5\p@\@minus2\p@}
	{\FormatHangHeading{\fontsize{14}{16}\selectfont}}
        {10\p@\@plus3\p@}
\makeatother	



%%%%%%%%%%%%%%%%%%%%%%%%%%%%%%%%%%%%%%%%%%%%%%
%%%%%%%%%%%%%%%%%%%%%%%%%%%%%%%%%%%%%%%%%%%%%%
%%%%%%%%%%%%%% jakies inne pomocnicze definicje, ja na przykład lubię
% \R
%%%%%%%%%%%%%%%%%%%%%%%5
%%%%%%%%%%%%%%%%%%%%%%%
%%%% tak naprawdę są t potrzebne tylko po to
%%%% by zadziałały przykłady poniżej w tekście
%%%% które w sposób dość losowy zostały 
%%%% pobrane z jakichś moich starych plików
%%%%%%%%%%%%%%%%%%%%%%%%%%%%%%%%%%
%%%%%%%%%%%%%%%%%%%%%%%%%%%%%%%%%%%
%%%% w realnej pracy te poniższe śmieci możecie oczywiście
%%%% usunąć
%%%%%%%%%%%%%%%%%%%%%%%%%%%%
\newcommand\R{\mathbb{R}}
\newcommand{\ff}{\mathbf{f}}
\newcommand{\hh}{\mathbf{h}}
\newcommand{\xx}{\mathbf{x}}
\newcommand{\yy}{\mathbf{y}}
\newcommand{\zz}{\mathbf{z}}
\newcommand{\gggg}{\mathbf{g}}
\newcommand{\skalar}[2]{\pmb{\langle}#1,#2\pmb{\rangle}}
%%%%%%%%%%%% koniec tych dodatkowych definicji

%%%%%% trocę więcej ``luzu'' przy rozmieszczaniu {fgur} i~{table}

 \renewcommand{\topfraction}{0.9}	% max fraction of floats at top
    \renewcommand{\bottomfraction}{0.8}	% max fraction of floats at bottom
    %   Parameters for TEXT pages (not float pages):
    \setcounter{topnumber}{2}
    \setcounter{bottomnumber}{2}
    \setcounter{totalnumber}{4}     % 2 may work better
    \setcounter{dbltopnumber}{2}    % for 2-column pages
    \renewcommand{\dbltopfraction}{0.9}	% fit big float above 2-col. text
    \renewcommand{\textfraction}{0.07}	% allow minimal text w. figs
    %   Parameters for FLOAT pages (not text pages):
    \renewcommand{\floatpagefraction}{0.7}	% require fuller float pages
    % N.B.: floatpagefraction MUST be less than topfraction !!
    \renewcommand{\dblfloatpagefraction}{0.7}	% require fuller float pages
    % remember to use [htp] or [htpb] for placement

    
%%% DWA proste polecenia służące do ujednolicenia podawania źródeł przy rysunkach i~tabelkach    
    
    \newcommand\zrodlo[1]{\par\vspace{-3mm}{\small\textit{Źródło: }#1 }}
    \newcommand\zrodlotab[1]{{\par\vspace{2mm}\small\textit{Źródło: }#1 }}

\raggedbottom   %%% to znaczy, że nie będzie siłowego wyrównywania typowych
                %%     stron do jednakowej wysokości

\linespread{1.3}

\begin{document}

%%%%%%%%%%%%%%%%%%%%%%%%%%%%%%%%%%%%%%%%%
%%%%%%%%%%%%%%%%%%%%%%%%%%%%%%%%%%%%%%%%%
%%%%%%%% STRONA TYTUŁOWA %%%%%%%%%%%%%%%%

\thispagestyle{empty}  % tu wszak nie chcemy żadnej numeracji stron


%%%%%%%%%%%%%%%%%%%%%%%%%%%%%%%%%%%%%%%%%%%%%%%%%%%%%%%%%%%%%%%
%%%%%tytuły definiuje jako makrodefinicje, gdyż zamierzam je%%%
%%%%%powtórzyć na stronie ze streszczeniami, to nic nie boli%%%
%%%%%a gwarantuje, że będą one takie same, i~tak ma być.%%%%%%%
%%%%%%%%%%%%%%%%%%%%%%%%%%%%%%%%%%%%%%%%%%%%%%%%%%%%%%%%%%%%%%%
\newcommand\tytul{Zastosowanie modeli mieszanych w analizie rozwoju pandemii wirusa COVID-19 na świecie}

\newcommand\tytulangielski{The Use of Mixed-Effects Models in the Analysis of the COVID-19 Pandemic in the World}


\begin{center}


{\large \bf POLITECHNIKA LUBELSKA}

{\bf WYDZIAŁ PODSTAW TECHNIKI}

\emph{Kierunek:} MATEMATYKA

%%% BEZ SPEC.!!! \emph{Specjalność:} Matematyka w~finansach i~ubezpieczeniach

\vfill %%%% \vfill to taki rozpychacz w pionie, pcha ile mu pozwolą
     

\includegraphics[width=3.5cm]{rys/logopl}

\vfill

\textbf{Praca inżynierska}

\vfill
\vfill
\vfill

\large
\tytul

\vfill

\emph{\tytulangielski}


\vfill
\vfill
\vfill
\vfill
\vfill

\begin{tabular}[t]{l}
\emph{Praca wykonana pod kierunkiem:}
\\
dra Dariusza Majerka
\end{tabular}
\hfill
\begin{tabular}[t]{l}
	\emph{Autor:}
\\
Alicja Hołowiecka\\
nr albumu: 89892 
\end{tabular}

\vfill
\vfill
\vfill

\textbf{Lublin 2021}

\end{center}


%%%%% koniec tytułów


%%%%%%%%%%%%%%%%%%%%%%%5
%%%%%%%%%%%%%%%%%%%%%%
%%% teraz spis treści
%%%%%%%%%%%%%%%%%%%%%
%%% pamiętaj! po jakiejkolwiek zmianie w tekście
%%% która wpływa na zmianę spisu treści, spis będzie dobry co najmniej
%%% po dwóch przebiegach latexa - to samo dotyczy odwołań do wzorów i literatury
%%% ogólnie to przed wydrukiem warto przelatexować o jedne raz więcej niż
%%% to się wydaje konieczne, no chyba że korzystamy z funkcji typu BUILD
%%% w zintegrowanym systemie wspomagającym TeX, BUILD powinien takie sprawy 
%%% wziąć pod uwagę

\tableofcontents


\chapter*{Wstęp}




Pandemia choroby COVID-19 jest wydarzeniem, które wstrząsnęło całym światem w roku 2019. Właściwie nikt chyba nie może powiedzieć, że nie poczuł się dotknięty przez sytuację związaną z rozprzestrzenianiem się wirusa. Pierwsze przypadki pojawiły się pod koniec 2019 roku we wschodnich Chinach, w mieście Wuhan. Na początku 2020 roku chorowali już obywatele większości państw na świecie. Na moment pisania tej pracy, sytuacja nadal nie jest opanowana i nie wiadomo, jak się rozwinie.

Biorąc to pod uwagę, tym ważniejszy wydaje się temat poruszany w tej pracy. Wiele jednostek naukowych podejmuje próby znalezienia odpowiedniego modelu, aby przewidzieć rozwój pandemii. Przedstawione w tej pracy modele mieszane co prawda nie pozwalają na dokładną predykcję, ale są dobrym narzędziem, aby odkryć, które czynniki mają wpływ na rozwój pandemii w przeciętnym kraju.











\chapter{Teoretyczne podstawy badań własnych}
W tej części pracy przedstawimy metody matematyczne, które zostaną użyte w rozdziale poświęconym modelom rozwoju pandemii. Przybliżymy zagadnienia związane z modelami liniowymi: regresję prostą i wieloraką, metody estymacji parametrów modelu liniowego i badania ich istotności oraz interpretację takich modeli. Powiemy także o tym, kiedy jest potrzebna transformacja zmiennych. Następnie rozszerzymy te pojęcia o modele mieszane, skupiając się przede wszystkim na estymacji, wyborze najlepszego modelu oraz interpretacji efektów stałych i losowych.
\section{Modele liniowe}
Na początek przypomnimy podstawowe wiadomości o modelach liniowych. Model \textbf{regresji prostej} ma postać $$y=x \beta_1+\beta_0 + \varepsilon,$$ i ma on przedstawiać relację pomiędzy dwoma zmiennymi ilościowymi. Oszacowania parametrów $\beta_1$ i $\beta_0$ obliczamy następująco:
$$\hat{\beta_1}=\frac{Cov(x,y)}{Var(x)},$$
$$\hat{\beta_0}=\overline{y}-\overline{x}\hat{\beta_1}.$$ Zmienną $y$ nazywamy zmienną zależną, a $x$ - niezależną.

Jeżeli w modelu występuje więcej niż jedna zmienna niezależna, to mówimy o \textbf{regresji wielorakiej} (lub wielokrotnej). Wówczas model ma postać: $$y=\beta_0+\beta_1 x_1+\beta_2 x_2 + \ldots + \beta_k x_k + \varepsilon,$$ lub w zapisie macierzowym $$\yy=\mathbf{X}\beta+\varepsilon$$
gdzie $\varepsilon$ to niezależne (tzn. $Cov(\varepsilon_i, \varepsilon_j)=0 \text{ dla }i\neq j$) zakłócenie losowe o rozkładzie normalnym ze średnią 0 i wariancją $\sigma^2$.

\subsection{Metody estymacji parametrów modelu liniowego}
Aby oszacować wartości parametrów modelu liniowego, możemy wykorzystać poniższe metody estymacji:
\begin{enumerate}
	\item \textbf{Metoda najmniejszych kwadratów} (ang. \emph{Ordinary Least Squares, OLS}) - w metodzie tej minimalizujemy błąd kwadratowy, czyli sumę kwadratów reszt, którą oznaczamy $RSS$ (ang. \emph{Residual Sum of Squares}).
	$$RSS= \sum_{i=1}^{n}(y_i-\hat{y_i})^2$$
	Twierdzenie Gaussa-Markowa mówi, że taki estymator jest najlepszym (w sensie najmniejszej wariancji) liniowym nieobciążonym estymatorem (ang. \textit{BLUE, Best Linear Unbiased Estimator}) przy założeniach, że $E(\varepsilon_i)=0$ i $Var(\varepsilon_i)=\sigma^2$ dla każdego $i$ oraz $Cov(\varepsilon_i, \varepsilon_j)=0$ dla $i \neq j$.

	\item \textbf{Metoda największej wiarogodności} (ang.\textit{Maximum Likelihood, ML}) polega na maksymalizacji wartości funkcji prawdopodobieństwa ze względu na $\beta$ (w praktyce maksymalizujemy zwykle logarytm z tej funkcji). Estymator wariancji otrzymany tą metodą wygląda następująco:
	$$\hat{\sigma}^{2}_{ML}=\frac{RSS}{n}.$$
	Estymując $\sigma^2$, maksymalizujemy funkcję wiarogodności zarówno ze względu na $\beta$, jak i $\sigma^2$.
	Estymatory uzyskane tą metodą są asymptotycznie nieobciążone.
	
	\item \textbf{Resztowa metoda największej wiarogodności} (ang. \textit{Residual/Restricted Maximum Likelihood Method, REML}) - z estymacji parametru $\sigma^2$ usuwamy wpływ parametrów zakłócających $\beta$. W tym celu redukuje się wektor $y$ do podprzestrzeni ortogonalnej do macierzy $X$. Estymator wariancji otrzymany metodą REML wygląda następująco:
	$$\hat{\sigma}^2_{REML}=\frac{RSS}{n-p}$$
	Estymatory uzyskane tą metodą są nieobciążone \cite{biecek}.
\end{enumerate}

W przypadku modeli liniowych, estymacja współczynników $\beta$ jest prosta, więc rzadko używa się do tego metody REML. Metoda ta jest wykorzystywana przy szacowaniu parametrów bardziej skomplikowanych modeli. Do modeli liniowych najczęściej stosuje się estymację metodą najmniejszych kwadratów.

\subsection{Badanie istotności parametrów}

Aby zbadać istotność współczynników modelu liniowego, weryfikujemy hipotezę postaci $H_0: \beta_i = 0$ przeciw hipotezie alternatywnej $H_1: \beta_i \neq 0.																																																																																																																																																																																																																																																																																																																																																																																																																																																																																																																													$ Do zweryfikowania tej hipotezy wykorzystujemy \textbf{test Walda}. Statystyka testowa ma postać $$T=\frac{\hat{\beta_i}}{\operatorname{se}(\hat{\beta_i})}$$ i jest nazywana \textbf{statystyką t}. Przy założeniu prawdziwości $H_0$ statystyka ta ma rozkład t-Studenta o $n-k-1$ stopniach swobody ($n$ - liczba obserwacji, $k$ - liczba parametrów w modelu, nie licząc wyrazu wolnego).

Badanie efektów brzegowych poszczególnych zmiennych należy poprzedzić \textbf{testem F} (testem globalnym), który weryfikuje hipotezę $$H_0: \beta_1=\beta_2=\ldots=\beta_k=0$$
 przeciwko hipotezie alternatywnej $$H_1: \exists{j} \beta_j \neq 0 $$

\subsection{Interpretacja parametrów modelu liniowego}

W modelu postaci $y=\beta_0 +\beta_1 x$ dodatnia wartość $\beta_1$ oznacza, że wzrostowi $x$ towarzyszy wzrost $y$, a ujemna wartość $\beta_1$, że wraz ze wzrostem $x$, maleje $y$ \cite{rozrzut}. Jeżeli model jest dobrze dopasowany do danych, to możemy go interpretować w ten sposób, że wzrost zmiennej $x$ o 1, powoduje zmianę zmiennej $y$ o $\beta_1$. Przykłady takich zależności widać na rysunku \ref{fig:regresje}. Podobnie w przypadku modelu regresji wielorakiej
$y=\beta_0+\beta_1 x_1+\beta_2 x_2 + \ldots + \beta_k x_k$,
wzrost zmiennej $x_i$ o jedną jednostkę, przy niezmienionych poziomach pozostałych zmiennych, powoduje zmianę wartości $y$ o $\beta_i$.

\begin{figure}[!h]
	\centering
	\includegraphics[width=\linewidth]{rys/regresje}
	\caption{Przykładowe modele regresji prostej. Po lewej dla $\beta_1>0$, po prawej dla $\beta_1<0$}
	\label{fig:regresje}
	\zrodlo{\cite{rozrzut}}
\end{figure}


Miarami jakości dopasowania modelu do danych są m. in. współczynnik determinacji $R^2$ oraz skorygowany współczynnik determinacji. \textbf{Współczynnik determinacji} informuje o tym, jaka część zmienności (wariancji) zmiennej zależnej w próbie jest wyjaśniona zmiennością modelu. Przyjmuje wartości z przedziału $[0,1]$. Jeżeli w modelu występuje wyraz wolny, a do estymacji wykorzystano metodę najmniejszych kwadratów, to współczynnik determinacji można interpretować jako procent wariancji zmiennej zależnej, która jest wyjaśniana przez model (więc dopasowanie jest tym lepsze, im wartość $R^2$ jest bliższa jedności \cite{r2}). Współczynnik determinacji jest wyrażony wzorem:
$$R^2=\frac{\sum\limits_{i=1}^{n}(\hat{y_i}-\bar{y})^2}{\sum\limits_{i=1}^{n}(y_i-\bar{y})^2},$$ gdzie $y_i$ - $i$-ta obserwacja zmiennej zależnej, $\hat{y_i}$ - oszacowanie $i$-tej wartości zmiennej zależnej na podstawie modelu, $\bar{y}$ - średnia arytmetyczna zaobserwowanych empirycznie wartości zmiennej zależnej.

W przypadku modeli zawierających więcej niż jedną zmienną, zdarza się, że dodanie do modelu nowej zmiennej podniesie współczynnik $R^2$, mimo że faktycznie nie będzie poprawiała dopasowania modelu. Dlatego można także korzystać z miary nazywanej \textbf{skorygowanym współczynnikiem determinacji}, określonej wzorem:
$$\tilde{R}^2=1-\frac{n-1}{n-k-1}(1-R^2),$$
gdzie $R^2$ - współczynnik determinacji, $n$ - liczba obserwacji, $k$ - liczba zmiennych w modelu (nie licząc wyrazu wolnego) \cite{skorygowany}. Interpretacja $\tilde{R}^2$ jest taka sama, jak $R^2$, ale jeśli te dwie wartości znacznie się od siebie różnią, to warto interpretować raczej skorygowany współczynnik determinacji niż zwykłe $R^2$.

\subsection{Transformacja zmiennych}

Jeżeli obserwacje charakteryzują się wariancją, która rośnie lub maleje wraz ze wzrostem zmiennej niezależnej, to przydatna może być \textbf{transformacja zmiennych}. Często używana jest na przykład \textbf{transformacja logarytmiczna}, która jest łatwa w interpretacji (zmiany wartości zlogarytmowanej odpowiadają zmianom procentowym w oryginalnej skali). Przekształcenie takie warto stosować, kiedy zaobserwowane wartości zmiennej charakteryzują się silną asymetrią prawostronną. Jednakże, nie można go stosować, jeśli pojawiają się wartości niedodatnie. Jeżeli oryginalne obserwacje oznaczymy jako $x_1, x_2,\ldots, x_n$, to przekształcone obserwacje $w_1, w_2,\ldots,w_n$ będą takie, że $w_i=\log(x_i)$ dla $i \in \lbrace{1,2,\ldots,n\rbrace}$ \cite{forecasting}.

Innym rodzajem transformacji są \textbf{transformacje potęgowe}, np. pierwiastki kwadratowe lub sześcienne. Te przekształcenia nie zawsze są tak proste w interpretacji jak logarytmiczne. Zapisujemy je jako $w_i=x_i^p$, gdzie $i \in \lbrace{1,2,\ldots,n\rbrace}$, a $p$ to potęga, jaką przekształcamy obserwacje \cite{forecasting}.

\textbf{Transformacja Boxa-Coxa} jest rodziną transformacji zawierającą zarówno przekształcenia logarytmiczne, jak i potęgowe. Przekształcenie Boxa-Coxa ma postać:
$$g^{(\lambda)}(X)=\begin{cases}
\log(X), & \text{ gdy } \lambda=0 \\
\frac{X^{\lambda}-1}{\lambda},&\text{ gdy } \lambda \neq 0
\end{cases}. \cite{boxcox}$$

\section{Modele mieszane} 
W powyżej opisanych modelach liniowych z efektami stałymi zakładamy niezależność kolejnych pomiarów, dlatego nie są to odpowiednie modele w przypadku, kiedy mamy np. kilka pomiarów dla pojedynczego elementu. W takiej sytuacji możemy użyć modeli liniowych z efektami mieszanymi (stałymi i losowymi), które krótko nazywamy modelami mieszanymi.

Modeli mieszanych używamy w przypadku powtarzanych pomiarów bądź hierarchicznej, czyli zagnieżdżonej struktury. Takie dane charakteryzują się korelacją między obserwacjami z tej samej grupy, co nie pozwala na użycie modelu liniowego z efektami stałymi, ponieważ założenie o braku seryjnej korelacji błędu modelu nie jest spełnione. Dlatego do modelu wprowadza się czynnik losowy.  Czynnik stały jest pewnym parametrem, którego wartość estymujemy na podstawie próbki, natomiast czynnik losowy jest zmienną losową, dla której próbujemy oszacować parametry jej rozkładu \cite{faraway}. W przypadku efektu stałego interesuje nas jego wielkość (średnia), natomiast przy efektach losowych bierzemy pod uwagę jedynie fakt, że wprowadzona zmienna wnosi do modelu pewną zmienność (a dokładniej, pozwala odjąć tą zmienność od całkowitej zmienności) i szacujemy wariancję lub odchylenie standardowe, a nie parametry rozkładu. Ponadto efektów losowych można się spodziewać wtedy, gdy nie kontrolujemy wszystkich poziomów zmiennej niezależnej. Przykładową sytuacją, gdzie możemy użyć modelu mieszanego, jest badanie działania leku na grupie pacjentów, gdzie dokonujemy kilku pomiarów na danym pacjencie. W tym przypadku nie interesuje nas efekt konkretnego pacjenta, ale raczej wpływ leku na przeciętną osobę. Dodatkowo zakładamy, że pacjenci byli losowo wybrani. W modelu mieszanym, wpływ konkretnego pacjenta będzie traktowany jako czynnik zakłócający.




\subsection{Metody estymacji parametrów modelu mieszanego}

\textbf{Modelem mieszanym} nazywamy model postaci
$$y=X\beta +Z u + \varepsilon$$
gdzie $X$ - macierz zmiennych będących efektami stałymi, $Z$ - macierz zmiennych będących efektami losowymi, $\beta$ to wektor nieznanych efektów stałych, $\varepsilon \sim \mathcal{N}(0, \sigma^2 I_{n\times n})$ to zakłócenie losowe, a $u \sim \mathcal{N} (0, \sigma^2D)$ to wektor zmiennych losowych odpowiadających efektom losowym \cite{biecek}.

Znając macierz $D$, możemy estymować parametry $\beta$ uogólnioną metodą najmniejszych kwadratów. Do estymowania $D$ możemy użyć np. metody największej wiarogodności.

Do oceny wartości parametrów modelu mieszanego można stosować metody ML (Największej Wiarogodności) oraz REML (Resztowej Największej Wiarogodności), wspomniane w tej pracy przy okazji modeli liniowych. W przypadku modeli mieszanych obydwoma metodami możemy uzyskać estymatory obciążone, ale to obciążenie jest zazwyczaj mniejsze w przypadku estymatorów uzyskanych metodą REML.


Estymacja parametrów modelu mieszanego jest trudnym zagadnieniem. Można wyróżnić dwa podejścia, jedno z nich wykorzystuje własności macierzy $ZDZ^T$, a drugie polega na rozwijaniu metod numerycznych używanych do znalezienia maksimum funkcji wiarogodności. Przykładową metodą jest użycie algorytmu Newtona-Rapshona - iteracyjnej metody optymalizacji. To podejście jest dobre dla zbioru danych o praktycznie dowolnym rozmiarze, ale ze względu na złożoność pamięciową rzędu $O((p+q)^2)$, źle sprawdza się dla modelu z dużą liczbą parametrów do oszacowania. Pod tym względem bardziej efektywne jest rozwiązanie przy wykorzystaniu własności macierzy rzadkich. Ta właśnie metoda zostanie przedstawione poniżej.

\textbf{Macierz rzadka} jest to macierz, w której większość elementów ma wartość zero. Algorytmy dla macierzy rzadkich są zwykle szybsze niż analogiczne algorytmy dla macierzy gęstych. Zamiast przechowywać wszystkie wartości takiej macierzy, wystarczy zapisać w pamięci wartości i indeksy elementów, które są różne od zera. Macierze rzadkie w praktyce maja często tak wielki rozmiar, że niemożliwe by było opracowanie na nich zwykłymi algorytmami \cite{rzadka}. W modelu mieszanym macierz $Z$ jest macierzą rzadką. Często również macierz X jest taką macierzą.

Z definicji modelu mieszanego $u \sim \mathcal{N}(0,~\sigma^2D)$, gdzie $\sigma^2$ to wariancja wektora $\varepsilon$. Niech $$D=\Lambda \Lambda^T,$$ gdzie $\Lambda$ to macierz trójkątna dolna. Macierz $D$ (a przez to także macierz $\Lambda$) jest parametryzowana wektorem $\theta$. Obierzmy także wektor $w$ taki, że $$u=\Lambda w.$$ Taki wektor $w$ ma rozkład $\mathcal{N}(0,~I_{q \times q})$.



Rozkład warunkowy $y|u$ jest rozkładem $\mathcal{N}(X\beta+Zu,~\sigma^2I)$. Ale ponieważ nie obserwujemy $u$, a jedynie $y$, to aby wnioskować o $u$, będziemy rozważać gęstość $u|y$. Z twierdzenia Bayesa \cite{bayes} mamy $$f_{u|y}=\frac{f_{y|u}f_u}{f_y}.$$ Zaczniemy od wyznaczenia gęstości łącznej $f_{y,u}=f_{y|u}f_u$.

$$f_{y,u}(\beta, \theta, \sigma^2)=f_{y|u}(\beta, \theta, \sigma^2)f_u(\beta, \theta, \sigma^2)=$$
$$=\frac{\exp(-(y-X\beta -Z \Lambda u)^T(y-X\beta-Z\Lambda u)/(2\sigma^2))}{(2\pi \sigma^2)^{n/2}}\cdot \frac{\exp(-u^Tu/(2\sigma^2))}{(2\sigma^2)^{q/2}}=$$
$$=\frac{\exp(-(||y-X\beta-Z\Lambda u||^2+||u||^2)/(2\sigma^2))}{(2\pi \sigma^2)^{(n+q)/2}},$$ gdzie $||a||^2$ to suma kwadratów współrzędnych wektora $a$.

Minimalizacja tej gęstości po $u$ lub $\beta$ jest równoważna minimalizacji sumy kwadratów reszt z karą za współczynniki $u$:
$$r^2(\theta, \beta, w)=||y-X\beta-Z\Lambda w||^2+||w||^2.$$ Funkcję $r^2(\theta, \beta, w)$ nazywamy \textbf{sumą kwadratów reszt z karą} i oznaczamy $PRSS$ (\textit{Penalized Residual Sum of Squares}). Przez $r_{\theta}$ określimy $\min\limits_{w,\beta} r^2(\theta, \beta,w)$. Wartości minimalizujące $PRSS$ po $w$ i $\beta$ można znaleźć, rozwiązując układ równań liniowych \ref{eq:ukladPRSS}.

\begin{equation} \label{eq:ukladPRSS}
\begin{bmatrix}
X^TX & X^TZ\Lambda \\
\Lambda^TZ^TX & \Lambda^TZ^TZ\Lambda +I
\end{bmatrix}
\begin{bmatrix}
\beta \\
w
\end{bmatrix}=
\begin{bmatrix}
X^Ty\\
\Lambda^TZ^Ty
\end{bmatrix}
\end{equation}

Ten układ jest rozwiązywalny efektywnie nawet dla bardzo dużych $q$, dzięki wykorzystaniu rzadkiej dekompozycji Choleskiego \cite{cholesky}. Po lewej stronie równania macierz $\Lambda^TZ^TZ\Lambda+I$ jest macierzą rzadką. Można ją przedstawić jako

$$\begin{bmatrix}
A & 0 \\
B & L
\end{bmatrix}
\begin{bmatrix}
A^T & B^T\\
0 & L
\end{bmatrix}
$$
gdzie $A$ i $B$ to nieduże macierze (o ile $p$ jest nieduże), a $L$ to rzadki pierwiastek Choleskiego macierzy $\Lambda^TZ^TZ\Lambda+I_{q\times q}$, czyli jest to rzadka macierz trójkątna dolna spełniająca warunek $$LL^T=\Lambda^TZ^TZ\Lambda +I_{q \times q}.$$ Użycie dekompozycji Choleskiego macierzy rzadkich, która w wyniku także daje macierz rzadką, jest kluczowym momentem pozwalającym na operowanie na dużych macierzach. W tym celu kolumny macierzy $Z$ odpowiednio się permutuje. Dzięki temu można operować na macierzach, które w postaci pełnej nie mieściłyby się w pamięci.

Po wyznaczeniu dekompozycji Choleskiego łatwo rozwiązać układ równań \ref{eq:ukladPRSS}. Co więcej, mamy następującą zależność:

\begin{equation} \label{eq:2log}
-2\log l(\theta,\beta,\sigma|y)=\log(2\pi \sigma^2)+\log(|L|^2) +\frac{r^2_{\theta}}{\sigma^2}, 
\end{equation}
gdzie $|L|$ to wyznacznik macierzy $L$ (która jest trójkątna, więc łatwo go policzyć). Minimalizując powyższe wyrażenie po $\sigma^2$, otrzymujemy warunkowy estymator wariancji (dla zadanego $\theta$):
$$\hat{\sigma}_{\theta}^2=\frac{r^2_{\theta}}{n}$$
Podstawiając tą ocenę wariancji do równości \ref{eq:2log}, otrzymujemy funkcję wariancji w zależności od parametru $\theta$

$$-2 \log l(\theta|y)=\log(|L|^2)+n+n\log\left(\frac{2\pi r^2_{\theta}}{n}\right)$$

Wartość tej funkcji możemy wyznaczyć efektywnie, nawet dla dużych $p+q$, a sama funkcja wiarogodności zależy jedynie od parametru $\theta$, którego wymiar $g$ jest zazwyczaj nieduży (jest to liczba komponentów wariancyjnych). Maksymalizację tak opisanej funkcji wiarogodności po przestrzeni parametrów o niewielkim wymiarze wykonuje się standardowymi algorytmami numerycznymi.
Wyznaczywszy metodą ML (lub REML) ocenę parametru $\hat{\theta}$, możemy obliczyć oceny pozostałych parametrów modelu. 

W powyższej metodzie macierz $V=\sigma^2(I+ZDZ^T)$ mogła mieć prawie dowolną postać. W wielu praktycznych sytuacjach macierz $D$, a tym samym macierz $V$, ma bardzo protą strukturę.
Rozważmy model niezależnych $g$ komponentów losowych, każdy komponent złożony z niezależnych $q_i$ efektów takich, że $\sum\limits_i q_i=q$. Dodatkowo oznaczmy wariancję kolejnych $q_i$ przez $\sigma^2_i=\sigma^2\theta_i$. W takim modelu macierz $D$ jest macierzą diagonalną

$$D=\begin{bmatrix}
\theta_1I_{q_1} & 0 & \cdots & 0 \\
0 & \theta_2I_{q_2} & \cdots & 0 \\
\cdots & \cdots & \ddots & \cdots \\
0 & 0 & \cdots & \theta_gI_{q_g}
\end{bmatrix}$$
Dla takiej macierzy $D$ macierz wariancji $y$ można wyrazić jako

$$Var(y)=V=\sigma^2(I+ZDZ^T)=I\sigma^2+\sum_{i=1}^{g}\sigma_i^2Z_iZ_i^T$$
gdzie $Z_i$ to macierz złożona z kolumn macierzy $Z$ odpowiadających tym efektom losowym, które mają wariancję $\sigma^2_i$. Każda $Z_i$ to jeden komponent wariancyjny.

\begin{uwaga}
	Może się zdarzyć, że w wyniku estymacji otrzymamy ujemne oceny pewnych parametrów $\sigma_i^2$. Oczywiście nie można takich wartości interpretować jako oceny wariancji. Problem z ujemnymi wartościami $\hat{\sigma}^2_i$ można rozwiązać na kilka sposobów, np.
	\begin{itemize}
		\item wartość ujemną zastąpić przez 0. Jest to metoda najprostsza, ale generuje obciążenie;
		\item zastosować optymalizację z ograniczeniami. Na przestrzeni parametrów zadajemy liniowe ograniczenia i szukamy maksimum funkcji wiarogodności na zbiorze ograniczonym do nieujemnych parametrów;
		\item zamiast $\sigma_i^2$ można używać innych parametrów, które da się przekształcić w nieujemne oceny $\sigma^2_i$. Przykładowo dla nowej parametryzacji $\gamma_i=\log(\sigma^2_i)$ możemy optymalizować funkcję wiarogodności ze względu na parametry $\gamma_i$ po całej prostej, a następnie otrzymane oceny $\hat{\gamma}_i$ możemy przekształcić na dodatnie oceny $\hat{\sigma}_i^2=\exp(\hat{\gamma}_i)$. Wadą tego podejścia jest niemożliwość uzyskania oceny z brzegu przedziału, tzn. nie otrzymamy nigdy oceny $\hat{\sigma}^2_i=0$ \cite{biecek}.
	\end{itemize}
\end{uwaga}


\subsection{Badanie istotności parametrów i wybór najlepszego modelu}

W modelach mieszanych konieczne jest zbadanie istotności dla efektów stałych oraz losowych. Dla efektów stałych testujemy hipotezę $H_0: \beta_i=0$ przeciwko hipotezie alternatywnej $H_1: \beta_i \neq 0,$ a dla komponentów wariancyjnych weryfikujemy hipotezę $H_0: \sigma^2_j=0$ przy jednostronnej hipotezie alternatywnej  $H_1: \sigma^2_j>0.$ 



Metody, które mają zastosowanie dla modeli liniowych z efektami stałymi, nie zawsze dają się zastosować w przypadku modeli mieszanych. Wymienimy teraz kilka metod doboru najlepszego modelu i opiszemy, które z nich są najskuteczniejsze \cite{faraway}.

\begin{enumerate}
	\item \textbf{Iloraz wiarogodności} (ang. \textit{likelihood ratio}) - tworzymy dwa zagnieżdżone modele: model 0 - niezawierający elementów, których istotność chcemy zbadać, i model 1, który zawiera te elementy. Pozostałe zmienne muszą być takie same w obu modelach.
	Statystyka testowa wygląda następująco:
	$$2(l(\hat{\beta_1}, \hat{\sigma_1}, \hat{D_1}|y)-l(\hat{\beta_0}, \hat{\sigma_0}, \hat{D_0}|y)),$$
	gdzie $l$ - logarytm z funkcji prawdopodobieństwa (ang. \textit{Log Likelihood}). Tego testu nie można używać do modeli wyznaczonych metodą REML \cite{faraway}.
	
	\item \textbf{Test F dla efektów stałych} - metoda taka sama jak ta używana przy modelach z efektami stałymi. W przypadku modeli mieszanych może sprawiać problemy, ponieważ statystyka testowa niekoniecznie musi mieć rozkład F. Należy wówczas wprowadzać poprawkę na liczbę stopni swobody. Na ogół ta metoda daje dobre rezultaty dla mniej skomplikowanych modeli, gdy układ jest zbalansowany (wszystkie grupy są równoliczne). Dla modeli bardziej skomplikowanych lub kiedy brak równoliczności, wartości $p$ oraz statystyki $t$ mogą być błędne \cite{faraway}.
	
	\item \textbf{Test permutacyjny} -można go stosować do dokładniejszego wyznaczenia wartości $p$ dla efektu stałego. Funkcja wiarogodności może być użyta jako statystyka testowa. Rozkład statystyki testowej otrzymujemy wykonując permutacje na tej kolumnie macierzy $X$, która odpowiada interesującemu nas efektowi \cite{biecek}. Dla każdej permutacji wyliczamy logarytm funkcji wiarogodności i sprawdzamy, ile z nich przekroczyło logarytm funkcji wiarogodności dla modelu z niepermutowanymi kolumnami. Testy permutacyjne mają wiele zalet, między innymi, nie muszą być spełnione założenia dotyczące rozkładu normalnego danych w próbce. Przy dostatecznie dużej liczbie permutacji, zwykle dają dokładne wartości $p$, niezależnie od wielkości próby \cite{bootstrap}.
	


	\item \textbf{Kryteria informacyjne} - służą do wyboru najlepszego spośród modeli.Pozwalają ocenić jakość dopasowania modelu, kontrolując jednocześnie jego stopień złożoności, aby uniknąć przeuczenia (z ang. \textit{overfitting}) \cite{szeregi}.
	 Najpopularniejszym jest Kryterium Informacyjne Akaikego (ang. \textit{Akaike Information Criterion, AIC}). Jest ono zdefiniowane następującym wzorem:
	$$-2(\operatorname{max} \operatorname{log} likelihood)+ 2p,$$
	gdzie $p$ to liczba parametrów modelu. Można stosować to kryterium do modeli, które różnią się jedynie efektami stałymi, gdzie liczba efektów losowych jest identyczna dla wszystkich modeli, które porównujemy. Gdyby modele różniły się liczbą efektów losowych, należałoby rozważyć, w jaki sposób zliczyć liczbę parametrów $p$ \cite{faraway}. Kryterium Akaikego jest miarą utraconej informacji, więc po obliczeniu go dla rozważanych modeli, należy wybrać ten, gdzie otrzymana wartość jest najmniejsza. Inne popularne kryteria informacyjne to m. in. Skorygowane Kryterium Informacyjne Akaikego (ang. \textit{Corrected AIC}, $AIC_c$) oraz Bayesowskie Kryterium Informacyjne Schwarza (ang. \textit{Bayesian Information Criterion, BIC}), które interpretuje się tak samo jak kryterium Akaikego \cite{szeregi}.

\end{enumerate}

Przy obliczeniach dotyczących stosunkowo małych zbiorów danych, można użyć każdej z tych metod, ale w przypadku dużej liczby obserwacji, niektóre obliczenia mogą zająć zbyt wiele czasu. Najmniej skomplikowany obliczeniowo jest test Walda, gdzie dokonujemy tylko jednej estymacji współczynników. Przy użyciu testu ilorazu wiarygodności, należy dokonać dwóch estymacji - dla modelu z i bez testowanego efektu. Stosując testy permutacyjne, musimy dokonać obliczeń setki lub tysiące razy. Dlatego w przypadku najbardziej skomplikowanych problemów zwykle stosuje się dla efektów stałych test Walda, mimo jego gorszych właściwości statystycznych \cite{biecek}.





\subsection{Predykcja z modelu mieszanego}
Proces predykcji jest trudniejszy w przypadku modelu mieszanego niż dla zwykłego modelu liniowego. Musimy zdecydować, czy uwzględnić, czy wykluczyć efekt losowy z predykcji. Efekty losowe mogą mieć różny wkład w predykcję. Mogą być całkowicie pominięte, mogą być uśrednione lub mogą być na pewnym ustalonym poziomie. Uśrednienie efektów losowych powoduje predykcję zależną od wartości efektów losowych, które zostały zaobserwowane do tej pory. Pominięcie efektów losowych powoduje predykcję na poziomie średniej populacyjnej
 \cite{prediction}.
 
 Aby lepiej przybliżyć zagadnienie predykcji z modelu mieszanego, posłużymy się przykładem badania mleczości krów. Dla 10 krów (oznaczonych literami od A do J) zmierzono ilość mleka wyprodukowaną przez każdą z nich w ciągu dnia. Pomiary powtórzono pięciokrotnie \cite{biecek}.
Model ma postać 
$$y_{milk.amount}=\mu+Z_{cow}u_{cow}+\varepsilon,$$
$$u_{cow} \sim \mathcal{N}(0, \sigma^2_{cow}).$$

Jeżeli chcielibyśmy dokonać predykcji dla nieznanej lub do tej pory niezbadanej przez nas krowy, to wynikiem byłaby ocena średniej dla całej populacji, czyli $\hat{\mu}$.
Aby dokonać predykcji dla konkretnej krowy spośród tych przebadanych, potrzebne nam są oceny efektów osobniczych krów.
Znając macierz $D$ i parametry $\beta$, \textbf{predykcje efektów losowych} $\widetilde{u}$ można wyznaczyć ze wzoru
$$\widetilde{u}=DZ^TV^{-1}(y-X\beta),$$
gdzie $V$ to macierz $\sigma^2(I+ZDZ^T)$ \cite{biecek}. Wówczas predykcja będzie sumą $\hat{\mu}$ oraz oceny efektu losowego dla odpowiedniego osobnika.

 
 
 
 
 \subsection{Interpretacja parametrów modelu mieszanego}
 W modelu mieszanym efekty stałe należy interpretować tak jak w przypadku regresji, analizy wariancji lub analizy kowariancji, w zależności od rodzaju zmiennej niezależnej. Trzeba jednak pamiętać, że oszacowane wartości współczynników reprezentują wartość średnią dla całej populacji, a dla poszczególnych obiektów badania będą się różnić o wartość oceny efektu osobniczego.
 Dla efektu losowego możemy oszacować jego wariancję. Informuje nas ona o tym, jak bardzo mogą się różnić współczynniki efektów stałych dla poszczególnych obiektów badania
 \cite{experimental}.
 
 Można wyróżnić dwa główne rodzaje modeli mieszanych. Pierwszym typem jest \textbf{model z losowym wyrazem wolnym} (ang. \textit{Random Intercept Model}). W takim modelu jedynie wyraz wolny różni się pomiędzy grupami obserwacji\cite{insurance}. Przykład takiego modelu jest przedstawiony na rysunku \ref{fig:random_types} na wykresie $A$. Widać, że prosta regresji dla poszczególnych grup może być przesunięta w górę lub w dół w stosunku do średniej globalnej ($\mu_{group}$). 
 Innym rodzajem modelu mieszanego jest model, w którym także współczynniki przy niektórych zmiennych także różnią się pomiędzy grupami \cite{insurance}. Przykładem takiego modelu jest \textbf{model z losowym wyrazem wolnym i współczynnikiem nachylenia} (ang. \textit{Random Intercept and Slope}) pokazany na rysunku \ref{fig:random_types} na wykresie $B$. Oprócz zmian w wyrazie wolnym między grupami, widać, że prosta regresji może być nachylona do osi OX pod mniejszym lub większym kątem niż prosta regresji dla całej populacji (oznaczona kolorem czarnym.)
 

 
 \begin{figure}[!ht]
 	\centering
 	\includegraphics[width=\linewidth]{rys/random_types.jpg}
 	\caption{Rodzaje modeli mieszanych}
 	\label{fig:random_types}
 	\zrodlo{\cite{brief}}
 \end{figure}

 Warto dodać, że model mieszany dla danych dotyczących zmian w czasie (ang. \textit{longitudinal data}) lub dla danych grupowanych (ang. \textit{clustered data}) może być uzyskany z odpowiedniego modelu dla danych przekrojowych (ang. \textit{cross-sectional data}) poprzez wprowadzenie do niego efektów losowych. Oznacza to, że oprócz omawianych w tej pracy modeli liniowych z efektami stałymi i losowymi (ang. \textit{Linear Mixed Effects Models, LME models}) wyróżniamy także nieliniowe modele mieszane (ang. \textit{Nonlinear Mixed Effects Models, NLME models}) oraz uogólnione liniowe modele mieszane (ang. \textit{Generalized Linear Mixed Model, GLMM}), które uzyskujemy odpowiednio z modeli nieliniowej regresji oraz uogólnionych modeli liniowych ($GLM$). Z kolei w analizie przeżycia pojawiają się mieszane modele przeżycia nazywane \textit{frailty models} \cite{insurance}. W części praktycznej tej pracy będziemy mieli do czynienia jedynie z modelami liniowymi z efektami stałymi i losowymi.
 
\chapter{Modele rozwoju pandemii}
\section{Opis zbioru badawczego}

Zbiór danych pochodzi z witryny internetowej Our World In Data \cite{owid}, gdzie dane zostały zebrane z różnych źródeł, m. in. ze Światowej Organizacji Zdrowia (WHO) oraz Europejskiego Centrum ds. Zapobiegania i Kontroli Chorób (ECDC). W zbiorze znajduje się 210 krajów, dane dotyczące terytoriów międzynarodowych oraz łącznie dla całego świata. Mamy ponad 40 kolumn z różnymi parametrami - w dalszej części pracy opiszemy, które zmienne będą przez nas użyte.

W zbiorze znajdowało się wiele braków danych. Dla każdego kraju zostały usunięte dane sprzed rozpoczęcia się epidemii na jego terytorium, dni są numerowane kolejnymi liczbami całkowitymi.
Ze zbioru danych zostały usunięte wszystkie kraje o populacji poniżej miliona mieszkańców, ponieważ w większości były to nieduże wysepki, dla których dane były wybrakowane. Oprócz tego, kilka innych krajów zostało usuniętych, ponieważ mimo większej populacji, dane były niepełne.

Do formułowania hipotez i budowania modeli będziemy się posługiwać następującymi zmiennymi \cite{codebook}:
\begin{itemize}
	\item liczba zachorowań (\textit{total\_cases\_per\_million}) - jest to liczba potwierdzonych przypadków koronawirusa w danym kraju od momentu rozpoczęcia epidemii. Zamiast wartości liczby zachorowań, będziemy używać liczby zachorowań na milion mieszkańców,
	\item czas (\textit{time}) - numer dnia od początku pandemii w danym kraju,
	\item liczba wykonanych testów (\textit{total\_tests\_per\_thousand}) - będziemy używać liczby wykonanych testów w przeliczeniu na tysiąc mieszkańców danego kraju,
	\item wskaźnik siły obostrzeń (\textit{stringency\_index}) - wskaźnik tego, jak silne obostrzenia wprowadził rząd danego kraju. Jest to kombinacja dziewięciu zmiennych, m.in. zamykanie szkół, polityka wykonywania testów, ograniczenie kontaktów międzyludzkich itp. Może przyjmować wartości od 0 do 100, im większa wartość, tym silniejsze obostrzenia w danym kraju \cite{stringency},
	\item gęstość zaludnienia (\textit{population\_density}),
	\item PKB danego kraju na osobę (\textit{GDP\_per\_capita}) - Produkt Krajowy Brutto, przeliczony na hipotetyczną walutę dolara międzynarodowego \cite{dollars},
	\item część społeczeństwa żyjąca w skrajnym ubóstwie (\textit{extreme\_poverty}) - stan na możliwie aktualny rok po 2010,
	\item powszechność występowania cukrzycy (\textit{diabetes\_prevalence}) - odsetek populacji z cukrzycą, brane pod uwagę są osoby w wieku od 20 do 70 lat, stan na rok 2017,
	\item wskaźnik rozwoju społecznego, tzw. HDI (\textit{human\_development\_index}) -  miara opisująca stopień rozwoju społeczno-ekonomicznego poszczególnych krajów, do którego pomiaru służą m. in. oczekiwana długość życia, średnia liczba lat edukacji otrzymanej przez mieszkańców w wieku co najmniej 25 lat, oczekiwana liczba lat edukacji, PKB na osobę \cite{hdi},
	\item oczekiwana długość życia (\textit{life\_expectancy}) - stan na 2019 r.
	
\end{itemize}
Dane były zbierane do dnia 1 grudnia 2020 r. 

 




\section{Wyniki}

We wszystkich modelach mieszanych kraj jest czynnikiem losowym. Modele mieszane są budowane na podstawie całego zbioru danych. Modele regresji prostej są budowane na podstawie zbioru danych, gdzie znajdują się po maksymalnie cztery obserwacje dla każdego kraju: po 3, 6, 9 i 12 miesiącach trwania epidemii. W przypadku modeli liniowych zastosowano konieczne przekształcenia zmiennych za pomocą transformacji Boxa-Coxa.

Modele są dopasowywane przy użyciu środowiska R. Do dopasowania modeli mieszanych zostały wykorzystane pakiety \texttt{lme4} oraz \texttt{lmerTest}. Funkcja \texttt{lmer} z pakietu \texttt{lme4} wykorzystuje opisaną w części teoretycznej metodę oszacowania parametrów modelu mieszanego - algorytm używający macierzy rzadkich oraz dekompozycji Choleskiego \cite{lme4}. Pakiet \texttt{lmerTest} umożliwia obliczenie $p$-value dla parametrów modelu mieszanego \cite{lmerTest}. W przypadku modeli liniowych użyto funkcji \texttt{lm} z pakietu bazowego R. Funkcja ta wykorzystuje metodę najmniejszych kwadratów estymacji parametrów modelu liniowego \cite{lm}. W modelach, gdzie dokonano transformacji zmiennych, została użyta funkcja \texttt{powerTransform} z pakietu \texttt{car}. Funkcja ta wykorzystuje transformację Boxa-Coxa \cite{powerTransform}.

Dla uproszczenia zapisu, we wszystkich poniższych modelach nazwa $total\_cases$ oznacza zmienną $total\_cases\_per\_million$, a $total\_tests$ oznacza \\$total\_tests\_per\_thousand$.

\subsection{Zależność między liczbą zachorowań a czasem}

Na początek interesuje nas dynamika rozwoju pandemii, badamy więc hipotezę, że czas ma istotny wpływ na liczbę zachorowań w poszczególnych krajach.
Na rysunku \ref{fig:total_cases_countries} jest pokazana zależność pomiędzy liczbą zachorowań a czasem (gdzie czas rozumiemy jako kolejne dni trwania epidemii). Widzimy, że największa liczba zachorowań (na milion mieszkańców) pojawia się w Bahrajnie i Katarze. W krajach takich jak Polska, Niemcy, Włochy, Wielka Brytania, początkowo wzrost liczby zachorowań jest dość wolny, dopiero pomiędzy dwusetnym a trzysetnym dniem epidemii następuje bardziej gwałtowna zmiana. Jest to szczególnie widoczne na przykładzie Belgii, gdzie pod koniec badanego okresu są notowane bardzo wysokie liczby zachorowań. Dla Chin wykres jest dość płaski, prawdopodobnie z powodu bardzo dużej populacji - w przeliczeniu na milion mieszkańców, zachorowań w Chinach było  najmniej.

\begin{figure}[!ht]
	\centering
	\includegraphics[width=\linewidth]{rys/total_cases_countries.png}
	\caption{Wykres przedstawiający zależność liczby zachorowań na milion mieszkańców od czasu w podziale na kraje}
	\label{fig:total_cases_countries}
	\zrodlo{Opracowanie własne}
\end{figure}


\noindent Zaczynamy od budowy modelu z losowym wyrazem wolnym. Ma on postać
$$y_{total\_cases}=\beta_0 + X_{time}\beta_{time}+Z_{location}u_{location}+\varepsilon,$$
a więc przedstawia zależność liczby zachorowań od czasu, a kraj jest efektem losowym.
Podsumowanie tego modelu jest przedstawione w tabeli \ref{table:model1}.





\begin{table}[!h]
	\begin{center}
		\begin{tabular}{l c}
			\hline
			& Model 1.1 \\
			\hline
			(Intercept)               & $-1941.35^{***}$ \\
			& $(337.72)$       \\
			time                      & $37.31^{***}$    \\
			& $(0.26)$         \\
			\hline
			AIC                       & $806648.05$      \\
			BIC                       & $806682.57$      \\
			Log Likelihood            & $-403320.03$     \\
			Num. obs.                 & $41287$          \\
			Num. groups: location     & $151$            \\
			Var: location (Intercept) & $16967251.36$    \\
			Var: Residual             & $17530563.89$    \\
			\hline
			\multicolumn{2}{l}{\scriptsize{$^{***}p<0.001$; $^{**}p<0.01$; $^{*}p<0.05$}}
		\end{tabular}
		\caption{Wyniki dla modelu mieszanego z uwzględnieniem wpływu kraju na wyraz wolny}
		\label{table:model1}
		\zrodlo{Opracowanie własne}
	\end{center}
\end{table}



Widać, że efekt losowy jest odpowiedzialny za około połowę wariancji resztowej. Oznacza to, że zmienność liczby zachorowań dla danego kraju jest około dwukrotnie mniejsza niż zmienność liczby zachorowań dla różnych krajów.
Zarówno wyraz wolny, jak i współczynnik przy zmiennej $time$ są istotne statystycznie ($p$-value poniżej $0.001$). Dodatkowo, $\beta_{time}$ wynosi $37.31$, jest dodatni, więc upływający czas sprawia, że liczba zachorowań rośnie.



Na rysunku \ref{fig:mod1_predict} widzimy linie dopasowane do liczby zachorowań w poszczególnych krajach. Każda prosta ma taki sam współczynnik kierunkowy, jedynie punkt przecięcia z osią OY (\textit{Intercept}) różni się pomiędzy poszczególnymi krajami. Z tego wykresu możemy odczytać jedynie, jak różnią się średnie poziomy liczby zachorowań między krajami, widać na przykład, że w Katarze i Bahrajnie jest średnio najwięcej zachorowań, a w Chinach najmniej. Nie ma tu żadnej informacji na temat różnic w dynamice rozwoju pandemii.

\newpage

\begin{figure}[!h]
	\centering
	\includegraphics[width=\linewidth]{rys/mod1_predict.png}
	\caption{Wykres przedstawiający zależność między liczbą zachorowań a czasem oszacowaną za pomocą modelu mieszanego z losowym wyrazem wolnym}
	\label{fig:mod1_predict}
	\zrodlo{Opracowanie własne}
\end{figure}



\noindent Oprócz powyższego modelu, w którym tylko wyraz wolny różni się pomiędzy krajami, można rozważyć także model, gdzie współczynnik nachylenia prostej także będzie zależał od efektu losowego, czyli model postaci:
$$y_{total\_cases}=\beta_0+X_{time}\beta_{time}+Z_{0,location}u_{0,location}+$$$$+Z_{time,location}u_{time,location}+\varepsilon,$$
dla którego podsumowanie jest przedstawione w tabeli \ref{table:model1-1}.
\newpage
\begin{table}[!htbp]
	\begin{center}
		\begin{tabular}{l c}
			\hline
			& Model 1.2 \\
			\hline
			(Intercept)                    & $-1784.78^{***}$ \\
			& $(145.44)$       \\
			time                           & $35.49^{***}$    \\
			& $(2.93)$         \\
			\hline
			AIC                            & $759067.36$      \\
			BIC                            & $759119.13$      \\
			Log Likelihood                 & $-379527.68$     \\
			Num. obs.                      & $41287$          \\
			Num. groups: location          & $151$            \\
			Var: location (Intercept)      & $3113329.04$     \\
			Var: location time             & $1296.95$        \\
			Cov: location (Intercept) time & $-53442.24$      \\
			Var: Residual                  & $5450792.56$     \\
			\hline
			\multicolumn{2}{l}{\scriptsize{$^{***}p<0.001$; $^{**}p<0.01$; $^{*}p<0.05$}}
		\end{tabular}
		\caption{Wyniki dla modelu mieszanego z uwzględnieniem wpływu kraju na wyraz wolny i przesunięcie linii regresji }
		\label{table:model1-1}
		\zrodlo{Opracowanie własne}
	\end{center}
\end{table}



\begin{figure}[!h]
	\centering
	\includegraphics[width=\linewidth]{rys/mod1_slope_predict.png}
	\caption{Wykres przedstawiający zależność między liczbą zachorowań a czasem oszacowaną za pomocą modelu, gdzie zarówno wyraz wolny, jak i współczynnik nachylenia prostej różnią się pomiędzy krajami}
	\label{fig:mod1_slope_predict}
	\zrodlo{Opracowanie własne}
\end{figure}
 Wykres \ref{fig:mod1_slope_predict} pozwala na dostrzeżenie różnic w tempie rozwoju pandemii w poszczególnych krajach. Widać, że dla Kataru i Bahrajnu tempo rozwoju jest najszybsze, a dla Chin najwolniejsze. Jednakże, model ten nie uwzględnia fazy gwałtownego wzrostu, która występowała pod koniec badanego okresu (widać to było na przykładzie Belgii, Polski czy Włoch na rys. \ref{fig:total_cases_countries}). Dlatego do modelu spróbujemy włączyć kolejne potęgi zmiennej $time$.

\newpage

Zbudujemy model mieszany, gdzie zależność liczby zachorowań od czasu jest opisana funkcją kwadratową:
$$y_{total\_cases}=\beta_0+X_{time}\beta_{time}+X_{time^2}\beta_{time^2}+$$$$+Z_{0,location}u_{0,location}+Z_{time,location}u_{time,location}+$$$$Z_{time^2,location}u_{time^2, location}+\varepsilon$$
Wyniki dla tego modelu znajdują się w tabeli \ref{table:model1-2}.

\newpage

\begin{table}[!h]
	\begin{center}
		\begin{tabular}{l c}
			\hline
			& Model 1.3 \\
			\hline
			(Intercept)                           & $383.14^{***}$ \\
			& $(107.34)$     \\
			time                                  & $-10.66$       \\
			& $(161.66)$     \\
			time$^2$                              & $0.16$         \\
			& $(0.21)$       \\
			\hline
			AIC                                   & $721041.58$    \\
			BIC                                   & $721127.86$    \\
			Log Likelihood                        & $-360510.79$   \\
			Num. obs.                             & $41287$        \\
			Num. groups: location                 & $151$          \\
			Var: location (Intercept)             & $1671050.31$   \\
			Var: location time                    & $3946212.95$   \\
			Var: location time$^2$             & $6.52$         \\
			Cov: location (Intercept) time        & $24812.00$     \\
			Cov: location (Intercept) time$^2$ & $279.78$       \\
			Cov: location time time$^2$        & $5030.38$      \\
			Var: Residual                         & $2045067.48$   \\
			\hline
			\multicolumn{2}{l}{\scriptsize{$^{***}p<0.001$; $^{**}p<0.01$; $^{*}p<0.05$}}
		\end{tabular}
		\caption{Wyniki dla modelu wielomianowego mieszanego uwzględniającego wpływ kraju}
		\label{table:model1-2}
		\zrodlo{Opracowanie własne}
	\end{center}
\end{table}

Wizualnie model z drugą potęgą zmiennej $time$ wydaje się być lepiej dopasowany do danych (rys. \ref{fig:mod1_kw_predict}), jednakże w tym modelu ani pierwsza, ani druga potęga zmiennej niezależnej nie są istotne statystycznie. 

\newpage

\begin{figure}[!h]
	\centering
	\includegraphics[width=\linewidth]{rys/mod1_kw_predict.png}
	\caption{Wykres przedstawiający zależność między liczbą zachorowań a czasem oszacowaną za pomocą modelu, gdzie zarówno wyraz wolny, jak i współczynniki przy $time$ i $time^2$ zależą od kraju}
	\label{fig:mod1_kw_predict}
	\zrodlo{Opracowanie własne}
\end{figure}






Wracając do rysunku \ref{fig:total_cases_countries}, można podejrzewać, że do danych będzie dobrze pasował model mieszany, w którym efekt czasu jest opisany wielomianem trzeciego stopnia, czyli
$$y_{total\_cases}=\beta_0+X_{time}\beta_{time}+X_{time^2}\beta_{time^2}+X_{time^3}\beta_{time^3}+$$$$+Z_{0,location}u_{0,location}+Z_{time,location}u_{time,location}+$$$$+Z_{time^2,location}u_{time^2, location}+Z_{time^3,location}u_{time^3, location}+\varepsilon$$
Podsumowanie tego modelu można odczytać z tabeli \ref{table:mod1-poly}.

\newpage

\begin{table}[!h]
	\begin{center}
		\begin{tabular}{l c}
			\hline
			& Model 1.4 \\
			\hline
			(Intercept)                                 & $3101.45^{***}$   \\
			& $(320.37)$        \\
			time                              & $577597.25^{***}$ \\
			& $(54939.11)$      \\
			time$^2$                              & $202728.96^{***}$ \\
			& $(28196.37)$      \\
			time$^3$                              & $71609.14^{***}$  \\
			& $(20647.78)$      \\
			\hline
			AIC                                         & $675017.16$       \\
			BIC                                         & $675146.58$       \\
			Log Likelihood                              & $-337493.58$      \\
			Num. obs.                                   & $41287$           \\
			Num. groups: location                       & $151$             \\
			Var: location (Intercept)                   & $15495330.05$     \\
			Var: location time                & $455584118475.42$ \\
			Var: location time$^2$                & $119854000709.20$ \\
			Var: location time$^3$                & $64197657040.51$  \\
			Cov: location (Intercept) time    & $2548609171.19$   \\
			Cov: location (Intercept) time$^2$    & $329435355.67$    \\
			Cov: location (Intercept) time$^3$    & $-105891364.41$   \\
			Cov: location time time$^2$ & $113956385885.80$ \\
			Cov: location time time$^3$ & $10930922943.81$  \\
			Cov: location time$^2$ time$^3$ & $68483035447.11$  \\
			Var: Residual                               & $681134.37$       \\
			\hline
			\multicolumn{2}{l}{\scriptsize{$^{***}p<0.001$; $^{**}p<0.01$; $^{*}p<0.05$}}
		\end{tabular}
		\caption{Wyniki dla modelu mieszanego wielomianowego stopnia trzeciego}
		\label{table:mod1-poly}
	\end{center}
\end{table}

\newpage

Wszystkie trzy potęgi zmiennej $time$ są istotne statystycznie ($p$-value poniżej $0.001$), w dodatku przy każdej z nich współczynnik jest dodatni. Wykresy dopasowane na podstawie tego modelu są przedstawione na rysunku \ref{fig:mod1_poly_predict}. Ten model uwzględnia fazę wzrostu liczby zachorowań pod koniec obserwowanego okresu. Dla większości badanych krajów epidemię można z grubsza podzielić więc na trzy fazy: początkowy powolny wzrost, następnie stabilizacja, a jeszcze później - gwałtowny wzrost.

\begin{figure}[!h]
	\centering
	\includegraphics[width=\linewidth]{rys/mod1_poly_predict.png}
	
	\caption{Model mieszany, gdzie zależność liczby zachorowań od czasu jest opisana wielomianem trzeciego stopnia}
	\label{fig:mod1_poly_predict}
\end{figure}

\subsection{Zależność między liczbą zachorowań a liczbą wykonywanych testów na COVID-19}

Liczba zachorowań jest ustalana na podstawie liczby pozytywnych wyników testów na COVID-19. W niektórych krajach potwierdzić zachorowanie można także na podstawie innych badań, na przykład tomografii albo RTG płuc, jednakże najpopularniejszą metodą diagnozowania tej choroby są testy. Z tego powodu zbadamy hipotezę, że istnieje związek pomiędzy liczbą zachorowań a liczbą wykonywanych testów na COVID-19.

Na rysunku \ref{fig:tests} przedstawiona jest zależność liczby zachorowań od liczby wykonywanych testów. Na pierwszy rzut oka widać, że wraz ze wzrostem liczby testów rośnie liczba zachorowań. W niektórych krajach (np. Peru, Brazylia) ten wzrost jest bardzo gwałtowny, mimo stosunkowo niewielkiej liczby testów. W krajach takich jak Polska czy Włochy widać też fazy epidemii, o których była mowa przy okazji zależności liczby zachorowań od czasu. W Danii i Zjednoczonych Emiratach Arabskich wykonywane jest bardzo dużo testów, ale wzrost liczby zachorowań jest wolniejszy niż w większości krajów.


\begin{figure}[!h]
	\centering
	\includegraphics[width=\linewidth]{rys/total_tests_countries.png}
	\caption{Wykres przedstawiający zależność między liczbą zachorowań a liczbą wykonywanych testów w poszczególnych krajach}
	\label{fig:tests}
	\zrodlo{Opracowanie własne}
\end{figure}

\noindent Model mieszany ma postać:
$$y_{total\_ cases}=\beta_0 + X_{total\_tests}\beta_{total\_ tests}+Z_{location}u_{location}+\varepsilon$$
Badamy tutaj, czy liczba wykonywanych testów (w przeliczeniu na 1000 mieszkańców) ma wpływ na liczbę zachorowań.
Dla tego modelu otrzymujemy wyniki przedstawione w tabeli \ref{table:model2}.
\newpage
\begin{table}[!htbp]
	\begin{center}
		\begin{tabular}{l c}
			\hline
			& Model 2.1 \\
			\hline
			(Intercept)                 & $1741.18^{***}$ \\
			& $(464.57)$      \\
			total\_tests\_per\_thousand & $34.85^{***}$   \\
			& $(0.28)$        \\
			\hline
			AIC                         & $430649.03$     \\
			BIC                         & $430681.01$     \\
			Log Likelihood              & $-215320.52$    \\
			Num. obs.                   & $21907$         \\
			Num. groups: location       & $97$            \\
			Var: location (Intercept)   & $20699413.62$   \\
			Var: Residual               & $19714018.35$   \\
			\hline
			\multicolumn{2}{l}{\scriptsize{$^{***}p<0.001$; $^{**}p<0.01$; $^{*}p<0.05$}}
		\end{tabular}
		\caption{Wyniki dla modelu 2}
		\label{table:model2}
		\zrodlo{Opracowanie własne}
	\end{center}
\end{table}

Widać po pierwsze, że efekt losowy jest odpowiedzialny za ponad połowę zmienności resztowej modelu. Po drugie, widać, że efekt stały liczby wykonywanych testów jest istotny statystycznie ($p$-value poniżej $0.001$), i ma wpływ stymulujący na liczbę zachorowań (współczynnik $\beta_{total\_ tests}$ wynosi $34.85$). Dopasowanie tego modelu jest przedstawione na rysunku \ref{fig:mod2}. Wykres pokazuje, gdzie średnia liczba zachorowań jest największa w zależności od liczby wykonywanych testów, ale nie pozwala na porównanie tego, jak szybko wzrasta liczba zachorowań w krajach.



\begin{figure}[!ht]
	\centering
	\includegraphics[width=\linewidth]{rys/mod2_predict.png}
	\caption{Wykres przedstawiający dopasowanie modelu mieszanego do zależności pomiędzy liczbą zachorowań a liczbą wykonywanych testów}
	\label{fig:mod2}
	\zrodlo{Opracowanie własne}
\end{figure}

\newpage

\noindent Można także dopasować model, w którym współczynnik nachylenia prostej zależy od kraju:
$$y_{total\_ cases}=\beta_0 + X_{total\_tests}\beta_{total\_ tests}+$$
$$+Z_{0,location}u_{0,location}+Z_{total\_tests, location}u_{total\_tests, location}+\varepsilon$$
Dla takiego modelu otrzymujemy wyniki przedstawione w tabeli \ref{table:mod2-1}.
\newpage
\begin{table}[!h]
\begin{center}
	\begin{tabular}{l c}
		\hline
		& Model 2.2 \\
		\hline
		(Intercept)                                           & $-324.65$      \\
		& $(175.92)$     \\
		total\_tests\_per\_thousand                           & $109.17^{***}$ \\
		& $(13.83)$      \\
		\hline
		AIC                                                   & $391507.90$    \\
		BIC                                                   & $391555.87$    \\
		Log Likelihood                                        & $-195747.95$   \\
		Num. obs.                                             & $21907$        \\
		Num. groups: location                                 & $97$           \\
		Var: location (Intercept)                             & $2910697.88$   \\
		Var: location total\_tests\_per\_thousand             & $18059.27$     \\
		Cov: location (Intercept) total\_tests\_per\_thousand & $-6245.44$     \\
		Var: Residual                                         & $3206479.62$   \\
		\hline
		\multicolumn{2}{l}{\scriptsize{$^{***}p<0.001$; $^{**}p<0.01$; $^{*}p<0.05$}}
	\end{tabular}
	\caption{Wyniki dla modelu mieszanego z losowym wyrazem wolnym i współczynnikiem nachylenia}
	\label{table:mod2-1}
\end{center}
\end{table}

Proste regresji dopasowane z tego modelu są widoczne na wykresie \ref{fig:mod2-slope}. Oddają one szybkość rozwoju pandemii w zależności od liczby testów w poszczególnych krajach. Widać przykładowo, że proste regresji dla Danii i Zjednoczonych Emiratów Arabskich są nachylone pod stosunkowo małym kątem, z kolei dla Brazylii i Peru kąt nachylenia jest duży. Wykres ten nie oddaje dokładnie dynamiki rozwoju pandemii, nie uwzględnia tego, że w międzyczasie krzywe wypłaszczały się, aby potem gwałtowniej wzrosnąć. Ten model umożliwia jednak proste porównanie tego, jak szybko w różnych krajach wzrastała liczba zachorowań w zależności od liczby wykonywanych testów.
\newpage
\begin{figure}[!h]
	\centering
	\includegraphics[width=\linewidth]{rys/mod2_slope_predict.png}
	\caption{Wykres przedstawiający dopasowanie modelu typu \textit{Random Intercept and Slope} do zależności pomiędzy liczbą zachorowań a liczbą wykonywanych testów}
	\label{fig:mod2-slope}
	\zrodlo{Opracowanie własne}
\end{figure}




\subsection{Zależność między liczbą zachorowań a oczekiwaną długością życia}

Można się spodziewać, że kraje o wyższej oczekiwanej długości życia charakteryzują się wyższą liczbą zachorowań, gdyż osoby starsze są bardziej narażone na zachorowanie, podczas gdy młodsi albo nie chorują, albo przechodzą infekcję bezobjawowo. Zbadamy hipotezę o tym, że kraje o różnej oczekiwanej długości życia różnią się liczbą zachorowań.

Trzeci model jest modelem liniowym, ponieważ cecha $life\_expectancy$ nie zmienia się dla danego kraju. Z transformacji Boxa-Coxa otrzymujemy następujące potęgi dla zmiennych: $3.72$ dla $life\_expectancy$ oraz $0.09$ dla $total\_cases\_per\_million$. W przybliżeniu przyjmiemy czwartą potęgę dla pierwszej zmiennej, a dla drugiej logarytm.
$$log(y_{total\_cases})=\beta_0+\beta_{life\_expectancy}X_{life\_expectancy}^4+\varepsilon$$
Podsumowanie tego modelu jest przedstawione w tabeli \ref{table:mod3-potega}.

\newpage

\begin{table}[!htbp]
	\begin{center}
		\begin{tabular}{l c}
		\hline
		& Model 3 \\
		\hline
		(Intercept)          & $3.22^{***}$ \\
		& $(0.04)$     \\
		life\_expectancy$^4$ & $8.54\cdot10^{-8~  ***}$ \\
		& $(0.00)$     \\
		\hline
		R$^2$                & $0.10$       \\
		Adj. R$^2$           & $0.10$       \\
		Num. obs.            & $41287$      \\
		\hline
		\multicolumn{2}{l}{\scriptsize{$^{***}p<0.001$; $^{**}p<0.01$; $^{*}p<0.05$}}
	\end{tabular}
		\caption{Wyniki dla modelu trzeciego po przekształceniu zmiennych}
		\label{table:mod3-potega}
	\end{center}
\end{table}

Efekt $life\_expectancy$ jest istotny statystycznie ($p$-value poniżej $0.001$). Współczynnik $\beta_{life\_expectancy}$ wynosi około $8.54 \cdot 10^{-8}$. Jest on dodatni, co oznacza, że ze wzrostem oczekiwanej długości życia rośnie liczba zachorowań na COVID-19. Niska wartość tego współczynnika może wynikać z tego, że zmienna $life\_expectancy$ jest podniesiona do potęgi $4$, więc obserwacje mają duże wartości. Współczynnik $R^2$ wynosi około $0.1$, więc jest niski. Dopasowanie modelu liniowego do danych po przekształceniu jest przedstawione na rysunku \ref{fig:mod3-potega}.

\begin{figure}[!ht]
	\centering
	\includegraphics[width=\linewidth]{rys/mod3-potega.png}
	\caption{Wykres przedstawiający dopasowanie modelu liniowego z przekształceniem zmiennych do zależności pomiędzy liczbą zachorowań a oczekiwaną długością życia}
	\label{fig:mod3-potega}
	\zrodlo{Opracowanie własne}
\end{figure}







\subsection{Zależność między liczbą zachorowań a gęstością zaludnienia}

Koronawirus rozprzestrzenia się drogą kropelkową, rozwojowi pandemii sprzyjają więc miejsca, gdzie dużo osób ma ze sobą kontakt osobisty. Dlatego rozważymy hipotezę, że gęstość zaludnienia ma wpływ na liczbę zachorowań.

Czwarty model również jest modelem regresji prostej. Stosujemy przekształcenie logarytmiczne obu zmiennych.
$$\log(y_{total\_cases})=\beta_0+log(X_{population\_density})\beta_{population\_density}+\varepsilon$$
 Wykres rozrzutu obu zmiennych po przekształceniu znajduje się na rysunku \ref{fig:mod4-lin}.
Wyniki dla modelu są przedstawione w tabeli \ref{table:mod4}.



\begin{table}[!htbp]
	\begin{center}
		\begin{tabular}{l c}
			\hline
			& Model 4 \\
			\hline
			(Intercept)              & $5.05^{***}$ \\
			& $(0.46)$     \\
			log(population\_density) & $0.13$       \\
			& $(0.10)$     \\
			\hline
			R$^2$                    & $0.00$       \\
			Adj. R$^2$               & $0.00$       \\
			Num. obs.                & $537$        \\
			\hline
			\multicolumn{2}{l}{\scriptsize{$^{***}p<0.001$; $^{**}p<0.01$; $^{*}p<0.05$}}
		\end{tabular}
		\caption{Wyniki dla modelu 4 z przekształceniem zmiennych}
		\label{table:mod4}
	\end{center}
\end{table}

Gęstość zaludnienia nie jest czynnikiem istotnym statystycznie ($p$-value powyżej $0.05$), a $R^2$ jest bliskie 0. Nie można mówić o istotnym związku pomiędzy gęstością zaludnienia a liczbą zachorowań.

\newpage

\begin{figure}[!h]
	\centering
	\includegraphics[width=\linewidth]{rys/mod4-log.png}
	\caption{Wykres przedstawiający dopasowanie modelu liniowego do zależności pomiędzy liczbą zachorowań a gęstością zaludnienia po przekształceniu logarytmicznym }
	\label{fig:mod4-lin}
	\zrodlo{Opracowanie własne}
\end{figure}








\subsection{Zależność między liczbą zachorowań a siłą obostrzeń}

Aby spowolnić wzrost liczby zachorowań, kraje wprowadzają różne polityki obostrzeń takich jak ograniczenia w przemieszczaniu się, zamknięcie szkół itp. Sprawdzimy więc prawdziwość hipotezy, że siła obostrzeń ma wpływ na liczbę zachorowań.

Siła obostrzeń jest mierzona za pomocą zmiennej $stringency\_index$. Zarówno wartość średnia, jak i zmienność $stringency\_index$ bardzo się różnią pomiędzy krajami. Na rysunku \ref{fig:stringency_time} przedstawiono przebieg zmienności $stringency\_index$ w czasie. Jak widać, są kraje jak np. Nikaragua, gdzie siła obostrzeń jest na niskim poziomie i ma małą zmienność. Mamy też kraje takie jak Chile lub Peru, w których obostrzenia wzrastały silnie na początku pandemii, a potem zmieniały się już mniej znacznie. W Norwegii obostrzenia początkowo były wysokie, a potem znacząco spadły. W Polsce początkowo poziom siły obostrzeń zmieniał się podobnie jak w Norwegii, ale pod koniec badanego przedziału czasowego znacząco wzrósł.
 
\newpage

\begin{figure}[!ht]
	\centering
	\includegraphics[width=\linewidth]{rys/stringency_time.png}
	\caption{Zmiany siły obostrzeń w poszczególnych krajach}
	\label{fig:stringency_time}
	\zrodlo{Opracowanie własne}
\end{figure}



Przed budową modelu, zostały przeanalizowane korelacje pomiędzy liczbą zachorowań, a zmienną $stringency\_index$ bez opóźnienia oraz z opóźnieniem kolejno 7, 14, 21, 28, 35 i 42 dni. Najwyższa korelacja wystąpiła przy opóźnieniu o 7 dni, więc tak opóźnionej zmiennej użyjemy do budowy modelu mieszanego. Rysunek \ref{fig:stringency_lag} jest wykresem zależności liczby zachorowań od siły obostrzeń opóźnionej o 7 dni.
\newpage
\begin{figure}[!h]
	\centering
	\includegraphics[width=\linewidth]{rys/stringency_lag.png}
	\caption{Zależność liczby zachorowań od siły obostrzeń opóźnionej o 7 dni}
	\label{fig:stringency_lag}
	\zrodlo{Opracowanie własne}
\end{figure}

Na początku zbudujemy model postaci
$$y_{total\_cases}=\beta_0+\beta_{stringency\_lag7} X_{stringency\_lag7}+Z_{location}u_{location}+\varepsilon$$
Wyniki dla tego modelu są przedstawione w tabeli \ref{table:mod5-intercept}. Zmienna $stringency\_lag7$ jest istotna statystycznie, choć jest na granicy istotności, gdyż $p$-value wynosi około $0.0464$. Współczynnik $\beta_{stringency\_lag7}$ ma wartość $-2.89$, jest ujemny, co oznacza, że im wyższą wartość przyjmuje $stringency\_lag7$, tym mniejsza liczba zachorowań.

\newpage
\begin{table}[!h]
	\begin{center}
		\begin{tabular}{l c}
			\hline
			& Model 5.1 \\
			\hline
			(Intercept)               & $3377.31^{***}$ \\
			& $(364.95)$      \\
			stringency\_lag7          & $-2.89^{*}$     \\
			& $(1.45)$        \\
			\hline
			AIC                       & $780586.43$     \\
			BIC                       & $780620.73$     \\
			Log Likelihood            & $-390289.21$    \\
			Num. obs.                 & $39190$         \\
			Num. groups: location     & $147$           \\
			Var: location (Intercept) & $18340804.07$   \\
			Var: Residual             & $25666097.17$   \\
			\hline
			\multicolumn{2}{l}{\scriptsize{$^{***}p<0.001$; $^{**}p<0.01$; $^{*}p<0.05$}}
		\end{tabular}
		\caption{Wyniki dla modelu mieszanego z losowym wyrazem wolnym}
		\label{table:mod5-intercept}
	\end{center}
\end{table}

\noindent Spróbujemy także zbudować model, gdzie nachylenie prostej regresji również będzie zależało od kraju
$$y_{total\_cases}=\beta_0+\beta_{stringency\_lag7} X_{stringency\_lag7}+Z_{location}u_{location}+$$$$+Z_{stringency\_lag7,location}u_{stringency\_lag7,location}+\varepsilon$$
Podsumowanie tego modelu można odczytać z tabeli \ref{table:mod5-slope}.
\newpage
\begin{table}[!htpb]
	\begin{center}
		\begin{tabular}{l c}
			\hline
			& Model 5.2 \\
			\hline
			(Intercept)                                & $4570.86^{***}$ \\
			& $(403.65)$      \\
			stringency\_lag7                           & $-14.86^{*}$    \\
			& $(6.13)$        \\
			\hline
			AIC                                        & $777617.12$     \\
			BIC                                        & $777668.58$     \\
			Log Likelihood                             & $-388802.56$    \\
			Num. obs.                                  & $39190$         \\
			Num. groups: location                      & $147$           \\
			Var: location (Intercept)                  & $21876819.86$   \\
			Var: location stringency\_lag7             & $5020.82$       \\
			Cov: location (Intercept) stringency\_lag7 & $-200474.29$    \\
			Var: Residual                              & $23567356.48$   \\
			\hline
			\multicolumn{2}{l}{\scriptsize{$^{***}p<0.001$; $^{**}p<0.01$; $^{*}p<0.05$}}
		\end{tabular}
		\caption{Wyniki dla modelu mieszanego, w którym wyraz wolny oraz $\beta_{stringency\_lag7}$ zależą od kraju}
		\label{table:mod5-slope}
	\end{center}
\end{table}


Współczynnik przy $stringency\_lag7$ jest istotny statystycznie ($p$-value około $0.0162$) i wynosi $-14.86$, co oznacza, że w przeciętnym kraju siła obostrzeń wpływa na zmniejszenie liczby zachorowań po 7 dniach. Proste regresji dopasowane za pomocą tego modelu widać na rysunku \ref{fig:stringency_1st}. Z tego wykresu widać, że w niektórych krajach (np. Bahrain i Peru) zależność faktycznie jest malejąca. Z kolei np. w Norwegii zależność jest prawie nieistniejąca. Mamy też kraje takie jak Polska lub Chile, gdzie silniejszym obostrzeniom towarzyszy wyższa liczba zachorowań.
\newpage
\begin{figure}[!h]
	\centering
	\includegraphics[width=\linewidth]{rys/stringency_1st.png}
	\caption{Proste regresji dopasowane do zależności między opóźnioną o 7 dni siłą obostrzeń a liczbą zachorowań }
	\label{fig:stringency_1st}
	\zrodlo{Opracowanie własne}
\end{figure}

Ponieważ z wykresu \ref{fig:stringency_lag} widać, że zależność między opóźnioną siłą obostrzeń a liczbą zachorowań raczej nie jest liniowa, to dopasujemy model mieszany z drugą potęgą zmiennej $stringency\_lag7$
$$y_{total\_cases}=\beta_0+\beta_{stringency\_lag7} X_{stringency\_lag7}+ \beta_{stringency\_lag7^2} X_{stringency\_lag7^2}+$$$$+Z_{location}u_{location}+Z_{stringency\_lag7,time}u_{stringency\_lag7,time}+$$$$+Z_{stringency\_lag7^2, location}u+_{stringency\_lag7^2, location}+\varepsilon$$
Podsumowanie tego modelu widać w tabeli \ref{table:mod5-2st}. Współczynnik przy drugiej potędze $stringency\_lag7$ jest istotny statystycznie ($p$-value poniżej $0.001$) i wynosi $-402699.65$. Efekt pierwszej potęgi $stringency\_lag7$ nie jest istotny, ale nie możemy go usunąć z modelu, ponieważ jest istotny efekt wyższego rzędu.


\begin{table}
	\begin{center}
		\begin{tabular}{l c}
			\hline
			& Model 5.3 \\
			\hline
			(Intercept)                                                         & $3163.02^{***}$    \\
			& $(378.38)$         \\
			stringency\_lag7                                          & $-21460.51$        \\
			& $(37503.93)$       \\
			(stringency\_lag7)$^2$                                          & $-402699.65^{***}$ \\
			& $(77604.12)$       \\
			\hline
			AIC                                                                 & $763894.16$        \\
			BIC                                                                 & $763979.92$        \\
			Log Likelihood                                                      & $-381937.08$       \\
			Num. obs.                                                           & $39190$            \\
			Num. groups: location                                               & $147$              \\
			Var: location (Intercept)                                           & $20524092.27$      \\
			Var: location stringency\_lag7                            & $188332308392.00$  \\
			Var: location (stringency\_lag7)$^2$                            & $869155957279.70$  \\
			Cov: location (Intercept) stringency\_lag7                & $168631325.28$     \\
			Cov: location (Intercept) (stringency\_lag7)$^2$                & $-1952065450.06$   \\
			Cov: location stringency\_lag7 (stringency\_lag7)$^2$ & $125443897377.66$  \\
			Var: Residual                                                       & $16253620.55$      \\
			\hline
			\multicolumn{2}{l}{\scriptsize{$^{***}p<0.001$; $^{**}p<0.01$; $^{*}p<0.05$}}
		\end{tabular}
		\caption{Wyniki dla modelu mieszanego z drugą potęgą zmiennej $stringensy\_lag7$}
		\label{table:mod5-2st}
	\end{center}
\end{table}
\newpage
Wykres zależności liczby zachorowań od $stringency\_lag7$ i drugiej potęgi tej zmiennej widać na rysunku \ref{fig:stringency_2st}. W większości krajów współczynnik przy drugiej potędze $stringency\_lag7$ rzeczywiście jest ujemny, co oznacza, że po przekroczeniu pewnego punktu (a dokładniej wierzchołka paraboli) liczba zachorowań zaczyna spadać. Wyjątkiem jest tu np. Chile, gdzie zależność jest odwrotna.

\newpage

\begin{figure}[!h]
	\centering
	\includegraphics[width=\linewidth]{rys/stringency_2st.png}
	\caption{Krzywe dopasowane do zależności między pierwszą i drugą potęgą opóźnionej o 7 dni siły obostrzeń a liczbą zachorowań }
	\label{fig:stringency_2st}
	\zrodlo{Opracowanie własne}
\end{figure}


\subsection{Zależność między liczbą zachorowań a wskaźnikiem rozwoju społecznego}

Wskaźnik rozwoju społecznego mówi o tym, jak rozwinięty jest kraj pod względem m. in. gospodarki i edukacji, bierze się w nim pod uwagę także oczekiwaną długość życia. Zbadamy hipotezę mówiącą, że kraje o różnej wysokości wskaźnika rozwoju społecznego (HDI) różnią się liczbą zachorowań.

Ponieważ wartość HDI nie zmienia się dla kraju w trakcie trwania epidemii, to szósty model jest modelem regresji prostej.
Z przekształcenia Boxa-Coxa otrzymujemy potęgę $2$ dla zmiennej $HDI$ oraz logarytm dla $total\_cases\_per\_million$.
$$\log(y_{total\_cases})=\beta_0+\beta_{HDI}X^2_{HDI}+\varepsilon$$
Podsumowanie modelu wygląda jak w tabeli \ref{table:mod6-log}.
\newpage
\begin{table}[!htbp]
	\begin{center}
		\begin{tabular}{l c}
			\hline
			& Model 6 \\
			\hline
			(Intercept)                   & $2.60^{***}$ \\
			& $(0.33)$     \\
			human\_development\_index$^2$ & $5.57^{***}$ \\
			& $(0.57)$     \\
			\hline
			R$^2$                         & $0.15$       \\
			Adj. R$^2$                    & $0.15$       \\
			Num. obs.                     & $534$        \\
			\hline
			\multicolumn{2}{l}{\scriptsize{$^{***}p<0.001$; $^{**}p<0.01$; $^{*}p<0.05$}}
		\end{tabular}
		\caption{Wyniki dla modelu 6 po przekształceniu zmiennych}
		\label{table:mod6-log}
	\end{center}
\end{table}

Współczynnik $\beta_{HDI}$ wynosi około $5.57$ i jest istotny statystycznie ($p$-value poniżej $0.001$). Wartość tego współczynnika jest dodatnia, co oznacza, że im wyższy wskaźnik rozwoju społecznego w danym kraju, tym więcej potwierdzonych przypadków COVID-19. Współczynnik determinacji wynosi około $15\%$, więc jest dość niski. Dopasowanie modelu do danych jest ukazane na rysunku \ref{fig:mod6-log}.

\begin{figure}[!ht]
	\centering
	\includegraphics[width=\linewidth]{rys/mod6-log.png}
	\caption{Wykres przedstawiający dopasowanie modelu liniowego do zależności pomiędzy liczbą zachorowań a wskaźnikiem rozwoju społecznego po przekształceniu zmiennych}
	\label{fig:mod6-log}
	\zrodlo{Opracowanie własne}
\end{figure}




\subsection{Zależność między liczbą zachorowań a powszechnością cukrzycy}

Wśród czynników osłabiających odporność organizmu często jest wymieniana cukrzyca. Dlatego zbadamy hipotezę, że rozpowszechnienie cukrzycy wpływa na liczbę zachorowań na COVID-19.

Model ten jest modelem regresji prostej. Na podstawie transformacji Boxa-Coxa otrzymujemy pierwiastek trzeciego stopnia ze zmiennej $diabetes\_prevalence$ oraz logarytm z $total\_cases\_per\_million$.
$$\log(y_{total\_cases})=\beta_0+\beta_{diabetes\_prevalence}\sqrt[3]{X_{diabetes\_prevalence}}+\varepsilon$$
Wyniki znajdują się w tabeli \ref{table:mod8-log}.

\begin{table}[htbp]
	\begin{center}
		\begin{tabular}{l c}
			\hline
			& Model 7 \\
			\hline
			(Intercept)                & $2.49^{**}$  \\
			& $(0.79)$     \\
			diabetes\_prevalence\^~(1/3) & $1.65^{***}$ \\
			& $(0.41)$     \\
			\hline
			R$^2$                      & $0.03$       \\
			Adj. R$^2$                 & $0.03$       \\
			Num. obs.                  & $537$        \\
			\hline
			\multicolumn{2}{l}{\scriptsize{$^{***}p<0.001$; $^{**}p<0.01$; $^{*}p<0.05$}}
		\end{tabular}
		\caption{Wyniki dla modelu 8 po przekształceniu zmiennych}
		\label{table:mod8-log}
	\end{center}
\end{table}

Wartość współczynnika $\beta_{diabetes\_prevalence}$ jest równa $1.65$. Efekt ten jest istotny statystycznie ($p$-value poniżej $0.001$). Ponieważ wartość współczynnika jest dodatnia, to wraz ze wzrostem odsetka osób chorujących na cukrzycę, rośnie liczba zachorowań. Dopasowanie modelu liniowego do danych po przekształceniu jest zaprezentowane na rysunku \ref{fig:mod8-log}.

\newpage

\begin{figure}[!ht]
	\centering
	\includegraphics[width=\linewidth]{rys/mod8-log.png}
	\caption{Wykres przedstawiający dopasowanie modelu liniowego z przekształceniem zmiennych do zależności pomiędzy odsetkiem osób chorych na cukrzycę a liczbą zachorowań na COVID-19}
	\label{fig:mod8-log}
	\zrodlo{Opracowanie własne}
\end{figure}



\subsection{Zależność między liczbą zachorowań a odsetkiem osób żyjących w skrajnej biedzie}

Kraje, w których więcej osób żyje w ubóstwie na ogół charakteryzują się słabszym dostępem do służby zdrowia oraz niższym poziomem opieki medycznej. Zajmiemy się więc hipotezą mówiącą, że kraje o różnej liczbie osób żyjących w skrajnej biedzie różnią się liczbą zachorowań.



Ten model jest modelem regresji prostej. Stosując transformację Boxa-Coxa, otrzymujemy przekształcenie logarytmiczne dla obu zmiennych w modelu:
$$\log(y_{total\_cases})=\beta_0+\beta_{extreme\_poverty}\log(X_{extreme\_poverty})+\varepsilon$$
Podsumowanie dla modelu po przekształceniu zmiennych znajduje się w tabeli \ref{table:mod9-log}. Efekt $\log(extreme\_poverty)$ jest istotny statystycznie i ma wartość $-0.46$, co oznacza, że wraz ze wzrostem zmiennej niezależnej maleje liczba zachorowań. $R^2$ wzrosło do $9\%$, a więc nieznacznie w stosunku do modelu przed przekształceniem zmiennych. Dopasowanie modelu liniowego do danych po przekształceniu widać na rysunku \ref{fig:mod9-log}.

\newpage

\begin{table}[!htbp]
	\begin{center}
		\begin{tabular}{l c}
			\hline
			& Model 8 \\
			\hline
			(Intercept)           & $6.00^{***}$  \\
			& $(0.16)$      \\
			log(extreme\_poverty) & $-0.46^{***}$ \\
			& $(0.07)$      \\
			\hline
			R$^2$                 & $0.09$        \\
			Adj. R$^2$            & $0.09$        \\
			Num. obs.             & $389$         \\
			\hline
			\multicolumn{2}{l}{\scriptsize{$^{***}p<0.001$; $^{**}p<0.01$; $^{*}p<0.05$}}
		\end{tabular}
		\caption{Wyniki dla modelu 8 z przekształceniem logarytmicznym obu zmiennych}
		\label{table:mod9-log}
	\end{center}
\end{table}

Czynnik $extreme\_poverty$ jest istotny statystycznie ($p$-value poniżej 0.001). \\Współczynnik $\beta_{extreme\_poverty}$ wynosi $-0.46$. Jest on ujemny, więc im większy jest w danym kraju odsetek osób żyjących w biedzie, tym niższa liczba zachorowań. Prawdopodobnie jest to spowodowane mniejszą dostępnością do służby zdrowia w biedniejszych krajach i mniejszą liczbą wykonywanych testów. $R^2$ wynosi około $9\%$, co jest niską wartością. Na rysunku \ref{fig:mod9-log} przedstawiona jest zależność między przekształconymi logarytmicznie liczbą zachorowań a odsetkiem osób żyjących w skrajnej biedzie, z dopasowanym modelem liniowym. 

\newpage

\begin{figure}[!ht]
	\centering
	\includegraphics[width=\linewidth]{rys/mod9-log.png}
	\caption{Wykres przedstawiający dopasowanie modelu liniowego z przekształceniem zmiennych dla zależności pomiędzy liczbą zachorowań a odsetkiem osób żyjącym w skrajnej biedzie}
	\label{fig:mod9-log}
	\zrodlo{Opracowanie własne}
\end{figure}



\subsection{Zależność między liczbą zachorowań a wysokością PKB na osobę}

Wysokie PKB na osobę wskazuje na kraje dobrze rozwinięte gospodarczo, spodziewamy się w nich dobrego dostępu do opieki medycznej. Sprawdzimy hipotezę, że kraje o różnej wysokości PKB na osobę różnią się liczbą zachorowań.


Model dziewiąty jest modelem regresji prostej. Na podstawie przekształcenia Boxa-Coxa otrzymujemy pierwiastek piątego stopnia z $gdp\_per\_capita$ oraz logarytm z $total\_cases\_per\_million$.
$$\log(y_{total\_cases})=\beta_0+\beta_{GDP}\sqrt[5]{X_{GDP}}+\varepsilon$$
W tabeli \ref{table:mod10-log} widać podsumowanie modelu 10 po przekształceniu zmiennych.

\newpage

\begin{table}[!htbp]
	\begin{center}
		\begin{tabular}{l c}
			\hline
			& Model 9 \\
			\hline
			(Intercept)          & $0.42$       \\
			& $(0.55)$     \\
			gdp\_per\_capita\^~0.2 & $0.79^{***}$ \\
			& $(0.08)$     \\
			\hline
			R$^2$                & $0.15$       \\
			Adj. R$^2$           & $0.15$       \\
			Num. obs.            & $531$        \\
			\hline
			\multicolumn{2}{l}{\scriptsize{$^{***}p<0.001$; $^{**}p<0.01$; $^{*}p<0.05$}}
		\end{tabular}
		\caption{Wyniki dla modelu 9 z przekształceniem zmiennych}
		\label{table:mod10-log}
	\end{center}
\end{table}

Efekt PKB na osobę jest istotny statystycznie ($p$-value poniżej 0.001). Współczynnik $\beta_{GDP}$ wynosi 0.79. Jest on dodatni, więc im wyższe PKB danego kraju, tym wyższa liczba zachorowań. Współczynnik $R^2$ wynosi 0.15, co oznacza, że zmienna $gdp\_per\_capita$ wyjaśnia około $15\%$ zmienności modelu. Procent wyjaśnionej zmienności jest więc niski. Na rysunku \ref{fig:mod10-log} widać wykres rozrzutu zmiennej \\$total\_cases\_per\_million$ w zależności od $gdp\_per\_capita$ po przekształceniu, z dopasowanym modelem liniowym.

\begin{figure}[!ht]
	\centering
	\includegraphics[width=\linewidth]{rys/mod10-log.png}
	\caption{Wykres przedstawiający dopasowanie modelu liniowego z przekształceniem zmiennych do zależności pomiędzy liczbą zachorowań a PKB na osobę}
	\label{fig:mod10-log}
	\zrodlo{Opracowanie własne}
\end{figure}

\chapter*{Dyskusja wyników i wnioski}


W tabeli \ref{tab:istotnosc} znajduje się podsumowanie istotności poszczególnych czynników, których wpływ na liczbę zachorowań był badany w tej pracy.


\begin{longtable}{| p{.30\textwidth} | p{.60\textwidth} |}
	\hline
	Cecha & Wpływ na liczbę zachorowań \\ \hline \hline 
	Czas & istotny, można do tej zależności dopasować wielomian trzeciego stopnia \\ \hline
	Liczba wykonywanych testów na COVID-19 & istotny, wraz ze wzrostem liczby testów rośnie liczba zachorowań \\ \hline
	Oczekiwana długość życia & istotny, ze wzrostem oczekiwanej długości życia rośnie liczba zachorowań \\ \hline 
	Gęstość zaludnienia & nieistotny \\ \hline
	Wskaźnik siły obostrzeń & istotny jest wskaźnik opóźniony o 7 dni, można do tej zależności dopasować model kwadratowy, w większości krajów przy wysokiej sile obostrzeń jest mniej zachorowań \\ \hline
	Wskaźnik rozwoju społecznego & istotny, w krajach o wysokim wskaźniku rozwoju jest więcej zachorowań \\ \hline
	Powszechność występowania cukrzycy & istotny, im wyższy jest ten współczynnik, tym więcej przypadków koronawirusa \\ \hline
	Część populacji żyjąca w skrajnym ubóstwie & istotny, im większa jest część mieszkańców żyjąca w biedzie, tym mniej zachorowań \\ \hline
	PKB na osobę & istotny, im wyższe PKB, tym więcej zachorowań \\ \hline
	\caption{Porównanie istotności i wpływu różnych czynników na liczbę zachorowań w przeciętnym kraju}
	
	\label{tab:istotnosc}
	
	\end{longtable}
\begin{center}\zrodlo{Opracowanie własne}
	\end{center}


W niniejszej pracy wpływ różnych czynników na liczbę zachorowań na COVID-19 w około 150 krajach został zbadany przy pomocy siedmiu modeli liniowych oraz trzech mieszanych. Z modeli liniowych otrzymaliśmy cztery czynniki, których wzrost powoduje wyższą liczbę zachorowań. Są to: oczekiwana długość życia, wskaźnik rozwoju społecznego, powszechność występowania cukrzycy oraz PKB na osobę. Takie wnioski wydają się dość oczywiste. Wiele źródeł mówi, że na COVID-19 są szczególnie narażone osoby starsze oraz osoby z obniżoną odpornością lub chorobami współistniejącymi \cite{narazenie}. Szczególnie wiek, nadwaga i cukrzyca są wymieniane jako czynniki zwiększające prawdopodobieństwo zakażenia i ciężkiego przebiegu choroby \cite{covid-19}. Z kolei wysokie PKB oraz wskaźnik rozwoju społecznego sugerują, że w danym kraju opieka medyczna jest dobrze rozwinięta, co skutkuje większą liczbą wykonywanych testów, pozwalając wykryć więcej przypadków wirusa. Dodatkowo, w krajach wysoko rozwiniętych większą popularnością cieszą się podróże międzynarodowe (zarówno prywatne, jak i służbowe), co sprzyja rozprzestrzenianiu się pandemii.

Ograniczająco na liczbę wykrytych przypadków koronawirusa wpływa odsetek populacji żyjący w skrajnym ubóstwie. Jednakże najprawdopodobniej ubóstwo nie wpływa bezpośrednio na zmniejszenie liczby chorych. Można przypuszczać, że mniejsza liczba zachorowań jest raczej spowodowana słabiej rozwiniętą opieką medyczną i niższą liczbą wykonywanych testów, więc znaczna część osób chorych nie jest diagnozowana.

Czynnikiem niemającym wpływu na liczbę przypadków koronawirusa okazała się gęstość zaludnienia. Jest to zaskakujący wniosek, ponieważ zdawałoby się, że w krajach o większej gęstości zaludnienia wirus może łatwiej się rozprzestrzeniać, więc zachorowań powinno być więcej. Brak zależności może być spowodowany tym, że gęstość zaludnienia jest podawana dla całego kraju, a przecież może bardzo się różnić wewnątrz terytorium danego państwa.

Za pomocą modeli mieszanych w tej pracy zbadano wpływ czasu, liczby wykonywanych testów oraz wskaźnika siły obostrzeń na liczbę zachorowań. Zależność liczby zachorowań od czasu została opisana wielomianem trzeciego stopnia, którego współczynniki różnią się pomiędzy krajami. Wpływ liczby testów na liczbę wykrytych przypadków koronawirusa jest stymulujący, czego można było się spodziewać, ponieważ liczba wykrytych przypadków to nic innego jak liczba testów z pozytywnym wynikiem (w niektórych krajach można także potwierdzić przypadek COVID-19 na podstawie samych objawów). Model badający zależność liczby zachorowań od wskaźnika siły obostrzeń dawał niejednoznaczne wyniki, generalnie silniejsze obostrzenia sprawiały, że liczba zachorowań w danym kraju malała (po około tygodniu od wprowadzenia silniejszych obostrzeń), ale w niektórych państwach zależność ta nie zachodziła.

Wszystkie modele mieszane jednoznacznie pokazują, że efekt kraju jako czynnika zakłócającego jest bardzo istotny, w wielu przypadkach wyjaśnia ponad połowę wariancji resztowej modelu, więc zmienność liczby zachorowań pomiędzy różnymi krajami jest około dwukrotnie większa niż zmienność liczby zachorowań w pojedynczym kraju.

Na to, co jest nazywane w tej pracy ,,efektem kraju'', składa się tak naprawdę wiele innych czynników, m. in. gęstość zaludnienia, sytuacja ekonomiczna danego kraju, odsetek osób z chorobami towarzyszącymi, rozkład wieku, jak również przyjęta strategia walki z koronawirusem, na którą z kolei składają się m. in. liczba wykonywanych testów, przepisy w sprawie zamykania szkół, miejsc publicznych, ograniczenie kontaktów międzyludzkich, i wiele innych.



\begin{thebibliography}{99}

\bibitem{stringency} Blavatnik School of Government, University of Oxford, \emph{Coronavirus Government Response Tracker} \url{https://www.bsg.ox.ac.uk/research/research-projects/coronavirus-government-response-tracker} (dostęp 31.10.2020)

\bibitem{biecek} Biecek P., \emph{Analiza danych z programem R. Modele liniowe z efektami stałymi, losowymi i mieszanymi}, Wydawnictwo Naukowe PWN, Wydanie II, Warszawa 2013

\bibitem{lme4} CRAN, \emph{lme4: Linear Mixed-Effects Models using 'Eigen' and S4} \url{https://cran.r-project.org/web/packages/lme4/index.html} (dostęp: 03.01.2021)

\bibitem{lmerTest} CRAN, \emph{lmerTest: Tests in Linear Mixed Effects Models} \url{https://cran.r-project.org/web/packages/lmerTest/index.html} (dostęp: 03.01.2021)

\bibitem{covid-19}  Duszyński J., Afelt A.,  Ochab-Marcinek A.,
Owczuk R.,  Pyrć K.,  Rosińska M.,
Rychard A.,  Smiatacz T., \emph{Zrozumieć COVID-19. Opracowanie zespołu ds. COVID-19 przy Prezesie Polskiej Akademii Nauk}, Polska Akademia Nauk, 14 września 2020 r.

\bibitem{faraway} Faraway J., \emph{Extending the Linear Model with R. Generalized Linear, Mixed Effects and Nonparametric Regression Models. Second Edition}, CRC Press Taylor \& Francis  Group, 2016

\bibitem{brief}  Harrison XA, Donaldson L, Correa-Cano ME, Evans J, Fisher DN, Goodwin CED, Robinson BS, Hodgson DJ, Inger R., \emph{A brief introduction to mixed effects modelling and multi-model inference in ecology}, 2018, PeerJ 6:e4794 \url{https://doi.org/10.7717/peerj.4794} (dostęp: 11.11.2020)

\bibitem{bootstrap} Hesterberg T.,  Monaghan S., Moore D., Clipson A., Epstein R., \emph{Bootstrap Methods and Permutation Tests. Companion Chapter 18 to the Practice of Business Statistics}, W. H. Freeman and Company, New York, 2003


\bibitem{forecasting} Hyndman R.J.,  Athanasopoulos G. \emph{Forecasting: principles and practice, 3rd edition}, OTexts: Melbourne, Australia, 2019, \url{OTexts.com/fpp3} (dostęp: 07.01.2021)

\bibitem{insurance} de Jong P., Heller G., \emph{Generalized Linear Models for Insurance Data}, Cambridge University Press, New York, 2008

\bibitem{rozrzut} naukowiec.org, \emph{Interpretacja wykresów rozrzutu} \url{https://www.naukowiec.org/wiedza/statystyka/interpretacja-wykresow-rozrzutu_769.html} (dostęp: 07.01.2021)

\bibitem{codebook} Our World In Data, \emph{Covid Codebook} \url{https://github.com/owid/covid-19-data/blob/master/public/data/owid-covid-codebook.csv} (dostęp: 09.01.2021)

\bibitem{owid} Our World In Data, \emph{Statistics and Research. Coronavirus Pandemic} \url{https://ourworldindata.org/coronavirus} (dsotęp: 30.11.2020)

\bibitem{dollars} Our World In Data, Ortiz-Ospina E., Molteni M. \emph{What are PPP adjustments and why do we need them?} \url{https://ourworldindata.org/what-are-ppps} (dostęp 31.10.2020)

\bibitem{narazenie} Pawlak M., \emph{Kto jest w największej grupie ryzyka koronawirusa?} \url{https://www.medonet.pl/koronawirus-pytania-i-odpowiedzi/sars-cov-2,kto-jest-w-najwiekszej-grupie-ryzyka-koronawirusa-,artykul,48398767.html} (dostęp: 09.01.2021)

\bibitem{lm} R Documentation, \emph{lm function} \url{https://www.rdocumentation.org/packages/stats/versions/3.6.2/topics/lm} (dostęp: 03.01.2021)

\bibitem{powerTransform} R Documentation, \emph{powerTransform function} \url{https://www.rdocumentation.org/packages/car/versions/3.0-10/topics/powerTransform} (dostęp: 03.01.2021)

\bibitem{experimental} Seltman H., \emph{Experimental Design and Analysis}, \url{http://www.stat.cmu.edu/~hseltman/309/Book/}

\bibitem{cholesky} Taboga M. \emph{Cholesky decomposition, Lectures on matrix algebra.} \url{https://www.statlect.com/matrix-algebra/Cholesky-decomposition}, 2017 (dostęp: 09.01.2021)

\bibitem{rzadka} Wikipedia, \emph{Macierz rzadka} \url{https://pl.wikipedia.org/wiki/Macierz_rzadka} (dostęp: 08.01.2021)

\bibitem{skorygowany} Wikipedia, \emph{Model statystyczny. Skorygowany współczynnik determinacji} \url{https://pl.wikipedia.org/wiki/Model\_statystyczny#Skorygowany\_wsp\%C3\%B3\%C5\%82czynnik\_determinacji} (dostęp: 08.01.2021)

\bibitem{boxcox} Wikipedia, \emph{Przekształcenie Boxa-Coxa} \url{https://pl.wikipedia.org/wiki/Przekszta\%C5\%82cenie\_Boxa-Coxa} (dostęp: 07.01.2021)

\bibitem{bayes} Wikipedia, \emph{Twierdzenie Bayesa} \url{https://pl.wikipedia.org/wiki/Twierdzenie_Bayesa} (dostęp: 08.01.2021)

\bibitem{hdi} Wikipedia, \emph{Wskaźnik rozwoju społecznego} \url{https://pl.wikipedia.org/wiki/Wska\%C5\%BAnik\_rozwoju\_spo\%C5\%82ecznego} (dostęp: 09.01.2021)

\bibitem{r2} Wikipedia, \emph{Współczynnik determinacji} \url{https://pl.wikipedia.org/wiki/Wsp\%C3\%B3\%C5\%82czynnik\_determinacji} (dostęp: 08.01.2021)

\bibitem{prediction} Welham S, Cullis B., Gogel B.,  Gilmour A.R.,  Thompson R, \emph{Prediction in linear mixed models.} Australian \& New Zealand Journal of Statistics. vol. 46.  (2004). p.  325 - 347. 10.1111/j.1467-842X.2004.00334.x. 

\bibitem{complex} Wu L., \emph{Mixed Effects Models for Complex Data}, University of British Columbia, Vancouver, Canada, 2010

\bibitem{szeregi} Zagdański A., Suchwałko A., \emph{Analiza i prognozowanie szeregów czasowych. Praktyczne wprowadzenie na podstawie środowiska R}, PWN, Warszawa 2016

\bibitem{zuur} Zuur A., Ieno E., Walker N.,
Saveliev A., Smith G., \emph{Mixed Effects Models and Extensions in Ecology with R}, Springer, New York 2009




\end{thebibliography}



\listoffigures

\listoftables


\chapter*{Załączniki}
\begin{enumerate}
%\item Oświadczenie o oryginalności pracy i możliwości jej wykorzystania. 
%\item Opinia promotora na temat oryginalności pracy oraz w~sprawie dopuszczenia do obrony pracy dyplomowej.
%\item Potwierdzenie analizy antyplagiatowej.
\item Płyta CD z niniejszą pracą w wersji elektronicznej.
\end{enumerate}




\chapter*{Streszczenie (Summary)}

\bigskip
\bigskip

\begin{center}
  \textbf{\tytul}
\end{center}

Ta praca przedstawia zastosowanie modeli liniowych oraz modeli mieszanych do analizy rozwoju pandemii COVID-19 na świecie. Pierwszy rozdział jest poświęcony teorii matematycznej. Na początku omówione zostają zagadnienia dotyczące modeli liniowych. Następnie te pojęcia są poszerzane o modele liniowe z efektami stałymi i losowymi. Przedstawione zostały problemy takie jak: metody estymacji oraz badanie istotności parametrów modelu, predykcja z modelu mieszanego, jak również interpretacja modelu. W rozdziale drugim przeprowadzono badania na zbiorze danych dotyczącym zachorowań na COVID-19. Sprawdzanych jest dziewięć hipotez, które mają na celu zidentyfikowanie czynników istotnie wpływających na liczbę zachorowań na koronawirusa w różnych krajach. W ostatnim rozdziale zostały przedstawione wnioski wyciągnięte z badań.


\bigskip

\begin{center}
  \textbf{\textit{\tytulangielski}}
\end{center}



\selecthyphenation{english}
{\it
This paper presents the use of linear and mixed-effects models in analysis of the development of the COVID-19 pandemic worldwide. The first chapter is dedicated to mathematical theory. At the beginning, issues concerning linear models are discussed. Then these concepts are extended with linear models with fixed and random effects. Problems such as: estimation methods and testing the significance of model parameters, prediction from a mixed-effects model, as well as model interpretation are presented. In the second chapter, a study was conducted on the COVID-19 dataset. Nine hypotheses are tested to identify the factors significantly affecting the number of coronavirus cases in different countries. The last chapter presents the conclusions drawn from the study.

}

\end{document}

